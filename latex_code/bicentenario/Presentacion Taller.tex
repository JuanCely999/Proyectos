\documentclass[11pt]{beamer}
%\useoutertheme{split} % tema exterior si
%\definecolor{naranja}{cmyk}{0,0.5,1,0}
\usetheme{CambridgeUs}
%\usecolortheme{crane}% tema de color siwhale
\usepackage[utf8]{inputenc}
\usepackage[spanish]{babel}
\usepackage{amsmath}
\usepackage{amsfonts}
\usepackage{amssymb}
\usepackage{graphicx}
\usepackage{ragged2e}
\usepackage{tikz}
\usepackage{hyperref}
\author[Juan Pablo Cely]{Juan Pablo Cely}
\title[Análisis bibliométrico \\ \& \\ Análisis textual]{Análisis bibliométrico \\ \& \\ Análisis textual}
\setbeamercovered{transparent} 
\setbeamertemplate{navigation symbols}{} 
\institute[]{TALLER DE REFLEXIÓN TEÓRICA Y METODOLÓGICA "UNA HISTORIA POR HACER"}
 
\date[]{2019} 

\titlegraphic{\includegraphics[width=1.5cm]{Figura/uptc}}
%\titlegraphic{\includegraphics[width=2cm]{Figuras/UMNG.png}}
\begin{document}

\begin{frame}
\titlepage
\end{frame}

%\begin{frame}
%\tableofcontents
%\end{frame}

%%%%%%%%%%%%%%%%%
%%%se puede omitir %%%%%%%%%
%\begin{figure}
%\includegraphics[width=9.0cm]{R/temas}
%\label{fig:ejemplo}
%% \scriptsize \textbf{Nota:}....	\label{anos}
%\end{figure}

\begin{frame}
\frametitle{Contenido}
\begin{enumerate}
\item Parte l: Introducción a Web of Science \& R
\begin{enumerate} [a]
\item ¿Qué es Web of Science?
\item ¿Qué es R?
\item Pagina de R
\item ¿Qué es Rstudio?
\end{enumerate}
\item Parte ll: Análisis bibliométrico
\begin{enumerate} [a]
\item Publicaciones por año y en promedio
\item Redes de cooperación
\item Redes de palabras claves
\end{enumerate}
\item Parte lll: Análisis textual
\begin{enumerate} [a]
\item Análisis de sentimientos
\item Nubes de palabras 
\end{enumerate}
\end{enumerate}
\end{frame}


\begin{frame}
\frametitle{Contenido}
\begin{enumerate}[<i->]
\item<1-> Parte l: Introducción a Web of Science\& R
\begin{enumerate} [a]
\item<1-> ¿Qué es Web of Science?
\item<1-> ¿Qué es R? 
\item<1-> Pagina de R
\item<1->  ¿Qué es Rstudio?
\end{enumerate}
\item Parte ll: Análisis bibliométrico
\begin{enumerate} [a]
\item Publicaciones por año y en promedio
\item Redes de cooperación
\item Redes de palabras claves
\end{enumerate}
\item Parte lll: Análisis textual
\begin{enumerate} [a]
\item Análisis de sentimientos
\item Nubes de palabras 
\end{enumerate}
\end{enumerate}
\end{frame}



\section{Parte l: Introducción a Web of Science  \& R}
\begin{frame}
\frametitle{Parte l: Introducción a Web of Science}
\framesubtitle{¿Qué es Web of Science?}
\begin{itemize}
\item Es una plataforma on-line que contiene Bases de Datos de información bibliográfica y recursos de análisis de la información
\item Su contenido es multidisciplinar y proporciona información de alto nivel académico y científico.
\end{itemize}
\begin{figure}

\includegraphics[width=3.0cm]{Figura/wos}
\centering
  \label{figure:Consumo}  
  \end{figure}
\end{frame}

\begin{frame}
\frametitle{Parte l: Introducción a R}
\framesubtitle{¿Qué es R?}
\begin{itemize}
\item R es un lenguaje de programación para análisis de datos y elaboración de gráficos
\item Software libre, corre en diferentes sistemas operativos.
\item Interacción por linea de comandos (reglas de sintaxis).
\item https://www.r-project.org/
\end{itemize}
\end{frame}

\begin{frame}
\frametitle{Parte l: Introducción a R}
\framesubtitle{Pagina de R}
\begin{figure}

\includegraphics[width=10.0cm]{Figura/pag}
\centering
  \label{figure:Consumo}  
  \end{figure}
\end{frame}

\begin{frame}
\frametitle{Parte l: Introducción a R}
\framesubtitle{¿Qué es Rstudio?}
\begin{figure}

\includegraphics[width=13.0cm]{Figura/rstudio}
\centering
  \label{figure:Consumo}  
  \end{figure}
\end{frame}



\begin{frame}
\frametitle{Parte l: Introducción a R}
\framesubtitle{¿Qué es Rstudio?}
\begin{itemize}
\item Console: Ejecuta comandos y muestra los resultados.
\item Editor: Aca se escribe lo que se quiere ejecutar (script)
\item History - Environment
\item Files - Plots - Packages - Help - Viewer
\item https://www.rstudio.com/
\end{itemize}
\end{frame}


\begin{frame}
\frametitle{Contenido}
\begin{enumerate}[<i->]
\item Parte l: Introducción a Web of Science\& R
\begin{enumerate} [a]
\item ¿Qué es Web of Science?
\item ¿Qué es R? 
\item Pagina de R
\item  ¿Qué es Rstudio?
\end{enumerate}
\item<1-> Parte ll: Análisis bibliométrico
\begin{enumerate} [a]
\item<1-> Publicaciones por año y en promedio
\item<1-> Redes de cooperación
\item<1-> Redes de palabras claves
\end{enumerate}
\item Parte lll: Análisis textual
\begin{enumerate} [a]
\item Análisis de sentimientos
\item Nubes de palabras 
\end{enumerate}
\end{enumerate}
\end{frame}



\section{Parte ll: Análisis bibliométrico}
\begin{frame}
\frametitle{Parte ll: Análisis bibliométrico}
\framesubtitle{Publicaciones por año y en promedio}
\begin{figure}
{\includegraphics[width=0.47\textwidth]{Figura/ASP}} 	{\includegraphics[width=0.47\textwidth]{Figura/AACY}}	\\
%	 \scriptsize \textbf{Nota:}....	\label{anos}
	\end{figure}
\end{frame}

\begin{frame}
\frametitle{Parte ll: Análisis bibliométrico}
\framesubtitle{Redes de cooperación}
\begin{figure}
{\includegraphics[width=0.47\textwidth]{Figura/MPP}} 	{\includegraphics[width=0.47\textwidth]{Figura/RE}}	\\
%	 \scriptsize \textbf{Nota:}....	\label{anos}
	\end{figure}
\end{frame}



\begin{frame}
\frametitle{Parte ll: Análisis bibliométrico}
\framesubtitle{Redes de cooperación}

\begin{figure}
{\includegraphics[width=0.4\textwidth]{Figura/red}} 	{\includegraphics[width=1.0\textwidth]{Figura/red1}}	\\
%	 \scriptsize \textbf{Nota:}....	\label{anos}
	\end{figure}
\end{frame}

\begin{frame}
\frametitle{Parte ll: Análisis bibliométrico}
\framesubtitle{Redes de cooperación}

\begin{figure}
 	{\includegraphics[width=1\textwidth]{Figura/red3}}	\\
%	 \scriptsize \textbf{Nota:}....	\label{anos}
	\end{figure}
\end{frame}

\begin{frame}
\frametitle{Parte ll: Análisis bibliométrico}
\framesubtitle{Redes de cooperación}

\begin{figure}
{\includegraphics[width=0.3\textwidth]{Figura/red4}} 	{\includegraphics[width=0.7\textwidth]{Figura/red5}}	\\
%	 \scriptsize \textbf{Nota:}....	\label{anos}
	\end{figure}
\end{frame}

\begin{frame}
\frametitle{Parte ll: Análisis bibliométrico}
\framesubtitle{Redes de palabras claves}

\begin{figure}
{\includegraphics[width=0.4\textwidth]{Figura/dendo}} 	{\includegraphics[width=0.5\textwidth]{Figura/MCA2}}	\\
%	 \scriptsize \textbf{Nota:}....	\label{anos}
	\end{figure}
\end{frame}



\begin{frame}
\frametitle{Contenido}
\begin{enumerate}[<i->]
\item Parte l: Introducción a Web of Science\& R
\begin{enumerate} [a]
\item ¿Qué es Web of Science?
\item ¿Qué es R? 
\item Pagina de R
\item  ¿Qué es Rstudio?
\end{enumerate}
\item Parte ll: Análisis bibliométrico
\begin{enumerate} [a]
\item Publicaciones por año y en promedio
\item Redes de cooperación
\item Redes de palabras claves
\end{enumerate}
\item<1-> Parte lll: Análisis textual
\begin{enumerate} [a]
\item<1-> Análisis de sentimientos
\item<1-> Nubes de palabras 
\end{enumerate}
\end{enumerate}
\end{frame}


\section{Parte lll: Análisis textual}
\begin{frame}
\frametitle{Parte lll: Análisis textual}
\framesubtitle{Análisis de sentimientos}
El análisis de sentimientos, también llamado minería de opinión, es en una serie de técnicas informáticas que se utilizan para clasificar automáticamente un texto con un sentimiento positivo o negativo (Pang y Lee 2008) e incluso con alguna de las emociones básicas (Plutchik 1980).
\end{frame}



\begin{frame}
\frametitle{Parte lll: Análisis textual}
\framesubtitle{Análisis de sentimientos}
\begin{figure}
{\includegraphics[width=1\textwidth]{Figura/libro3}} 	\\
%	 \scriptsize \textbf{Nota:}....	\label{anos}
	\end{figure}
\end{frame}

\begin{frame}
\frametitle{Parte lll: Análisis textual}
\framesubtitle{Análisis de sentimientos}
\begin{figure}
{\includegraphics[width=1\textwidth]{Figura/libro2}} 	\\
%	 \scriptsize \textbf{Nota:}....	\label{anos}
	\end{figure}
\end{frame}


\begin{frame}
\frametitle{Parte lll: Análisis textual}
\framesubtitle{Análisis de sentimientos}
\begin{figure}
{\includegraphics[width=1\textwidth]{Figura/general}} 	\\
%	 \scriptsize \textbf{Nota:}....	\label{anos}
	\end{figure}
\end{frame}

\begin{frame}
\frametitle{Parte lll: Análisis textual}
\framesubtitle{Análisis de palabras}
\begin{figure}
{\includegraphics[width=1\textwidth]{Figura/wc3}} 	\\
%	 \scriptsize \textbf{Nota:}....	\label{anos}
	\end{figure}
\end{frame}

\end{document}