\chapter*{Resumen} \label{resumen}
\addcontentsline{toc}{chapter}{Resumen}
Este documento describe la dinámica de la inflación en las economías departamentales de Colombia en la última década, utilizando el marco de la curva de Phillips neokeynesiana (NKPC). Se encuentran diferencias en la formación de la inflación y evidencia que la NKPC permite describir baja probabilidad de cambios en los precios para departamentos principalmente de la zona central del país. Además, se evidencia que la inflación esperada y la brecha de los costos marginales  son impulsores de la inflación en departamentos que menos han avanzado en términos de apertura económica, ocupación y lo concerniente a la participación del sector primario en la economía regional. Para complementar, se observa que la mayor probabilidad de que los precios permanezcan en el tiempo están influenciados por la baja competitividad, aumento de los ingresos por habitante y la incidencia creciente que puede lograr el sector terciario.\\
 
En particular, los coeficientes de forma reducida y estimaciones implícitas de los parámetros estructurales del modelo, apoyan la importancia que tiene la inflación esperada sobre la formación de precios, mientras que el papel de la inflación rezagada (persistencia de la inflación) también es estadísticamente importante, pero con menor incidencia. Esta persistencia de la inflación podría ser un reflejo de rigideces estructurales  que reducen la capacidad de una región, en relación con otras, para adaptarse a diferentes choques. Estas diferencias en los procesos y mecanismos de inflación entre departamentos tienen implicaciones importantes para la conducción de política monetaria en Colombia.
\\\\

\textbf{Palabras claves:} curva de Phillips neokeynesiana, dinámica de inflación departamental, precios rígidos.

\chapter*{Introducción}
\addcontentsline{toc}{chapter}{Introducción}

El modelo de inflación objetivo puesto en funcionamiento por el Banco Central de la República de Colombia desde el año 1999, tiene como uno de sus pilares la curva de Phillips Neokeynesiana o Nueva curva de Phillips (NKPC, \textit{siglas en inglés}), ahora constituida por la relación entre inflación y la brecha del producto.  Sin embargo, estudios recientes han evidenciado que no es la brecha del producto la que incide en la inflación,  sino la brecha de los costos marginales \citep{gali1999inflation,gali2002new,rumler2007estimates,ramos2008inflation}.\\

Por lo anterior, \cite{gali1999inflation} encuentran que las empresas intentan mantener un beneficio fijo sobre el costo marginal, pero si este margen sobre los costos empieza a declinar, entonces las firmas intentan de nuevo fijar sus precios provocando con ello inflación. En este sentido, al asumir la nueva curva de Phillips se presume de ciertas rigideces en los precios y que estos evolucionan acorde con las decisiones de los productores o por el lado de la oferta.\\

Por esta razón, los neokeynesianos consideran que la dinámica inflacionaria se explica no solo en un fenómeno de demanda, sino también por el lado de la oferta con base en los costos de producción. Pues bien, tal enfoque concibe mecanismos distintos a través de los cuales, las dos variables -inflación y costos marginales- se relacionan otorgándole distintos grados de importancia a las variables intermedias involucradas en su análisis y establece diferentes direcciones en la causalidad de una sobre otra. Entre ellas se encuentran: a) efecto de persistencia (rezagos de la inflación), b) factores de demanda (procedentes de los desequilibrios reales) y c) choques de oferta (cambios en los precios procedentes de factores climáticos que recaen sobre los precios de los alimentos, cambios en la regulación que afectan los precios de servicios públicos o del petróleo en el exterior) \citep{gordon1997time}. \\

En este contexto, suponer que las empresas en un entorno regional  tienen la capacidad de fijar el precio y mantenerlo fijo por algún tiempo debido a las rigideces nominales, dará origen a hallazgos en la dinámica inflacionaria.  Las diferencias en el proceso de formación de inflación en los departamentos  será relevante al momento de determinar el grado de  efectividad de la política monetaria, es decir, debido a que cada región cuenta con estructuras económicas distintas, la política monetaria podría tener efectos desiguales a partir de  variables como:   a) el grado de industrialización y el tipo de especialización o diversificación que tenga la industria de una región, b) el desarrollo y la profundidad financiera, c) la posición neta en el sistema financiero, y d) la posición neta, en el comercio exterior  \citep{romero2008transmision}. \\

Así mismo, bajo la existencia de competencia imperfecta entre las empresas y la persistencia de los precios en el tiempo, dará lugar a la participación de la política económica como herramienta estabilizadora ante los ciclos adversos que enfrenta la economía colombiana. En este sentido, la estimación de la NKPC se realiza pensando en que la dinámica inflacionaria subnacional ha recibido menos atención en la literatura y en los análisis económicos y, sobre todo, en las economías emergentes como la colombiana.\\

El desarrollo del documento comienza con esta introducción, seguido por el la revisión de literatura (capítulo \ref{cap1}) donde se hace una repaso teórico que envuelve el desarrollo de la curva de Phillips, posteriormente se realiza la derivación de la NKPC y se culmina con las discusiones entorno a la NKPC. En el capítulo \ref{cap2} se describen detalladamente el comportamiento y la tendencia descriptiva de las principales variables que intervienen en el modelo. Para el capítulo \ref{cap3}, se procede a estimar la NKPC para los departamentos de Colombia junto con los resultados obtenidos. Finalmente, se esbozan las principales conclusiones del trabajo.

\chapter{Revisión de literatura} \label{cap1}
%\addcontentsline{toc}{chapter}{\numberline{}Capitulo 1}\\\\
\section{Contexto teórico} \label{sec1}

La curva de Phillips neokeynesiana es el resultado de la evolución teórica a partir de la regularidad empírica encontrada inicialmente entre salarios y desempleo, que posteriormente se fundamentó en un \textit{trade-off} entre inflación y la brecha del producto, elemento clave para el modelo de inflación objetivo.\footnote{La brecha del producto es la desviación porcentual que tiene el producto de su componente de equilibrio. Esta sirve como un indicador del ciclo económico.}  La NKPC revela novedades en las causas y persistencia de la inflación al adicionar la decisión de fijar precios dentro de un problema explícito de optimización individual, lo cual permite vincular la relación a corto plazo entre la inflación y alguna medida de la actividad real. Sin embargo, los adelantos teóricos no estaban lo suficientemente comprobados por la evidencia empírica, sería después del trabajo de \cite{gali1999inflation} que daría lugar al punto de partida hacia la estimación de la NKPC.\footnote{Inicialmente, el término de la NKPC fue usado por \cite{roberts1995new}.} \\

En principio, la curva de Phillips se dio a partir del trabajo de \cite{phillips1958relation}, el cual evidenció una relación inversa entre la inflación de los salarios y la tasa de desempleo para el Reino Unido entre los años 1861 hasta 1957. Más tarde, \cite{lipsey1960relation}   percibió la inflación de los salarios como una \textit{proxy} de la inflación de los precios, lo que daría lugar a una explicación teórica consistente a los resultados encontrados por \cite{phillips1958relation}, añadiendo que el origen de la inflación era producto de exceso de demanda. En seguida, \cite{samuelson1960analytical}  tomarían estos resultados para realizar recomendaciones de política según el \textit{trade-off} entre inflación y producto.\\

Así mismo, la introducción de la curva de Phillips llena un vacío en la teoría keynesiana y se convierte en el análogo de la teoría de salarios y empleo en Keynes \citep{tobin1972inflation}. La ausencia de una relación entre empleo, salarios e inflación no estaba ausente en la teoría general de Keynes, sino en la interpretación de Keynes hecha por \cite{hicks1937mr} , y era la ausencia de una relación entre dichas variables en el modelo IS-LM, lo que debilitaba un poco a este modelo para dar cuenta de los hechos. En este sentido, la curva de Phillips implicaría que un nivel más bajo de desempleo se puede lograr al costo de una inflación más alta.\footnote{Para la mayoría de los macroeconomistas de la segunda mitad del siglo XX, el punto de partida no es la Teoría general misma sino su versión propuesta desde 1937 por Hicks: el modelo IS-LM, el cual invita a llenar lo que es percibido como un vacío: aquel de los fundamentos microeconómicos de la macroeconomía keynesiana. La curva de Phillips anexaría la relación entre empleo, salario e inflación en el modelo macroeconómico inicial.} \\

Después de la aceptación del \textit{trade-off} entre inflación y desempleo, las críticas a la curva de Phillips no se hicieron esperar por parte de \cite{friedman1968role}, quien adicionó expectativas a la inflación, dejando viva la curva de Phillips y su incidencia solamente a corto plazo. Las criticas  de \cite{phelps1967phillips} y \cite{friedman1968role}   se fundamentan con base en la experiencia inflacionaria de los años 60 y principios de los 70 del siglo XX, debido a la expansión monetaria para financiar la guerra de Vietnam por parte de los Estados Unidos. En ese escenario surgió la denominada estanflación que para en ese entonces se convertiría en mayor desempleo con aumento en inflación.\\

Adicionalmente, los autores argumentaron que la curva de Phillips no consideraba como agentes racionales a las empresas y a los trabajadores, lo cual desconocía el costo de vida al momento de acordar los salarios del trabajador por parte del empleador. Por lo tanto, este hecho implica que los trabajadores y las empresas debían acordar el salario de hoy con base en sus expectativas de inflación. Todo esto llevaría a proponer una curva de Phillips aumentada con expectativas por parte de Friedman y Phelps, presentando la inflación del salario nominal en función de las expectativas de inflación de los precios y de la tasa de desempleo.\\

La curva de Phillips aumentada o a largo plazo vertical mostró que la relación entre la inflación y desempleo solo existiría si los negociadores de los salarios predecían sistemáticamente una inflación inferior a la efectiva y que era improbable que lo hicieran indefinidamente. Más tarde sería aceptada por los mismos keynesianos \citep{blinder1997there,mankiw1991reincarnation}. En este sentido, los autores demuestran que la tasa de desempleo no podría mantenerse hasta cierto nivel, la cual se llamó "tasa natural de desempleo". Esta es explicada como la tasa de desempleo necesaria para mantener constante la tasa de inflación, y posteriormente sería conocida como la tasa de desempleo no aceleradora de la inflación o NAIRU (\textit{Non-Accelerating Inflation Rate of Unemployment}).\\

En los años 80 del siglo XX, Estados Unidos presenció inflación baja acompañado por un incremento temporal en el desempleo que alcanzaría el 11\% (reacción no aceleradora de la inflación). Posteriormente, desde 1983 en adelante, el desempleo cayó en 6\% hasta llegar al año 1987 con inflación del 4\%, lo cual lleva a contradecir la relación inversa entre desempleo e inflación, elementos contenidos en la curva de Phillips. \\

Así mismo, la tasa natural se empieza a tener en cuenta por parte de los bancos centrales después de la gran inflación ocurrida a finales de la década de los 60 y comienzos de los 70 del siglo XX. \cite{friedman1968role}   sugirió mantener una inflación baja, al tiempo que el nivel de empleo incrementara a largo plazo, en cuanto se tratara de reducir el desempleo más allá de la tasa natural tan solo resultaría en una inflación mayor. Por lo tanto, la postura de Friedman fue permitir que la política monetaria actuara de manera automática sin la interferencia del Estado.\\

Sin embargo, \cite{lucas1973some} -referente de los nuevos clásicos- estaría de acuerdo con la curva de Phillips a corto plazo si los agentes no pueden anticipar la inflación  debido a problemas de información, es decir, si en el caso opuesto, existen perturbaciones no anunciadas o no esperadas, el dinero puede tener efectos reales, violando la neutralidad sólo en el corto plazo, por lo tanto, la autoridad siempre estaría tentada a explotar el \textit{trade-off} entre inflación y desempleo. Entonces, el modelo clásico sería compatible con la curva de Phillips si el supuesto de la información completa es abandonado, combinado a la hipótesis de la tasa natural de Phelps y Friedman, con el supuesto que los mercados se vacían y la hipótesis de las expectativas racionales \citep{snowdon1994modern}. \\

Por consiguiente, el corto plazo se convertiría en uno de los principales focos de debate en curso sobre la política de demanda agregada, donde los problemas cruciales parecen depender de la curva de Phillips y su propiedad dinámica \citep{taylor1979staggered}. Por lo anterior, entre los temas centrales para la macroeconomía a corto plazo entraría el estudio de la naturaleza de la dinámica inflacionaria. En respuesta, importantes modelos teóricos surgen en su comprensión, originando investigaciones preliminares como \cite{calvo1983staggered}, \cite{taylor1980aggregate} y \cite{fischer1977long}. Los autores enfatizan la participación de la fijación escalonada de salarios y precios nominales por individuos y empresas con visión futura en la explicación de la inflación.\\
 
 Los modelos teóricos planteados aparecen para sentar bases a la competencia imperfecta y las rigideces nominales de variables como los precios y nivel de salarios a partir de los fundamentos microeconómicos. Sin embargo, estos aspectos ya habían sido conocidos en la teoría keynesiana al argumentar que de acuerdo a las fluctuaciones de la producción surgen en gran medida oscilaciones en la demanda agregada nominal, es decir, los cambios en la demanda tienen efectos reales debido a que los salarios nominales y los precios nominales son rígidos. No obstante, estos argumentos simplemente serian suposiciones en lugar de ser explicados como en los modelos de desequilibrio,\footnote{Las demostraciones teóricas de que los modelos keynesianos pudieron ser considerados por la microeconomía  no constituyen prueba alguna de que las teorías keynesianas fueran correctas. Por lo tanto, implicaciones empíricas en modelos de rigideces nominales han resultado ser débiles \citep{summers1991should}.}  o introducido a través de suposiciones teóricamente arbitrarias sobre los contratos laborales \citep{ball1988new}.\footnote{Para modelos de desequilibrio ver en \cite{barro1971general} y contratos laborales en \cite{fischer1977long}.} \\
 
Posteriormente, aparecerían autores de enfoque neokeynesiano (nueva síntesis neoclásica)  para desarrollar la microfundamentación de la curva de Phillips \citep{woodford1998control,goodfriend1997new,clarida1999science}, derivados en modelos macroeconómicos no sujetos a la critica de \cite{lucas1976econometric}.\footnote{La critica de \cite{lucas1976econometric} señalaba que cuando se trata de predecir los efectos de un gran cambio de política -como el que estaba considerando la Reserva Federal de los Estados Unidos (FED) en ese momento- puede ser muy engañoso considerar dadas las relaciones calculadas a partir de datos pasados, es decir, la curva de Phillips  suponía que los encargados de fijar los salarios seguirían esperando que la futura inflación fuera igual que la pasada.} Estos incorporan imperfecciones en los mercados y tipos de rigideces nominales,  tales como costos de menú, costos de ajuste en la inversión, precios escalonados, entre otros. De esta manera, otros trabajos como \cite{fuhrer1995inflation}, \cite{yun1996nominal}, \cite{king1999should} darían desarrollo y aplicación de la NKPC en diferentes modelos macroeconómicos a nivel teórico.\\
 
Por lo anterior, la NKPC se distinguiría -con respecto de la curva de Phillips tradicional- por ser microfundamentada y adicionar el efecto que tienen las expectativas racionales en la decisión que toma cada una de las empresas en el momento de fijar los precios. La NKPC toma importancia por ser elemento fundamental en el diseño de modelos de pronóstico de inflación de los bancos centrales en países que operan bajo inflación objetivo, por la razón que al considerar agentes con comportamientos racionales (\textit{forward-looking}), el efecto de la política monetaria sobre la inflación sería mayor, dado que los agentes ajustan sus precios con base en la senda esperada del producto y de los precios. \\

Así mismo, \cite{ball1988new} plantea que el nivel de precios se ajusta lentamente con el tiempo a un choque nominal, dado que la velocidad del acople depende de la frecuencia de ajuste de precios por empresas individuales, que a su vez se deriva de la maximización de las ganancias. La incidencia de los precios rígidos subyace en la medida que  pueden ser tanto privados eficientes como socialmente ineficientes. Por lo tanto, el ciclo económico resulta del ajuste subóptimo de los precios en respuesta a un choque de demanda. En este sentido, si la política puede estabilizar la demanda agregada, a su vez, podrá mitigar la pérdida social debido a este ajuste subóptimo \citep{mankiw1985small}. En respuesta, uno de los pioneros de la NKPC como lo es \cite{gali2010new}, manifestaría que las rigideces nominales sería el elemento clave de los modelos neokeynesianos y  la principal causa de la no neutralidad de la política monetaria.\\

En suma, los modelos neokeynesianos aportarían una nueva perspectiva sobre la naturaleza de la dinámica de la inflación. \cite{gali2002new} enfatiza las siguientes características: en primer lugar, por la condición prospectiva que tiene la inflación, sostiene que los precios son fijados por empresas que enfrentan limitaciones en la frecuencia con la que pueden ajustar el precio de los bienes que producen. Por lo tanto, los precios establecidos se caracterizarán por permanecer vigentes durante más de un periodo después de su imposición. Estas empresas encuentran un óptimo, dado que toman decisiones de fijar precios actuales según condiciones futuras de costo y demanda, ya que el nivel de precios agregados resulta ser producto de decisiones actuales de fijación de precios, y a su vez, un componente importante a futuro para el caso de la inflación. Por otra parte, los modelos neokeynesianos destacan el papel que desempeñan las variaciones en los márgenes, es decir, la incidencia de los costos marginales como fuente de cambio en la inflación agregada, en este sentido, es explicado por los intentos periódicos de las empresas para corregir desalineación entre los márgenes reales y no deseados.  \\

Estas propiedades se reflejan en la llamada NKPC, y sería después del trabajo de \cite{gali1999inflation} quienes por medio de la evidencia empírica, dan inicio a la estimación de un modelo estructural de la NKPC. Allí capturan la persistencia de inflación incurrida por la rigidez nominal bajo expectativas racionales para Estados Unidos (1960-1997). En seguida, \cite{gali2001european}  estudiarían a la Zona Euro (1970-1998). Por su  relevancia, la NKPC también sería abordada en el ámbito subnacional como en los trabajos de \cite{ha2003causes} y \cite{mehrotra2010modelling}, los cuales analizan principalmente la dinámica inflacionaria de las provincias de China. Por lo tanto, a partir de la teoría económica, estos modelos estructurales han ayudado a comprender el efecto que tiene la inflación frente a la política monetaria de algunos países, mientras que permite evaluar la incidencia de problemas estructurales e instituciones débiles en un escenario de persistencia de inflación.\footnote{La figura \ref{A2} recoge la red de colaboración de trabajos en la NKPC, donde se aprecia que \cite{gali1999inflation} y \cite{calvo1983staggered} son los más relevantes (Para mayor información ver apéndice \ref{apendicea} ).} 

\section{Derivación de la curva de Phillips neokeynesiana}\label{sec2}
El modelo de expectativas racionales  \citep{lucas1976econometric,sargent1976rational} dio paso a la microfundamentación neoclásica para los modelos macroeconómicos, desconociendo la curva de Phillips y la base de la economía keynesiana, es decir, la suposición de que la política monetaria podría afectar sistemáticamente la producción incluso a corto plazo. Por lo tanto,  los modelos neokeynesianos aparecerían para considerar la existencia de precios rígidos, el uso de ecuaciones desde fundamentos microeconómicos que  permitieran obtener modelos estructurales a partir del objetivo de agentes. Además, estos modelos incorporan expectativas racionales bajo un entorno de competencia monopolística (las firmas diferencian su producto), por lo que se pueden fijar precios.\\

Los precios rígidos se convierten en la principal razón microeconómica, dado que permite establecer períodos en donde los factores de producción como la mano de obra, se subutilizan, con una producción agregada por debajo de su llamado nivel potencial. Así mismo, un aumento en el stock de dinero puede generar un incremento a corto plazo en el poder adquisitivo real, y a su vez, permite el impulso de la producción real bajo un entorno de precios rígidos. Por otra parte, suponer rigideces de precios implica que no todos los mercados se están ajustando al instante y la producción agregada puede estar por debajo de los que se obtendría con precios flexibles  \citep{ball1988new} .\\

Este enfoque moderno que presenta expectativas racionales y alguna forma de microfundamentación se conoce como macroeconomía neokeynesiana. En este sentido, el siguiente apartado describe uno de los modelos clave neokeynesianos como lo es la NKPC, y se explora sus implicaciones para el comportamiento de la inflación y alguna variable de la actividad económica. Por último, se discute sobre la relación entre los costos marginales y la brecha del producto para dar explicación a la rigidez de precios y así mismo evaluar el mejor estimador de la NKPC.

\subsubsection*{Rigideces de precios a la Calvo}\label{secrig}
La derivación de la NKPC parte de suponer que las empresas se encuentran en competencia monopolística y cuentan con algún tipo de restricción en el ajuste de precios, es decir, las empresas establecen los precios a través de una regla de tiempo dependiente.\footnote{ La regla de tiempo dependiente se entiende como el grado de autonomía que tienen las empresas para cambiar sus precios y la existencia de políticas periódicas de revisión de precios. A pesar de que este escenario sea similar al modelo de contratos escalonados propuesto por \cite{taylor1980aggregate}, la diferencia subyace en que la decisión de fijación de precios evoluciona según el  problema de maximización de ganancias de un competidor monopolista.}  La formulación conocida como fijación de precios de \cite{calvo1983staggered} permite simplificar el problema de agregación según la fijación de precios dependientes al tiempo, lo cual evita realizar un seguimiento de los historiales de precios de las empresas. En este sentido, esta forma de rigidez de precios permite suponer que en cualquier periodo, la empresa tiene una probabilidad fija de $1-\theta$ para ajustar su precio en ese periodo, así mismo, con una probabilidad $\theta$ debe mantener su precio sin cambios en el tiempo promedio durante su fijación dado por $\sum_{k=0}^{\infty} (\theta\beta)^{k-1}=\frac{1}{1-\theta}$. Por ejemplo, para \cite{galvis2010estimacion}, con $\theta=0,807$, el 80\% de las empresas en Colombia mantienen los precios fijos en el tiempo durante cinco trimestres y aproximadamente el 20\% de las empresas fijan su precio al instante.\footnote{Para tener en cuenta que en el caso que la flexibilidad de precios sea $\theta=0$, la empresa ajusta su precio al instante, por lo tanto, solo el futuro es relevante cuando hay rigidez de precios, es decir, $\theta>0$. }  \\ 

Se tiene entonces que las empresas van a procurar maximizar una función de beneficios sujeta a la restricción en el ajuste de precios, sin poder incidir en los precios cada vez que lo deseen. De ese modo, la adaptación de esta conducta aparece de la siguiente manera. Se tiene que $p_{t+k}^*$ es el logaritmo del precio óptimo que la empresa fijaría en el período $t+k$ en el caso que no existiera rigideces, y $z_{t}$ si el precio se intenta fijar en $t$.  Dado el precio óptimo del siguiente período, la empresa va a tratar de minimizar sus desviaciones a partir de la siguiente función de pérdidas:
 \begin{equation}\label{1}
L=\sum_{k=0}^{\infty} (\theta\beta)^{k}E_{t}(z_{t}-p_{t+k}^*)^{2}
\end{equation}
%\eqref{1}
Sea $E_{t}(z_{t}-p_{t+k}^*)^{2} $ las pérdidas esperadas de beneficios de la empresa en el tiempo $p_{t+k}^*$, dado que las rigideces de ese periodo impiden el ajuste del precio óptimo.\footnote{La función cuadrática aproxima a una función de ganancia más general, aunque lo relevante aquí es la perdida de ganancia  de no contar con rigideces de precios para $z_{t}$.}  Así mismo, la sumatoria  $\sum_{k=0}^{\infty}$ expresa que el precio establecido tendrá implicaciones a futuro. En el caso de $\beta$, se considera como el factor de descuento subjetivo,  en este sentido, $\beta<1$ le dará mas peso a las pérdidas de hoy que a las  pérdidas futuras, a la vez que serán descontadas con $(\theta\beta)^{k}$. Por lo anterior, la adición de $\theta$ permite descontar las pérdidas según la probabilidad en que las empresas no mantengan el precio fijo hasta el siguiente período.

\subsubsection*{Precio óptimo}\label{secpre}
Para encontrar el valor óptimo de $z_{t}$, es decir, el precio elegido por las empresas que pueden establecer, cada uno de los términos con la variable $z_{t}$ ($(z_{t}-p_{t+k}^*)^{2} $) se diferencia con $z_{t}$ para definir la suma de la derivación que será igual a cero. Ahora, se continua con la optimización de la función de pérdida.\\
 
 Condiciones de primer orden:
\begin{equation}\label{2}
L(z_{t})=2\sum_{k=0}^{\infty} (\theta\beta)^{k}E_{t}(p_{t+k}^*)=0
\end{equation}
En seguida, se separa en dos términos la ecuación \eqref{2}:
\begin{equation}\label{3}
\left[\sum_{k=0}^{\infty}(\theta\beta)^{k} \right] z_{t}=\sum_{k=0}^{\infty}(\theta\beta)^{k}E(p_{t+k}^*)
\end{equation}
Usando la suma geométrica para resolver el lado izquierdo de la ecuación \eqref{3}, se tiene que:
\begin{equation}\label{4}
\left[\sum_{k=0}^{\infty}(\theta\beta)^{k} \right]=\frac{1}{1-\theta\beta}
\end{equation}
Reescribiendo, se obtiene la forma siguiente:
\begin{equation}\label{5}
\frac{1}{1-\theta\beta}=\sum_{k=0}^{\infty}(\theta\beta)^{k}E(p_{t+k}^*)
\end{equation}
Resolviendo la ecuación \eqref{5}, se tiene que el precio óptimo es:
\begin{equation}\label{6}
z_{t}=(1-\theta\beta)\sum_{k=0}^{\infty}(\theta\beta)^{k}E(p_{t+k}^*)
\end{equation}
El precio óptimo señala que la empresa establece su precio similar a un promedio ponderado de los precios que hubiera esperado establecer en el futuro si no hubiera rigideces de precios. Por lo tanto, como no puede cambiar el precio en cada período, la empresa opta por mantenerse cerca del promedio del precio correcto o sin rigideces \citep{whelan2009lecture}. 

\subsubsection*{Costos marginales y \textit{Mark-up}}\label{seccioncm}
En la incertidumbre  del precio óptimo sin fricción ($p_{t}^*$), se asume que la estrategia de fijación de precios óptimos de la empresa sin fricciones implicaría establecer los precios como un \textit{mark-up} fijo sobre el costo marginal:\footnote{\textit{Mark-up} o margen, es un índice económico aplicado sobre el coste de producción e distribución de un producto o servicio para definir el precio.}
 \begin{equation}\label{9}
P_{t}^*=\left( \frac{\epsilon}{\epsilon-1}\right) E_{t-1}\frac{N_{t}^{j}W_{t}}{\alpha Y_{t}^{j}}
\end{equation}
Sea $\epsilon$ la elasticidad del precio de la demanda a la que se enfrenta la firma en el mercado y $(\frac{\epsilon}{\epsilon-1})=\mu$ el \textit{mark-up} sobre los costos marginales puestos por la empresa.\footnote{Tener en cuenta  que se parte inicialmente de una función de producción Cobb-Douglas de tipo $ Y_{t}^{j}=A(N_{t}^{j})^{\alpha}$, sin emplear el capital como factor de producción. La completa derivación del  \textit{mark-up} y los costos marginales se pueden apreciar en el apéndice \ref{apendiceb}.} Los costos marginales ($\frac{N_{t}^{j}W_{t}}{\alpha Y_{t}^{j}}$) son constituidos por el nivel de salarios pagados ($W_{t}$) al contratar $N_{t}^{j}$ empleados necesarios para la producción sobre la cantidad de demanda (y vendida) del producto que la empresa $j$ ofrece ($Y_{t}^{j}$).\footnote{La aproximación de los costos marginales corresponde a \cite{gagnon2005new} como se presentan en \cite{galvis2010estimacion} y \cite{cespedes2005new}. Otras aproximaciones se encuentran en \cite{woodford2011interest}.}\\

Reescribiendo la ecuación \eqref{9}, se tiene la forma siguiente:
  \begin{equation}\label{10}
P_{t}^*=\mu X_{t}^{j}
\end{equation}
 Siendo $X_{t}^{j}=E_{t-1}\frac{N_{t}^{j}W_{t}}{\alpha Y_{t}^{j}}$ los costos marginales nominales.\\
   
Aplicando logaritmos en  \eqref{10}  se tendría:
 \begin{equation}\label{11}
lnP_{t}^*=ln \mu X_{t}^{j}
\end{equation}
 \begin{equation}\label{12}
p_{t}^*=\mu +mc_{t}
\end{equation}
Donde $p_{t}^*=ln P_{t}^*$, $\mu^{*}=ln \mu$ y el costo marginal es igual a $mc=n_{t}^{j}+w_{t}-\alpha y_{t}^{j}$ con variables en logaritmos.\\

Retomando la ecuación \eqref{6}, se introduce la ecuación \eqref{12} para obtener   el precio óptimo con la presencia del  \textit{mark-up} sobre los costos marginales:
%el precio óptimo que la empresa fijaría con base en expectativas sobre el precio futuro y el margen sobre los costos marginales:
\begin{equation}\label{13}
z_{t}=(1-\theta\beta)\sum_{k=0}^{\infty}(\theta\beta)^{k}E(\mu +mc_{t+k})
\end{equation}
Al resolver la sumatoria de manera iterativa la ecuación \eqref{13} se tiene que: 
\begin{equation}\label{14}
z_{t}=\theta\beta Ez_{t+1}+(1-\theta\beta)(\mu +mc_{t}) 
\end{equation}
Por lo anterior, cada empresa fijaría su precio según las expectativas sobre el precio del futuro y el margen sobre los costos marginales.

\subsubsection*{La curva de Phillips neokeynesiana}
En adición, faltaría por agregar los resultados de la fijación de precios de cada empresa -dada en la ecuación \eqref{9}-  en la economía. Para este problema, la estructura de precios del trabajo de \cite{calvo1983staggered} permite que el nivel de precios agregados se determine como una combinación convexa del nivel de precios rezagados y el nuevo precio óptimo. Por lo anterior, se tiene que:\footnote{Las variables presentadas en la ecuación  \eqref{15} se encuentran en logaritmos.} 
\begin{equation}\label{15}
p_{t}=(1-\theta)z_{t}+\theta p_{t-1}
\end{equation}
Cada variable está expresada como porcentaje de desviación según un nivel de inflación cero. En este caso $\theta$ corresponde a la probabilidad de no alterar el precio, independiente del tiempo desde la última revisión, esta probabilidad se presenta en $\theta p_{t-1}$. Por su parte, la probabilidad en que las empresas vuelven a fijar el precio es ($1-\theta$), captado en  $(1-\theta)z_{t}$.\\

Ahora, despejando la ecuación \eqref{15}, el precio óptimo que fijan las empresas puede representarse así:
\begin{equation}\label{16}
z_{t}=\frac{1}{(1-\theta)}(p_{t}-\theta p_{t-1})
\end{equation}
Al igualar las ecuaciones \eqref{16} y \eqref{14}, se tiene que: 
\begin{equation}\label{17}
\frac{1}{(1-\theta)}(p_{t}-\theta p_{t-1})=\theta\beta Ez_{t+1}+(1-\theta\beta)(\mu +mc_{t}) 
\end{equation}
Y al sustituir $z_{t+1}\frac{1}{(1-\theta)}(p_{t+1}-\theta p_{t})$ en \eqref{17} se obtiene:
\begin{align}
\frac{1}{(1-\theta)}(p_{t}-\theta p_{t-1})&=\frac{\theta}{(1-\theta)}\beta (E_{t}p_{t+1}-\theta p_{t})+(1-\theta\beta)(\mu +mc_{t}) \\
(p_{t}-\theta p_{t-1})&=\theta\beta (E_{t}p_{t+1}-\theta p_{t})+(1-\theta\beta)(1-\theta)(\mu +mc_{t}) 
\end{align}
Reordenando y teniendo en cuenta que la tasa de inflación se define como $\pi =p_{t}-p_{t-1}$, se tiene la siguiente aproximación a la NKPC:
\begin{equation}\label{18}
\pi_{t}=\beta E_{t}(\pi_{t+1})+\frac{(1-\theta\beta)(1-\theta)}{\theta}(\mu +mc_{t}-p_{t})
\end{equation}
Así pues, se definirá $mcr_{t}=\mu +mc_{t}-p_{t}$ como el costo marginal real según el nivel de estado estacionario, en otras palabras, $mcr_{t}$ será la log-linealización del costo marginal real. Retomando  la \eqref{18}, si $\lambda =\frac{(1-\theta\beta)(1-\theta)}{\theta} $, se obtiene finalmente la NKPC: 
\begin{equation}\label{19}
\pi_{t}=\beta E_{t}(\pi_{t+1})+\lambda(mcr_{t})
\end{equation}
La ecuación de la NKPC establece que la inflación está compuesta por dos factores: en primer lugar, por la tasa de inflación esperada para el próximo periodo ($\beta E_{t}\pi_{t+1}$), y por otra parte,  la brecha entre el nivel de precio óptimo sin fricción ($\mu +mc_{t}$) y el nivel de precio actual ($p_{t}$). En este sentido, la inflación dependería de manera positiva con respecto a los costos marginales ($mc_{t} - p{t}$).\\

El aporte del modelo de \cite{calvo1983staggered} le permite a las empresas que mantengan su precio como un margen fijo sobre el costo marginal. Las presiones inflacionarias surgen a medida que la relación entre costo marginal y precios se vuelve alto, debido a que las empresas reestablecen en promedio sus precios en mayor cuantía. Además, \cite{gali1999inflation} agrega a la NKPC  el elemento $\lambda$ para explicar el decrecimiento en  $\theta$, es decir, la inflación seria menos sensible a movimientos en los costos marginales según la incidencia de las rigideces de precios.\\

Dentro de los aspectos relevantes que han caracterizado la NKPC, ha sido la participación de la brecha de los costos marginales en la inflación en lugar de la brecha del producto. La razón principal se debe a que al incorporar a las empresas en este modelo, intentaran mantener un margen de beneficio fijo sobre el costo marginal, en consecuencia provocarían inflación si las empresas procuran fijar nuevos precios por el motivo del declive del margen sobre los costos. No obstante, se asume que las rigideces de los precios y su evolución estarán acordes a los productores. La presencia de los participantes de la oferta explicaría que la dinámica inflacionaria parte de los costos de producción \citep{galvis2010estimacion}.

\section{Costos marginales y ciclo económico}\label{seccm}
Al recordar la curva de Phillips con expectativas adaptativas de los años 70 y 80 del siglo XX,\footnote{Teniendo en cuenta el anexo de  \cite{friedman1968role} y \cite{phelps1967phillips}. } es posible asumir que la brecha del producto ($y_{t}$) pueda ser vinculada en la NKPC al igualar $\lambda(mcr_{t})=k(y_{t})$ en la ecuación \eqref{19}. Así pues, se tiene que la NKPC modificada es:\footnote{También  la ecuación que compone a $k(y_{t})$ puede representarse a partir de un modelo de equilibrio general neokeynesiano, en donde $\varphi $ es el coeficiente de aversión al riesgo y $ \sigma$ es la inversa de la elasticidad precio de la oferta de trabajo. En suma, se da el grado de apertura de la economía $k$. En este sentido se supone: $\lambda(mcr_{t})=(\varphi + \sigma )y_{t}=k(y_{t})$.} 
\begin{equation}\label{20}
\pi_{t}=\beta E_{t}(\pi_{t+1})+k(y_{t})
\end{equation}

En los últimos años, ha sido discutido asumir el anterior supuesto principalmente por la estimación empírica que representa. Para \cite{gali1999inflation}, la brecha del producto usado en los modelos teóricos de la NKPC  es diferente a los tomados en las estimaciones empíricas, por la razón que el Producto Interno Bruto (PIB) sin tendencia o filtrada como \textit{proxy} de la brecha del producto, no es un enfoque adecuado debido a que la producción potencial es representada como una función suavizada en el tiempo. Así mismo, dichos autores afirman que la producción potencial en teoría enfrenta fluctuaciones volátiles debido a choques diferentes a los monetarios. Aunque se conoce de dichas falencias en las estimaciones empíricas, aún no hay un criterio teórico  para evaluar la brecha del producto.\\

La dificultad de detectar el efecto significativo de la actividad real (medido en la brecha del producto) sobre la inflación, permitió la búsqueda de otra variable que explicara la dinámica inflacionaria. La aceptación que tendría el costo marginal en lugar de la brecha del producto, origina características deseables en la explicación directa del impacto de las ganancias de productividad en la inflación. Este hecho lo pasaba por alto al adicionar la brecha del producto. De otra manera, la adición de un conjunto de empresas que cuentan con una fijación de precios en una regla empírica retrospectiva, implicó tener en cuenta la persistencia observada de la inflación.\\

El aspecto fundamental en que se basa la estimación de la NKPC es el vínculo entre la actividad agregada y los costos marginales. El factor común se encuentra contenido en los costos laborales unitarios, pero son los costos marginales quienes se van a caracterizar por retrasar la producción durante el ciclo en lugar de moverse al mismo tiempo, en contraste con la predicción del marco macroeconómico estándar de precios fijos.\footnote{Para mayor información frente al marco macroeconómico estándar de precios fijos ver en \cite{christiano1997sticky} }  Por lo tanto, la inercia de la inflación puede estar explicada por el ajuste lento de los costos marginales según los movimientos de la producción.\\

La NKPC al adicionar los costos marginales, revela importantes novedades en el entendimiento de las causas de la inflación. Entre una de ellas está el hecho en que la inflación tiene que ver con el margen superior que establecen las empresas, es decir, al estar sujeto este margen a la elasticidad del mercado, la inflación dependerá de la coyuntura que atraviesa la economía en cada momento. En este sentido, en épocas de auge aumenta el margen sobre los costos y por tanto la inflación, y en épocas de crisis las empresas bajan el margen para deshacerse de inventarios y baja por lo tanto la presión al alza de los precios disminuyendo así la inflación \citep{galvis2010estimacion}.\\

Estas consideraciones han sido evidenciadas en estudios previos realizados al análisis de empresas en la Zona Euro y más reciente en las empresas de Colombia. El patrón de resultados de los trabajos de  \cite{fabiani2005pricing} y \cite{misas2009formacion},\footnote{Para el análisis en la formación de precios para Colombia se tuvieron en cuenta 4626 empresas \citep{misas2009formacion} y para la Zona Euro más de 11000 empresas \citep{fabiani2005pricing}.}  respaldan la reciente ola de estimaciones de versiones híbridas de la NKPC, debido a que las empresas tienen algún tipo de poder de mercado y pueden establecer sus precios por encima de los costes marginales. Esto también sugiere que los modelos con competencia monopolística, como los modelos neokeynesianos, pueden ser una mejor descripción para la mayoría de los mercados de bienes y servicios que aquellos que suponen una competencia perfecta. \\

El comportamiento de las empresas y la manera en que fijan sus precios determinan la forma en que las decisiones de política monetaria afectan a la economía en general, es decir, el grado y tipo de rigidez que presentan los precios afectan el impacto de los cambios de las tasas de interés sobre la inflación y el producto \citep{misas2009formacion}. De esta manera, se evidencia que el comportamiento de las variables monetarias tiene impacto sobre las variables reales, en contradicción con el postulado central de la nueva macroeconomía clásica.\\

Por otra parte, estas encuestas recalcan que el tamaño de la empresa tendrá un comportamiento distinto frente al momento de revisar los precios, en este sentido, entre mayor sea el tamaño de las empresas sus decisiones sobre los precios tendrán un comportamiento  \textit{forward-looking}, es decir, actuarán de manera racional con la información futura, lo cual se le da más énfasis a la meta de inflación que al salario mínimo, mientras que para las no grandes es más relevante el salario mínimo. Esto explica que las empresas con mayor poder de mercado enfatizan sus decisiones actuales en los precios vistas desde el futuro, característica principal de la NKPC.\\

A escala sectorial, \cite{misas2009formacion}  encuentran que, tanto para la agricultura como para la industria, la información presente tiene una mayor relevancia. Para el caso de la pesca, la información futura compite en jerarquía con la información presente. Tanto la inflación presente como la esperada son importantes para la revisión de precios por parte de las empresas colombianas, si se miran tanto por tamaño como por sectores. Sin embargo, en términos relativos, los autores encuentran que la última es más importante que la primera en el caso de industria y pesca. Lo anterior toma importancia por el hecho en que la  heterogeneidad en el comportamiento de fijación de precios principalmente entre los sectores -y entre regiones según su composición económica- no solo complica la conducción de la política monetaria, sino que también afecta el mecanismo de transmisión de la política monetaria \citep{romero2008transmision}.\\

Para el caso del ciclo económico de la NKPC, \cite{gali1999inflation}  destacan la participación del ingreso laboral como \textit{proxy} en la creación del costo marginal real. Los autores encuentran que los costos marginales reales son un determinante significativo y cuantitativamente importante de la inflación. La acogida que tuvo este hallazgo fue masiva en la medida en que el modelo fue exitoso. Sin embargo, varios autores han cuestionado este \textit{proxy} de los costos marginales debido a que los ingresos laborales son un costo promedio y no un costo marginal \citep{rudd2007modeling}, y porque el ingreso laboral actúa de manera anticíclica \citep{mazumder2010new}.\\

La participación del ingreso laboral es anticíclica en el sentido en que aumenta durante los tiempos de recesión, contrario a lo que la intuición y la teoría nos dicen sobre el costo marginal. La teoría microeconómica estándar predice que el costo marginal a corto plazo debería ser procíclico, según autores como \cite{bils1987cyclical} y \cite{rotemberg1999cyclical}. En este sentido, una expansión en la economía generaría que las empresas aumentarán la producción, lo cual lleva a que la curva del costo marginal a corto plazo sea inclinada hacia arriba, teniendo en cuenta que algunos factores de producción permanecen fijos. Existe un consenso de trabajos empíricos que evidencian que los costos marginales tienen una pendiente ascendente a corto plazo, aunque queda en entredicho el grado de su pendiente  \citep{mazumder2010new}. En el caso contrario, si la recesión genera una reducción en la producción, el descenso de los costos conlleva a la disminución marginal en la producción.\\

Inicialmente, este problema esencial ya se había enfatizado por \cite{fuhrer1995inflation}, los cuales sostuvieron que la NKPC de referencia implicaba que la inflación debería asemejarse al comportamiento del ciclo de la brecha del producto,\footnote{Cabe recordar que lo mencionado anteriormente hace énfasis en que los costos marginales explican en gran medida la dinámica de la inflación que la misma brecha del producto.} es decir, por ejemplo, un aumento en la inflación actual esperaría que se diera un aumento en la brecha del producto, lo cual en caso opuesto se podría dar en los datos. Por tal motivo, \cite{gali1999inflation} demuestran por medio de una correlación cruzada que la  brecha de producto actual se mueve positivamente con la inflación futura y negativamente con la inflación rezagada, consistente con la antigua teoría de la curva de Phillips, pero en contradicción directa con la NKPC. Esta verificación del comportamiento de los ciclos se realiza a manera de visualizaciones y por medio del cálculo de correlaciones simples entre  la inflación, el ingreso laboral y/o la brecha del producto. \\

Por lo anterior, \cite{gali1999inflation}  establecen que los costos marginales reales serían la medida más consistente para explicar la inflación, apoyado por las características previamente explicadas y por su condición acertada en la teoría, es decir,  actuaban de manera procíclica los costos marginales, omitiendo así  el caso hipotético en que los datos tomados fueran opuestos a su requerimiento. Sin embargo, una posible solución a dicho inconveniente es la idea de dejar que la inflación dependa de una combinación convexa de la inflación futura esperada y la inflación rezagada, es decir,  adicionar rezagos para capturar la persistencia de la inflación que no se explica en el modelo de referencia. Por lo tanto, ahora la ecuación \eqref{19} tendría la siguiente forma:\footnote{En este caso $\beta$ puede actuar de manera parcial entre $\pi_{t+1}$ y $\pi_{t-1}$} 
\begin{equation}\label{21}
\pi_{t}= \gamma_{f}E_{t}(\pi_{t+1})+\gamma_{b}(\pi_{t-1})+\lambda(mcr_{t})
\end{equation}
 
\section{Evidencia empírica: Revisión de resultados}\label{sec3}
\subsection{Nivel internacional}\label{s141}
Los estudios preliminares sobre el NKPC híbrido, como el de \cite{fuhrer1995inflation}  continuaron utilizando la brecha del producto como la principal variable impulsora de la inflación, pero  \cite{gali1999inflation}, sugirieron usar el costo marginal real con base en la participación del ingreso laboral. La aceptación que tuvo los costos marginales sobre la dinámica inflacionaria en los Estados Unidos dio lugar al uso de modelos dinámicos de equilibrio general en el ámbito monetario, a partir de modelos derivados de fundamentos microeconómicos que explicarían procesos inflacionarios.\\

Los resultados  obtenidos por \cite{gali1999inflation} muestran que ambos parámetros son significativos ($\beta, \lambda$). Además, se encuentran que $\theta=0.829$, infiriendo que el 82,9\% de las empresas dejan fijos los precios en promedio durante cinco trimestres. Así mismo, se explica que alrededor del 17\% de las empresas ajustan su precio según valor actual del costo marginal real en la economía de Estados Unidos para el periodo comprendido de 1960 a 1997 (en datos trimestrales).\footnote{El calculo de $\theta$ se encuentra resolviendo la ecuación \eqref{18} en $\lambda$. En seguida, el promedio del período fijo se calcula como $\frac{1}{1-\theta}$, junto con el tamaño de las empresas que ajustan el precio según los costos marginales, $1-\theta$.} Los autores encuentran que los costos marginales reales son de hecho un determinante estadísticamente significativo y cuantitativamente importante de la inflación, como lo predice la teoría. Esto ha motivado a otros autores a estimar la NKPC en diferentes países (tabla \ref{t1}).

\begin{table}[H]
  \centering
  \caption{Resultados estimaciones de la NKPC nivel internacional}
    \begin{tabular}{ c  c c c c c }
      \hline
        País  & $\beta$ & $\lambda$ & $\theta$ & $\frac{1}{1-\theta}$ & Fecha \\
            \hline
              \hline
    Estados Unidos \dag & 0.926 & 0.047 & 0.829 & 5.8   & 1960:Q1-1997:Q4 \\
    Zona Euro \dag \dag & 0.914 & 0.088 & 0.771 & 4.4   & 1970:Q1-1997:Q4 \\
    Australia * & 0.942 & 0.113 & 0.73  & 3.7   & 1962:Q1-2000:Q4 \\
    Chile ** & 0.946 & 0.385 & 0.553 & 2.2   & 1990:Q1-2004:Q4 \\
      \hline
    \end{tabular}%
  \label{t1}\\
  \raggedright  \scriptsize \textbf{Nota:}  \cite{gali1999inflation}\dag, \cite{gali2001european}\dag\dag, \cite{neiss2005inflation}*, \cite{cespedes2005new}**.   
\end{table}%

Entre estas estimaciones, autores como \cite{neiss2005inflation} se concentran en gran parte por analizar la estabilidad de los parámetros de la NKPC y ampliar la discusión sobre la relación de los costos marginales y la brecha del producto en países como Reino Unido, Estados Unidos y Australia. Además, los autores adicionan a manera de variables \textit{dummy}   reformas que incidieron en el desarrollo normal de sus economías. No obstante, sus resultados no reflejan distanciamiento de los trabajos vistos en la tabla \ref{t1}.\\

Al mismo tiempo, \cite{cespedes2005new} evidencian en una economía emergente como la chilena, el coeficiente de \cite{calvo1983staggered} desciende a un rango de 0.55 a 0.80, sin salirse del rango de 2 a 5 trimestres  en duración promedio en que los precios permanecen sin cambios. Por otra parte, este estudio tiene la particularidad de respaldar la hipótesis de la existencia de una ruptura estructural en el NKPC, por motivo de la convergencia  a un objetivo de inflación de largo plazo. Esto evidenció que el proceso inflacionario tuviera una mirada de tipo \textit{forward-looking} en los últimos años, lo cual explica  mayor credibilidad de la meta de inflación.\\% la ruptura se da en el 2000
 
Otros autores interesados en estudiar la formación de precios, se han concentrado en analizar principalmente la incidencia de las expectativas adaptativas y racionales en la dinámica inflacionaria, partiendo que en el trabajo preliminar de  \cite{gali1999inflation} se encontró que  el comportamiento prospectivo puede proporcionar una descripción razonablemente lógico. \cite{vavsivcek2011inflation} encuentra para cuatro países de la Unión Europea, pruebas sólidas de que la inflación está determinada por las expectativas de inflación futura, aunque con un mayor grado de persistencia de la inflación que el encontrado en economías desarrolladas. Los autores intuyen que este fenómeno se debe a la gran cantidad de empresas que aún fijan precios de manera simple y retrospectiva, consistente con las expectativas adaptativas.\footnote{El estudio comprende un periodo libre de cambios importantes en los regímenes de política monetaria y durante el cual las series de inflación no estuvieron sujetas a una ruptura estructural.} \\ 

Para México, \cite{ramos2008inflation} evidencian de 1992 a 2007, tanto los componentes hacia atrás como los prospectivos, son importantes para explicar la dinámica de la inflación a corto plazo. Aunque las expectativas de inflación son un determinante importante de la inflación, la inflación rezagada (persistencia de la inflación) también juega un papel clave. Además, en la submuestra que realizan los autores para los años 1997-2007, las estimaciones para los coeficientes muestran alta importancia de las expectativas racionales (prospectiva) para la inflación y su vínculo con los costos marginales.\\

Por otra parte,  \cite{leith2007estimated},  \cite{rumler2007estimates} y \cite{mihailov2011small}  se basan en la  versión de economía pequeña y abierta de la NKPC derivada de \cite{gali2005monetary}. Los autores coinciden en que la tasa de inflación en las economías pequeñas y abiertas está impulsada por expectativas sobre factores externos en un grado sustancial. \cite{mihailov2011small} encuentran que para la mayoría de la muestra de los países  que integran la Organización para la Cooperación y el Desarrollo Económico (OCDE),  los términos de intercambio surgen como el  factor fundamental que impulsa la inflación.\\

Otros factores como el tamaño específico, la estructura de producción y/o los patrones comerciales de un país, así como las tendencias mundiales pueden lograr una influencia más fuerte o más débil de factores externos versus nacionales. Por lo anterior,  modelos alternativos del comportamiento de fijación de precios de las empresas o de rigideces reales en un entorno internacional también han incursionado en la explicación de la dinámica inflacionaria.\\

\begin{table}%[H]
  \centering
    \caption{Resumen de los enfoques de estimación en la literatura}
  \resizebox{16.5cm}{!} {
    \begin{tabular}{p{3cm} p{3cm} p{5cm} p{5cm} p{5cm}}
    \hline
  Articulos & Enfoque de estimación & Expectativa vs rezagos & Significancia & ¿Es rechazado el modelo?  \\
 \hline
  \hline
\cite{gali1999inflation}, \cite{gali2001european}, \cite{gali2005monetary}.    & RE GIV. & El comportamiento \textit{forward-looking} (prospectivo) es dominante, pero el término \textit{backward-looking} (retrospectivo) es significativo. & Significativamente positivo para la participación laboral. & No, basado en la prueba de identificación excesiva y el ajuste visual. \\ \\
     \cite{fuhrer1995inflation}, \cite{fuhrer1997importance}.  & RE VAR-ML. & La fijación de precios no es muy prospectiva; Necesita una gran persistencia intrínseca. & Positivo tanto para la participación laboral como para la brecha del producto, pero la importancia varia. & Rechazo puro de la NKPC con base en la prueba LR y los IRF. \\ \\
     \cite{roberts1995new}, \cite{roberts2005well}.       & GIV, VAR-ML, IRF correspondencia IRF; RE y encuestas de pronósticos. & Los pronósticos de encuestas lentas imparten la persistencia necesaria. Para RE, necesita más de un rezago de inflación. & Positivo tanto para la participación laboral como para la brecha del producto, pero la importancia varia. & No. \\ \\
     \cite{sbordone2002prices}, \cite{sbordone2005expected}.    & RE VAR-MD. & El comportamiento prospectivo es claramente dominante, pero el rezago es significativo. & Positivo pero marginalmente insignificante en el modelo híbrido. & No, basado en una prueba de identificación excesiva y ajuste visual. \\ \\
     \cite{rudd2005new}, \cite{rudd2007modeling}.       & RE GIV (iterado) & Inflación rezagada muy significativa. & Ni la participación laboral ni la brecha del producto agregan poder explicativo. & Si, forzar variable no ayuda a explicar la inflación. \\ \\
    \cite{rudebusch2002assessing}.       & OLS; pronósticos de encuestas & Cuarto trimestre de MA de la inflación rezagada recibe un peso mayor al previsto. & Coeficiente de brecha de producto positivo y significativo. & No. \\ \\
    \cite{ravenna2006optimal}.       & RE GIV, tasa de interés agregada a NKPC. & (NKPC puro) & (No estimado directamente). & No, basado en una prueba de identificación excesiva. \\ \\ 
    \cite{cogley2008trend}.       & Estimación bayesiana usando VAR con parámetros de deriva y volatilidad estocástica. & Término retrospectivo insignificante una vez que se tiene en cuenta la tendencia de la inflación. & (No estimado directamente). & No, según el ajuste visual y la magnitud de los errores de pronóstico. \\ \\ 
           \hline
    \end{tabular}%
 }
  \label{e2}\\
  \raggedright  \scriptsize \textbf{Nota:} Las siglas de la Tabla \ref{e2} indican lo siguiente:  expectativas racionales (RE), Variables instrumentales generalizadas (GIV), vectores autoregresivos (VAR), máxima verosimilitud (ML), mínima distancia (MD) función impulso respuesta (IRF), mínimos cuadrados ordinarios (OLS),  prueba de ratio de verosimilitud (LR), media móvil (MA).\\
  \cite{mavroeidis2014empirical} recoge los trabajos más importantes en la  literatura empírica de NKPC. Estos son clasificados por los autores según el número de citas de Google Scholar hasta mediados de septiembre de 2012.
\end{table}%

En resumen, \cite{mavroeidis2014empirical} examina  la literatura empírica sobre la NKPC.\footnote{La figura \ref{A3} recoge la red de histórica de citas de la NKPC, donde se aprecia que \cite{mavroeidis2014empirical} recoge gran parte de la producción académica (Para mayor información ver apéndice \ref{apendicea} ).} En este trabajo agrupa las diversas contribuciones en los principales enfoques econométricos,  resultados de algunos de los estudios más frecuentemente citados. Así mismo, los autores asocian los principales puntos de controversia en la literatura correspondientes a la importancia relativa del comportamiento de fijación de precios a futuro y hacia atrás, como también al grado en que la actividad real influye en la dinámica de la inflación. Estos aspectos pueden observarse en la tabla \ref{e2}.\\

A pesar que la NKPC ha tenido la capacidad de explicar la dinámica inflacionaria, las economías regionales aún siguen sin ser estudiadas. La importancia de la presencia de dicho estudio  radica en que el desempeño económico, diferencias institucionales y diferentes grados de desarrollo del mercado entre departamentos van a permitir diagnosticar el comportamiento de los costos marginales y la inflación.\\

No obstante, en los últimos años China se ha concentrado en analizar el proceso inflacionario de sus provincias aún cuando se ha prestado menor atención, teniendo en cuenta que el país asiático busca desarrollar e implementar una política monetaria independiente por la vía de la adopción de estabilidad de precios. La efectividad de su política monetaria está sujeto a la dinámica de la inflación y su posterior vinculación bajo un entorno regional heterogéneo como el de China.\\

Por lo anterior, recientes trabajos de la NKPC han avanzado para dar entendimiento a los procesos inflacionarios en la economía regional. En este sentido, con datos anuales en el período 1982-2002 en China, \cite{funke2006inflation}  desarrolla una NKPC con el modelo estándar, es decir, incorporación de expectativas de inflación, rezagos en la inflación y  la brecha del producto, en lugar de los costos marginales reales. Dichos componentes van a tener coeficientes consistentes, con excepción de la brecha del producto, la cual presentaría insignificancia estadística. Además, el autor considera variables instrumentales como la tasa de inflación rezagadas y las brechas de producto, el precio real del petróleo y el tipo de cambio efectivo nominal para el control de problemas de endogeneidad. \\

En el mismo periodo, pero en datos trimestrales y por medio de encuestas, \cite{scheibe2005phillips} evidencian que la NKPC se ajusta mejor al futuro que al pasado. Por otra parte, \cite{ha2003causes} encuentran que la NKPC representó mejor la dinámica de inflación que la Curva de Phillips convencional en China para el periodo 1989-2002.\footnote{Investigaciones anteriores con datos regionales que estimaron la Curva de Phillips pueden encontrarse en \cite{coen1999nairu,hassler2003inflation,dinardo1999phillips}, para  44 áreas metropolitanas en los Estados Unidos, estados alemanes y 9 países de la OCDE, respectivamente.}    Adicionalmente, sus hallazgos indican que la deflación o la baja inflación, reflejó un rápido crecimiento de la productividad.\\

Para \cite{mehrotra2010modelling}, 22 de las 29 provincias presentan significancia estadística en la brecha del producto y en el componente de inflación esperada para el periodo 1978-2004 (datos anuales), siendo variables importantes para el proceso de formación de inflación en China. Particularmente, estas provincias se ubican en la costa de China  y comparten características comunes como las de ser más abiertas al comercio internacional y poseer un porcentaje más bajo de empresas controladas por el estado en su producción total.  Por lo tanto, los autores concluyen que bajo un entorno de inflación muy baja, el comportamiento prospectivo de los agentes puede ser beneficioso para estimular la economía.\\%coclusiones de mehr

Otros trabajos encargados de estimar la NKPC ha escala regional han surgido con diferentes métodos al conocido GMM. \cite{yesilyurt2014regional} emplea un enfoque econométrico espacial  para estimar la NKPC en 67 provincias de Turquía (1987-2001). Los autores consideran que el comportamiento prospectivo es más importante que el comportamiento retrospectivo  si la tasa de inflación esperada está instrumentalizada por un rango de variables instrumentales. Además, evidencian significancia a favor de la convergencia, es decir, cuanto mayor sea la inflación atrasada en la propia provincia, o la inflación atrasada más baja en las provincias vecinas, menor será la tasa de inflación actual, lo cual respalda  procesos de integración regional de las tasas de inflación en Turquía. \\

De manera más reciente, \cite{saygili2020sectoral}  utiliza el enfoque de los errores estándar corregidos de los paneles heteroscedasticos de regresión Prais-Winsten (PCSE) para el análisis de la dinámica inflacionaria entre países de la OCDE (1990-2016). La estimación de los coeficientes evidencia variaciones entre sectores, lo cual explica las enormes diferencias de la respuesta sectorial en términos de política monetaria. En adición, los tamaños de los coeficientes estarían asociados al grado de integración a las cadenas de valor globales. Por lo anterior, entre diferentes métodos de estimación, series de tiempos y las diferencias entre regiones y sectores  dan muestra de importantes resultados al analizar el proceso inflacionario.  Por tal razón, estos antecedentes hacen viable la posibilidad de estimar la NKPC  para el análisis de la dinámica inflacionaria en los departamentos en Colombia bajo un entorno de inflación baja.

\subsection{Nivel nacional}
En Colombia,  \cite{bejarano2005estimacion} comprueba empíricamente la NKPC para el periodo 1984 a 2002. El autor encuentra que la inflación y los costos marginales tienen una relación positiva a partir del modelo neokeynesiano de optimización dinámica planteado por \cite{gali1999inflation}. Estos parámetros estructurales son consistentes con los del modelo y a su vez con los encontrados a escala internacional.\\

\cite{bejarano2005estimacion} evidencia que la inflación trimestral responde a cambios futuros de la brecha de los costos marginales, lo cual implicaría que los agentes tengan expectativas  racionales, indicando que no existirían costos de desinflación en Colombia. En datos anuales, la inflación responde ante cambios de la brecha del costo marginal real de manera parcial tanto en expectativas adaptativas como racionales.\\

Para \cite{galvis2010estimacion}, la verificación empírica de la NKPC es acorde para la economía colombiana en el periodo 1990-2006. El autor evidencia que los costos marginales de forma significativa explican la dinámica inflacionaria, dado que las empresas fijan el precio en promedio por cinco periodos para mantener cierto margen de ganancia sobre sus costos marginales (tabla \ref{t2}). En este sentido, se espera que las empresas mantengan sus precios en el tiempo en cuanto exista claridad entre reglas de política y la volatilidad de la inflación sea menor entre periodos. Adicionalmente observa, que cuando aumenta la inflación, las empresas comienzan a cambiar el precio con mayor frecuencia. Por último, \cite{galvis2010estimacion} abre la discusión sobre si la fijación de precios por parte de las empresas por varios periodos está estrechamente vinculado a la credibilidad del modelo o si es explicado por la demanda agregada de la economía en Colombia, la cual ha sido golpeada en los últimos años.\\

En la reciente monografía, \cite{hernandez2020evidencia} evidencian rigideces nominales vía precios, aunque con una pérdida de fuerza del modelo en Colombia, es decir, en comparación con los últimos dos trabajos mencionados para Colombia, existe una menor presencia en el grado de rigidez en los precios,  lo cual indica que un menor porcentaje de empresas mantienen los precios fijos ($\theta$) y por otra parte incrementan la velocidad en que las empresas cambian de precios (tabla \ref{t2}). Los autores enfatizan que este modelo ha evidenciado la disminución del impacto que tiene sobre las variables reales de la economía.

\begin{table}[H]
  \centering
  \caption{ Resultados estimaciones de la NKPC nivel nacional}
    \begin{tabular}{ c  c c c c c }
  \hline
        País  & $\beta$ & $\lambda$ & $\theta$ & $\frac{1}{1-\theta}$ & Fecha \\
         \hline
           \hline
    Colombia \dag & 0.87  & 0.171 & 0.696 & 3.3   & 1984:Q1-2003:Q4 \\
    Colombia \dag \dag & 0.832 & 0.0784 & 0.807 & 5.2   & 1990:Q1-2006:Q4 \\
    Colombia \dag \dag \dag & 0.912 & 0.201 & 0.662 & 3   & 2000:Q1-2019:Q4 \\
      \hline
    \end{tabular}%
  \label{t2}\\
  \raggedright  \scriptsize \textbf{Nota:} \cite{bejarano2005estimacion}\dag, \cite{galvis2010estimacion}\dag\dag, \cite{hernandez2020evidencia}\dag\dag\dag.   
\end{table}%
