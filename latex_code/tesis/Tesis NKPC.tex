\documentclass[12pt,spanish,openany,letterpaper,pagesize]{report}

\usepackage[utf8]{inputenc}
\usepackage[spanish]{babel}%escribir con acentos sin necesidad de comandos \'{} .

\usepackage{amsfonts}
\usepackage{amssymb}
\usepackage{graphicx}

%Quitar para cambiar la letra
%\usepackage{helvet}
%\renewcommand{\familydefault}{\sfdefault}


\usepackage{fancyhdr}
\usepackage{epsfig}
\usepackage{epic}
\usepackage{eepic}
\usepackage{amsmath}
\usepackage{upgreek} % para poner letras griegas sin cursiva
\usepackage{mathdots} % para el comando \iddots
\usepackage{mathrsfs} % para formato de letra
\usepackage{threeparttable}
\usepackage{amscd}
\usepackage{here}
\usepackage{graphicx}
\usepackage{lscape}
\usepackage{tabularx}
%\usepackage{subfigure}
\usepackage{subcaption}
\usepackage{longtable}
\usepackage[left=4cm,right=2cm,top=3cm,bottom=3cm]{geometry}
%\usepackage[sort&compress]{natbib} 
\usepackage{natbib}
\usepackage{rotating} %Para rotar texto, objetos y tablas seite. No se ve en DVI solo en PS. Seite 328 Hundebuch
\usepackage{ragged2e}

     
     
     
     
% PACKAGES DE TENDENCIAS


\usepackage{amsfonts}%
\usepackage{amssymb}%
\usepackage{makeidx}
\usepackage{xcolor}
\usepackage[stable]{footmisc}
\usepackage[section]{placeins}
%Paquetes necesarios para tablas
\usepackage{longtable}
\usepackage{array}
\usepackage{xtab}
\usepackage{multirow}
\usepackage{colortab}
%Paquetes necesarios para imágenes, pies de página, etc.
\usepackage{graphicx}%
\usepackage{rotating}
\usepackage{lmodern}
\usepackage{fancyhdr}     
\usepackage{titlesec}
%\usepackage{enumitem}
%\usepackage{subfigure}
\usepackage{array}
\usepackage{longtable}
\usepackage[colorlinks=true, 
            linkcolor = black,
            urlcolor  = black,
            citecolor = black,
            anchorcolor = blue]{hyperref}
\usepackage[sc]{mathpazo}
\usepackage{multicol}
\usepackage{titling}
\usepackage{titlesec}

     
     
                        %se usa junto con \rotate, \sidewidestable ....

%\renewcommand{\theequation}{\thechapter-\arabic{equation}}
%\renewcommand{\theequation}{\arabic{equation}}
\renewcommand{\thefigure}{\textbf{\thechapter-\arabic{figure}}}
\renewcommand{\thetable}{\textbf{\thechapter-\arabic{table}}}

%REVISAR ESTA PARTE
\pagestyle{fancyplain}%\addtolength{\headwidth}{\marginparwidth}
\textheight22.5cm \topmargin0cm \textwidth16.5cm
\oddsidemargin0.5cm \evensidemargin-0.5cm%
\renewcommand{\chaptermark}[1]{\markboth{\thechapter\; #1}{}}
\renewcommand{\sectionmark}[1]{\markright{\thesection\; #1}}
\lhead[\fancyplain{}{\thepage}]{\fancyplain{}{\rightmark}}
\rhead[\fancyplain{}{\leftmark}]{\fancyplain{}{\thepage}}
\fancyfoot{}
\thispagestyle{fancy}%
%\renewcommand{\baselinestretch}{1.5} %Interlineado


%\addtolength{\headwidth}{0cm}
%\unitlength1mm %Define la unidad LE para Figuras
%\mathindent8.25cm %Define la distancia de las formulas al texto,  fleqn las descentra 16.5 es 
%\marginparwidth0cm
\parindent0cm %Define la distancia de la primera linea de un parrafo a la margen

%Para tablas,  redefine el backschlash en tablas donde se define la posici\'{o}n del texto en las
%casillas (con \centering \raggedright o \raggedleft)
\newcommand{\PreserveBackslash}[1]{\let\temp=\\#1\let\\=\temp}
\let\PBS=\PreserveBackslash

%Espacio entre lineas
\renewcommand{\baselinestretch}{1.1}

%Neuer Befehl f\"{u}r die Tabelle Eigenschaften der Aktivkohlen
\newcommand{\arr}[1]{\raisebox{1.5ex}[0cm][0cm]{#1}}

%Neue Kommandos
\usepackage{Befehle}
%Inicio del documento. Tener en cuenta que hay archivos auxiliares

\begin{document}
\pagenumbering{roman}
%\newpage
%\setcounter{page}{1}
\begin{center}
%\begin{figure}
%\centering%
%\epsfig{file=HojaTitulo/EscudoUN,scale=1}%
%\end{figure}
\thispagestyle{empty}  \textbf{
DINÁMICA DE LOS PRECIOS EN LOS DEPARTAMENTOS EN COLOMBIA: UNA ESTIMACIÓN USANDO LA CURVA DE PHILLIPS NEOKEYNESIANA (2010-2019)}\\[6.0cm]
\textbf{JUAN PABLO CELY ACERO}\\[2.0cm]

\begin{figure}[H]
  	\centering 	
	\includegraphics[width=0.2\textwidth]{logo/logo}
	\end{figure}
\vspace*{2.0cm}
UNIVERSIDAD PEDAGÓGICA Y TECNOLÓGICA DE COLOMBIA\\
FACULTAD DE CIENCIAS ECONÓMICAS Y ADMINISTRATIVAS\\
TUNJA, COLOMBIA\\
2020\\
\end{center}

\newpage{\pagestyle{empty}\cleardoublepage}

\newpage
\begin{center}
\thispagestyle{empty}  \textbf{
DINÁMICA DE LOS PRECIOS EN LOS DEPARTAMENTOS EN COLOMBIA: UNA ESTIMACIÓN USANDO LA CURVA DE PHILLIPS NEOKEYNESIANA (2010-2019)}\\[3.5cm]
JUAN PABLO CELY ACERO\\[3.5cm]
\small Tesis o trabajo de grado presentada(o) como requisito parcial para optar al
título de:\\
\textbf{ECONOMISTA}\\[3.5cm]
Director:\\
JOSÉ MAURICIO GIL LEÓN\\[3.5cm]
UNIVERSIDAD PEDAGÓGICA Y TECNOLÓGICA DE COLOMBIA\\
FACULTAD DE CIENCIAS ECONÓMICAS Y ADMINISTRATIVAS\\
TUNJA, COLOMBIA\\
2020\\
\end{center}
\newpage
\begin{center}


%\newpage{\pagestyle{empty}\cleardoublepage}
\centering
\thispagestyle{empty} \textbf{Nota de aceptación}

\vspace{3cm}

\centering 
\rule{80mm}{0.1mm} 

\vspace{1cm}

\rule{80mm}{0.1mm} 

\vspace{1cm}

\rule{80mm}{0.1mm} 

\vspace{3cm}

\rule{80mm}{0.1mm} 

Firma del jurado

\vspace{2cm}

\rule{80mm}{0.1mm}

Firma del jurado

\vspace{5cm}

Tunja, Septiembre de 2020


\end{center}

\newpage


\renewcommand{\tablename}{\textbf{Tabla}}
\renewcommand{\figurename}{\textbf{Figura}}
\renewcommand{\listtablename}{Lista de Tablas}
\renewcommand{\listfigurename}{Lista de Figuras}
\renewcommand{\contentsname}{Contenido}
\renewcommand{\appendixname}{Anexo}


%\newcommand{\clearemptydoublepage}{\newpage{\pagestyle{empty}\cleardoublepage}}
\cleardoublepage
\addcontentsline{toc}{chapter}{Lista de figuras} % para que aparezca en el indice de contenidos

\listoffigures % indice de figuras

\cleardoublepage
\addcontentsline{toc}{chapter}{Lista de tablas} % para que aparezca en el indice de contenidos
\listoftables % indice de tablas
\tableofcontents
\cleardoublepage
%\include{Tab_Simbolos/TabSimbolosMSc}
%\include{Resumen}%\newcommand{\clearemptydoublepage}{\newpage{\pagestyle{empty}\cleardoublepage}}
\pagenumbering{arabic}

%Notas
%Seguir el planteamiento de las graficas de kap1, correlaciones cruzadas Gali 99, compilacion mazumder(2010) mavroeidis 2014 evidencia empirica grafica.
%Metadatos importante para recoger algunas discuciones en GMM y otros
%GMM Dufour rudd2005 Gali
%Federal Reserve Bank SF al final se puede pensar en un VAR  que se encuentra al final, asi como seguir mavroeidis 2014. Pedir codigos de GMM

%Preguntar sobre la presicion que se le debe hacer a los costos marginales si son reales o nominales
%Revisar especificaciones de las variables instrumentales que plantea Gali 99. Preguntar al profe.
%Seguir el F estadistico de Vard(2011) como se mensiona que los instrumentos son debiles en Forware-looking, importante por suspruebas. Preguntar al profe. 

%http://minisconlatex.blogspot.com/2012/03/simbolos-de-todo-tipo.html   
%http://www.alciro.org/alciro/Matematicas-Web-LaTeX_14/S1imbolos-Matematicos-LaTeX_103.htm
% Figuras para citar

\chapter*{Resumen} \label{resumen}
\addcontentsline{toc}{chapter}{Resumen}
Este documento describe la dinámica de la inflación en las economías departamentales de Colombia en la última década, utilizando el marco de la curva de Phillips neokeynesiana (NKPC). Se encuentran diferencias en la formación de la inflación y evidencia que la NKPC permite describir baja probabilidad de cambios en los precios para departamentos principalmente de la zona central del país. Además, se evidencia que la inflación esperada y la brecha de los costos marginales  son impulsores de la inflación en departamentos que menos han avanzado en términos de apertura económica, ocupación y lo concerniente a la participación del sector primario en la economía regional. Para complementar, se observa que la mayor probabilidad de que los precios permanezcan en el tiempo están influenciados por la baja competitividad, aumento de los ingresos por habitante y la incidencia creciente que puede lograr el sector terciario.\\
 
En particular, los coeficientes de forma reducida y estimaciones implícitas de los parámetros estructurales del modelo, apoyan la importancia que tiene la inflación esperada sobre la formación de precios, mientras que el papel de la inflación rezagada (persistencia de la inflación) también es estadísticamente importante, pero con menor incidencia. Esta persistencia de la inflación podría ser un reflejo de rigideces estructurales  que reducen la capacidad de una región, en relación con otras, para adaptarse a diferentes choques. Estas diferencias en los procesos y mecanismos de inflación entre departamentos tienen implicaciones importantes para la conducción de política monetaria en Colombia.
\\\\

\textbf{Palabras claves:} curva de Phillips neokeynesiana, dinámica de inflación departamental, precios rígidos.

\chapter*{Introducción}
\addcontentsline{toc}{chapter}{Introducción}

El modelo de inflación objetivo puesto en funcionamiento por el Banco Central de la República de Colombia desde el año 1999, tiene como uno de sus pilares la curva de Phillips Neokeynesiana o Nueva curva de Phillips (NKPC, \textit{siglas en inglés}), ahora constituida por la relación entre inflación y la brecha del producto.  Sin embargo, estudios recientes han evidenciado que no es la brecha del producto la que incide en la inflación,  sino la brecha de los costos marginales \citep{gali1999inflation,gali2002new,rumler2007estimates,ramos2008inflation}.\\

Por lo anterior, \cite{gali1999inflation} encuentran que las empresas intentan mantener un beneficio fijo sobre el costo marginal, pero si este margen sobre los costos empieza a declinar, entonces las firmas intentan de nuevo fijar sus precios provocando con ello inflación. En este sentido, al asumir la nueva curva de Phillips se presume de ciertas rigideces en los precios y que estos evolucionan acorde con las decisiones de los productores o por el lado de la oferta.\\

Por esta razón, los neokeynesianos consideran que la dinámica inflacionaria se explica no solo en un fenómeno de demanda, sino también por el lado de la oferta con base en los costos de producción. Pues bien, tal enfoque concibe mecanismos distintos a través de los cuales, las dos variables -inflación y costos marginales- se relacionan otorgándole distintos grados de importancia a las variables intermedias involucradas en su análisis y establece diferentes direcciones en la causalidad de una sobre otra. Entre ellas se encuentran: a) efecto de persistencia (rezagos de la inflación), b) factores de demanda (procedentes de los desequilibrios reales) y c) choques de oferta (cambios en los precios procedentes de factores climáticos que recaen sobre los precios de los alimentos, cambios en la regulación que afectan los precios de servicios públicos o del petróleo en el exterior) \citep{gordon1997time}. \\

En este contexto, suponer que las empresas en un entorno regional  tienen la capacidad de fijar el precio y mantenerlo fijo por algún tiempo debido a las rigideces nominales, dará origen a hallazgos en la dinámica inflacionaria.  Las diferencias en el proceso de formación de inflación en los departamentos  será relevante al momento de determinar el grado de  efectividad de la política monetaria, es decir, debido a que cada región cuenta con estructuras económicas distintas, la política monetaria podría tener efectos desiguales a partir de  variables como:   a) el grado de industrialización y el tipo de especialización o diversificación que tenga la industria de una región, b) el desarrollo y la profundidad financiera, c) la posición neta en el sistema financiero, y d) la posición neta, en el comercio exterior  \citep{romero2008transmision}. \\

Así mismo, bajo la existencia de competencia imperfecta entre las empresas y la persistencia de los precios en el tiempo, dará lugar a la participación de la política económica como herramienta estabilizadora ante los ciclos adversos que enfrenta la economía colombiana. En este sentido, la estimación de la NKPC se realiza pensando en que la dinámica inflacionaria subnacional ha recibido menos atención en la literatura y en los análisis económicos y, sobre todo, en las economías emergentes como la colombiana.\\

El desarrollo del documento comienza con esta introducción, seguido por el la revisión de literatura (capítulo \ref{cap1}) donde se hace una repaso teórico que envuelve el desarrollo de la curva de Phillips, posteriormente se realiza la derivación de la NKPC y se culmina con las discusiones entorno a la NKPC. En el capítulo \ref{cap2} se describen detalladamente el comportamiento y la tendencia descriptiva de las principales variables que intervienen en el modelo. Para el capítulo \ref{cap3}, se procede a estimar la NKPC para los departamentos de Colombia junto con los resultados obtenidos. Finalmente, se esbozan las principales conclusiones del trabajo.

\chapter{Revisión de literatura} \label{cap1}
%\addcontentsline{toc}{chapter}{\numberline{}Capitulo 1}\\\\
\section{Contexto teórico} \label{sec1}

La curva de Phillips neokeynesiana es el resultado de la evolución teórica a partir de la regularidad empírica encontrada inicialmente entre salarios y desempleo, que posteriormente se fundamentó en un \textit{trade-off} entre inflación y la brecha del producto, elemento clave para el modelo de inflación objetivo.\footnote{La brecha del producto es la desviación porcentual que tiene el producto de su componente de equilibrio. Esta sirve como un indicador del ciclo económico.}  La NKPC revela novedades en las causas y persistencia de la inflación al adicionar la decisión de fijar precios dentro de un problema explícito de optimización individual, lo cual permite vincular la relación a corto plazo entre la inflación y alguna medida de la actividad real. Sin embargo, los adelantos teóricos no estaban lo suficientemente comprobados por la evidencia empírica, sería después del trabajo de \cite{gali1999inflation} que daría lugar al punto de partida hacia la estimación de la NKPC.\footnote{Inicialmente, el término de la NKPC fue usado por \cite{roberts1995new}.} \\

En principio, la curva de Phillips se dio a partir del trabajo de \cite{phillips1958relation}, el cual evidenció una relación inversa entre la inflación de los salarios y la tasa de desempleo para el Reino Unido entre los años 1861 hasta 1957. Más tarde, \cite{lipsey1960relation}   percibió la inflación de los salarios como una \textit{proxy} de la inflación de los precios, lo que daría lugar a una explicación teórica consistente a los resultados encontrados por \cite{phillips1958relation}, añadiendo que el origen de la inflación era producto de exceso de demanda. En seguida, \cite{samuelson1960analytical}  tomarían estos resultados para realizar recomendaciones de política según el \textit{trade-off} entre inflación y producto.\\

Así mismo, la introducción de la curva de Phillips llena un vacío en la teoría keynesiana y se convierte en el análogo de la teoría de salarios y empleo en Keynes \citep{tobin1972inflation}. La ausencia de una relación entre empleo, salarios e inflación no estaba ausente en la teoría general de Keynes, sino en la interpretación de Keynes hecha por \cite{hicks1937mr} , y era la ausencia de una relación entre dichas variables en el modelo IS-LM, lo que debilitaba un poco a este modelo para dar cuenta de los hechos. En este sentido, la curva de Phillips implicaría que un nivel más bajo de desempleo se puede lograr al costo de una inflación más alta.\footnote{Para la mayoría de los macroeconomistas de la segunda mitad del siglo XX, el punto de partida no es la Teoría general misma sino su versión propuesta desde 1937 por Hicks: el modelo IS-LM, el cual invita a llenar lo que es percibido como un vacío: aquel de los fundamentos microeconómicos de la macroeconomía keynesiana. La curva de Phillips anexaría la relación entre empleo, salario e inflación en el modelo macroeconómico inicial.} \\

Después de la aceptación del \textit{trade-off} entre inflación y desempleo, las críticas a la curva de Phillips no se hicieron esperar por parte de \cite{friedman1968role}, quien adicionó expectativas a la inflación, dejando viva la curva de Phillips y su incidencia solamente a corto plazo. Las criticas  de \cite{phelps1967phillips} y \cite{friedman1968role}   se fundamentan con base en la experiencia inflacionaria de los años 60 y principios de los 70 del siglo XX, debido a la expansión monetaria para financiar la guerra de Vietnam por parte de los Estados Unidos. En ese escenario surgió la denominada estanflación que para en ese entonces se convertiría en mayor desempleo con aumento en inflación.\\

Adicionalmente, los autores argumentaron que la curva de Phillips no consideraba como agentes racionales a las empresas y a los trabajadores, lo cual desconocía el costo de vida al momento de acordar los salarios del trabajador por parte del empleador. Por lo tanto, este hecho implica que los trabajadores y las empresas debían acordar el salario de hoy con base en sus expectativas de inflación. Todo esto llevaría a proponer una curva de Phillips aumentada con expectativas por parte de Friedman y Phelps, presentando la inflación del salario nominal en función de las expectativas de inflación de los precios y de la tasa de desempleo.\\

La curva de Phillips aumentada o a largo plazo vertical mostró que la relación entre la inflación y desempleo solo existiría si los negociadores de los salarios predecían sistemáticamente una inflación inferior a la efectiva y que era improbable que lo hicieran indefinidamente. Más tarde sería aceptada por los mismos keynesianos \citep{blinder1997there,mankiw1991reincarnation}. En este sentido, los autores demuestran que la tasa de desempleo no podría mantenerse hasta cierto nivel, la cual se llamó "tasa natural de desempleo". Esta es explicada como la tasa de desempleo necesaria para mantener constante la tasa de inflación, y posteriormente sería conocida como la tasa de desempleo no aceleradora de la inflación o NAIRU (\textit{Non-Accelerating Inflation Rate of Unemployment}).\\

En los años 80 del siglo XX, Estados Unidos presenció inflación baja acompañado por un incremento temporal en el desempleo que alcanzaría el 11\% (reacción no aceleradora de la inflación). Posteriormente, desde 1983 en adelante, el desempleo cayó en 6\% hasta llegar al año 1987 con inflación del 4\%, lo cual lleva a contradecir la relación inversa entre desempleo e inflación, elementos contenidos en la curva de Phillips. \\

Así mismo, la tasa natural se empieza a tener en cuenta por parte de los bancos centrales después de la gran inflación ocurrida a finales de la década de los 60 y comienzos de los 70 del siglo XX. \cite{friedman1968role}   sugirió mantener una inflación baja, al tiempo que el nivel de empleo incrementara a largo plazo, en cuanto se tratara de reducir el desempleo más allá de la tasa natural tan solo resultaría en una inflación mayor. Por lo tanto, la postura de Friedman fue permitir que la política monetaria actuara de manera automática sin la interferencia del Estado.\\

Sin embargo, \cite{lucas1973some} -referente de los nuevos clásicos- estaría de acuerdo con la curva de Phillips a corto plazo si los agentes no pueden anticipar la inflación  debido a problemas de información, es decir, si en el caso opuesto, existen perturbaciones no anunciadas o no esperadas, el dinero puede tener efectos reales, violando la neutralidad sólo en el corto plazo, por lo tanto, la autoridad siempre estaría tentada a explotar el \textit{trade-off} entre inflación y desempleo. Entonces, el modelo clásico sería compatible con la curva de Phillips si el supuesto de la información completa es abandonado, combinado a la hipótesis de la tasa natural de Phelps y Friedman, con el supuesto que los mercados se vacían y la hipótesis de las expectativas racionales \citep{snowdon1994modern}. \\

Por consiguiente, el corto plazo se convertiría en uno de los principales focos de debate en curso sobre la política de demanda agregada, donde los problemas cruciales parecen depender de la curva de Phillips y su propiedad dinámica \citep{taylor1979staggered}. Por lo anterior, entre los temas centrales para la macroeconomía a corto plazo entraría el estudio de la naturaleza de la dinámica inflacionaria. En respuesta, importantes modelos teóricos surgen en su comprensión, originando investigaciones preliminares como \cite{calvo1983staggered}, \cite{taylor1980aggregate} y \cite{fischer1977long}. Los autores enfatizan la participación de la fijación escalonada de salarios y precios nominales por individuos y empresas con visión futura en la explicación de la inflación.\\
 
 Los modelos teóricos planteados aparecen para sentar bases a la competencia imperfecta y las rigideces nominales de variables como los precios y nivel de salarios a partir de los fundamentos microeconómicos. Sin embargo, estos aspectos ya habían sido conocidos en la teoría keynesiana al argumentar que de acuerdo a las fluctuaciones de la producción surgen en gran medida oscilaciones en la demanda agregada nominal, es decir, los cambios en la demanda tienen efectos reales debido a que los salarios nominales y los precios nominales son rígidos. No obstante, estos argumentos simplemente serian suposiciones en lugar de ser explicados como en los modelos de desequilibrio,\footnote{Las demostraciones teóricas de que los modelos keynesianos pudieron ser considerados por la microeconomía  no constituyen prueba alguna de que las teorías keynesianas fueran correctas. Por lo tanto, implicaciones empíricas en modelos de rigideces nominales han resultado ser débiles \citep{summers1991should}.}  o introducido a través de suposiciones teóricamente arbitrarias sobre los contratos laborales \citep{ball1988new}.\footnote{Para modelos de desequilibrio ver en \cite{barro1971general} y contratos laborales en \cite{fischer1977long}.} \\
 
Posteriormente, aparecerían autores de enfoque neokeynesiano (nueva síntesis neoclásica)  para desarrollar la microfundamentación de la curva de Phillips \citep{woodford1998control,goodfriend1997new,clarida1999science}, derivados en modelos macroeconómicos no sujetos a la critica de \cite{lucas1976econometric}.\footnote{La critica de \cite{lucas1976econometric} señalaba que cuando se trata de predecir los efectos de un gran cambio de política -como el que estaba considerando la Reserva Federal de los Estados Unidos (FED) en ese momento- puede ser muy engañoso considerar dadas las relaciones calculadas a partir de datos pasados, es decir, la curva de Phillips  suponía que los encargados de fijar los salarios seguirían esperando que la futura inflación fuera igual que la pasada.} Estos incorporan imperfecciones en los mercados y tipos de rigideces nominales,  tales como costos de menú, costos de ajuste en la inversión, precios escalonados, entre otros. De esta manera, otros trabajos como \cite{fuhrer1995inflation}, \cite{yun1996nominal}, \cite{king1999should} darían desarrollo y aplicación de la NKPC en diferentes modelos macroeconómicos a nivel teórico.\\
 
Por lo anterior, la NKPC se distinguiría -con respecto de la curva de Phillips tradicional- por ser microfundamentada y adicionar el efecto que tienen las expectativas racionales en la decisión que toma cada una de las empresas en el momento de fijar los precios. La NKPC toma importancia por ser elemento fundamental en el diseño de modelos de pronóstico de inflación de los bancos centrales en países que operan bajo inflación objetivo, por la razón que al considerar agentes con comportamientos racionales (\textit{forward-looking}), el efecto de la política monetaria sobre la inflación sería mayor, dado que los agentes ajustan sus precios con base en la senda esperada del producto y de los precios. \\

Así mismo, \cite{ball1988new} plantea que el nivel de precios se ajusta lentamente con el tiempo a un choque nominal, dado que la velocidad del acople depende de la frecuencia de ajuste de precios por empresas individuales, que a su vez se deriva de la maximización de las ganancias. La incidencia de los precios rígidos subyace en la medida que  pueden ser tanto privados eficientes como socialmente ineficientes. Por lo tanto, el ciclo económico resulta del ajuste subóptimo de los precios en respuesta a un choque de demanda. En este sentido, si la política puede estabilizar la demanda agregada, a su vez, podrá mitigar la pérdida social debido a este ajuste subóptimo \citep{mankiw1985small}. En respuesta, uno de los pioneros de la NKPC como lo es \cite{gali2010new}, manifestaría que las rigideces nominales sería el elemento clave de los modelos neokeynesianos y  la principal causa de la no neutralidad de la política monetaria.\\

En suma, los modelos neokeynesianos aportarían una nueva perspectiva sobre la naturaleza de la dinámica de la inflación. \cite{gali2002new} enfatiza las siguientes características: en primer lugar, por la condición prospectiva que tiene la inflación, sostiene que los precios son fijados por empresas que enfrentan limitaciones en la frecuencia con la que pueden ajustar el precio de los bienes que producen. Por lo tanto, los precios establecidos se caracterizarán por permanecer vigentes durante más de un periodo después de su imposición. Estas empresas encuentran un óptimo, dado que toman decisiones de fijar precios actuales según condiciones futuras de costo y demanda, ya que el nivel de precios agregados resulta ser producto de decisiones actuales de fijación de precios, y a su vez, un componente importante a futuro para el caso de la inflación. Por otra parte, los modelos neokeynesianos destacan el papel que desempeñan las variaciones en los márgenes, es decir, la incidencia de los costos marginales como fuente de cambio en la inflación agregada, en este sentido, es explicado por los intentos periódicos de las empresas para corregir desalineación entre los márgenes reales y no deseados.  \\

Estas propiedades se reflejan en la llamada NKPC, y sería después del trabajo de \cite{gali1999inflation} quienes por medio de la evidencia empírica, dan inicio a la estimación de un modelo estructural de la NKPC. Allí capturan la persistencia de inflación incurrida por la rigidez nominal bajo expectativas racionales para Estados Unidos (1960-1997). En seguida, \cite{gali2001european}  estudiarían a la Zona Euro (1970-1998). Por su  relevancia, la NKPC también sería abordada en el ámbito subnacional como en los trabajos de \cite{ha2003causes} y \cite{mehrotra2010modelling}, los cuales analizan principalmente la dinámica inflacionaria de las provincias de China. Por lo tanto, a partir de la teoría económica, estos modelos estructurales han ayudado a comprender el efecto que tiene la inflación frente a la política monetaria de algunos países, mientras que permite evaluar la incidencia de problemas estructurales e instituciones débiles en un escenario de persistencia de inflación.\footnote{La figura \ref{A2} recoge la red de colaboración de trabajos en la NKPC, donde se aprecia que \cite{gali1999inflation} y \cite{calvo1983staggered} son los más relevantes (Para mayor información ver apéndice \ref{apendicea} ).} 

\section{Derivación de la curva de Phillips neokeynesiana}\label{sec2}
El modelo de expectativas racionales  \citep{lucas1976econometric,sargent1976rational} dio paso a la microfundamentación neoclásica para los modelos macroeconómicos, desconociendo la curva de Phillips y la base de la economía keynesiana, es decir, la suposición de que la política monetaria podría afectar sistemáticamente la producción incluso a corto plazo. Por lo tanto,  los modelos neokeynesianos aparecerían para considerar la existencia de precios rígidos, el uso de ecuaciones desde fundamentos microeconómicos que  permitieran obtener modelos estructurales a partir del objetivo de agentes. Además, estos modelos incorporan expectativas racionales bajo un entorno de competencia monopolística (las firmas diferencian su producto), por lo que se pueden fijar precios.\\

Los precios rígidos se convierten en la principal razón microeconómica, dado que permite establecer períodos en donde los factores de producción como la mano de obra, se subutilizan, con una producción agregada por debajo de su llamado nivel potencial. Así mismo, un aumento en el stock de dinero puede generar un incremento a corto plazo en el poder adquisitivo real, y a su vez, permite el impulso de la producción real bajo un entorno de precios rígidos. Por otra parte, suponer rigideces de precios implica que no todos los mercados se están ajustando al instante y la producción agregada puede estar por debajo de los que se obtendría con precios flexibles  \citep{ball1988new} .\\

Este enfoque moderno que presenta expectativas racionales y alguna forma de microfundamentación se conoce como macroeconomía neokeynesiana. En este sentido, el siguiente apartado describe uno de los modelos clave neokeynesianos como lo es la NKPC, y se explora sus implicaciones para el comportamiento de la inflación y alguna variable de la actividad económica. Por último, se discute sobre la relación entre los costos marginales y la brecha del producto para dar explicación a la rigidez de precios y así mismo evaluar el mejor estimador de la NKPC.

\subsubsection*{Rigideces de precios a la Calvo}\label{secrig}
La derivación de la NKPC parte de suponer que las empresas se encuentran en competencia monopolística y cuentan con algún tipo de restricción en el ajuste de precios, es decir, las empresas establecen los precios a través de una regla de tiempo dependiente.\footnote{ La regla de tiempo dependiente se entiende como el grado de autonomía que tienen las empresas para cambiar sus precios y la existencia de políticas periódicas de revisión de precios. A pesar de que este escenario sea similar al modelo de contratos escalonados propuesto por \cite{taylor1980aggregate}, la diferencia subyace en que la decisión de fijación de precios evoluciona según el  problema de maximización de ganancias de un competidor monopolista.}  La formulación conocida como fijación de precios de \cite{calvo1983staggered} permite simplificar el problema de agregación según la fijación de precios dependientes al tiempo, lo cual evita realizar un seguimiento de los historiales de precios de las empresas. En este sentido, esta forma de rigidez de precios permite suponer que en cualquier periodo, la empresa tiene una probabilidad fija de $1-\theta$ para ajustar su precio en ese periodo, así mismo, con una probabilidad $\theta$ debe mantener su precio sin cambios en el tiempo promedio durante su fijación dado por $\sum_{k=0}^{\infty} (\theta\beta)^{k-1}=\frac{1}{1-\theta}$. Por ejemplo, para \cite{galvis2010estimacion}, con $\theta=0,807$, el 80\% de las empresas en Colombia mantienen los precios fijos en el tiempo durante cinco trimestres y aproximadamente el 20\% de las empresas fijan su precio al instante.\footnote{Para tener en cuenta que en el caso que la flexibilidad de precios sea $\theta=0$, la empresa ajusta su precio al instante, por lo tanto, solo el futuro es relevante cuando hay rigidez de precios, es decir, $\theta>0$. }  \\ 

Se tiene entonces que las empresas van a procurar maximizar una función de beneficios sujeta a la restricción en el ajuste de precios, sin poder incidir en los precios cada vez que lo deseen. De ese modo, la adaptación de esta conducta aparece de la siguiente manera. Se tiene que $p_{t+k}^*$ es el logaritmo del precio óptimo que la empresa fijaría en el período $t+k$ en el caso que no existiera rigideces, y $z_{t}$ si el precio se intenta fijar en $t$.  Dado el precio óptimo del siguiente período, la empresa va a tratar de minimizar sus desviaciones a partir de la siguiente función de pérdidas:
 \begin{equation}\label{1}
L=\sum_{k=0}^{\infty} (\theta\beta)^{k}E_{t}(z_{t}-p_{t+k}^*)^{2}
\end{equation}
%\eqref{1}
Sea $E_{t}(z_{t}-p_{t+k}^*)^{2} $ las pérdidas esperadas de beneficios de la empresa en el tiempo $p_{t+k}^*$, dado que las rigideces de ese periodo impiden el ajuste del precio óptimo.\footnote{La función cuadrática aproxima a una función de ganancia más general, aunque lo relevante aquí es la perdida de ganancia  de no contar con rigideces de precios para $z_{t}$.}  Así mismo, la sumatoria  $\sum_{k=0}^{\infty}$ expresa que el precio establecido tendrá implicaciones a futuro. En el caso de $\beta$, se considera como el factor de descuento subjetivo,  en este sentido, $\beta<1$ le dará mas peso a las pérdidas de hoy que a las  pérdidas futuras, a la vez que serán descontadas con $(\theta\beta)^{k}$. Por lo anterior, la adición de $\theta$ permite descontar las pérdidas según la probabilidad en que las empresas no mantengan el precio fijo hasta el siguiente período.

\subsubsection*{Precio óptimo}\label{secpre}
Para encontrar el valor óptimo de $z_{t}$, es decir, el precio elegido por las empresas que pueden establecer, cada uno de los términos con la variable $z_{t}$ ($(z_{t}-p_{t+k}^*)^{2} $) se diferencia con $z_{t}$ para definir la suma de la derivación que será igual a cero. Ahora, se continua con la optimización de la función de pérdida.\\
 
 Condiciones de primer orden:
\begin{equation}\label{2}
L(z_{t})=2\sum_{k=0}^{\infty} (\theta\beta)^{k}E_{t}(p_{t+k}^*)=0
\end{equation}
En seguida, se separa en dos términos la ecuación \eqref{2}:
\begin{equation}\label{3}
\left[\sum_{k=0}^{\infty}(\theta\beta)^{k} \right] z_{t}=\sum_{k=0}^{\infty}(\theta\beta)^{k}E(p_{t+k}^*)
\end{equation}
Usando la suma geométrica para resolver el lado izquierdo de la ecuación \eqref{3}, se tiene que:
\begin{equation}\label{4}
\left[\sum_{k=0}^{\infty}(\theta\beta)^{k} \right]=\frac{1}{1-\theta\beta}
\end{equation}
Reescribiendo, se obtiene la forma siguiente:
\begin{equation}\label{5}
\frac{1}{1-\theta\beta}=\sum_{k=0}^{\infty}(\theta\beta)^{k}E(p_{t+k}^*)
\end{equation}
Resolviendo la ecuación \eqref{5}, se tiene que el precio óptimo es:
\begin{equation}\label{6}
z_{t}=(1-\theta\beta)\sum_{k=0}^{\infty}(\theta\beta)^{k}E(p_{t+k}^*)
\end{equation}
El precio óptimo señala que la empresa establece su precio similar a un promedio ponderado de los precios que hubiera esperado establecer en el futuro si no hubiera rigideces de precios. Por lo tanto, como no puede cambiar el precio en cada período, la empresa opta por mantenerse cerca del promedio del precio correcto o sin rigideces \citep{whelan2009lecture}. 

\subsubsection*{Costos marginales y \textit{Mark-up}}\label{seccioncm}
En la incertidumbre  del precio óptimo sin fricción ($p_{t}^*$), se asume que la estrategia de fijación de precios óptimos de la empresa sin fricciones implicaría establecer los precios como un \textit{mark-up} fijo sobre el costo marginal:\footnote{\textit{Mark-up} o margen, es un índice económico aplicado sobre el coste de producción e distribución de un producto o servicio para definir el precio.}
 \begin{equation}\label{9}
P_{t}^*=\left( \frac{\epsilon}{\epsilon-1}\right) E_{t-1}\frac{N_{t}^{j}W_{t}}{\alpha Y_{t}^{j}}
\end{equation}
Sea $\epsilon$ la elasticidad del precio de la demanda a la que se enfrenta la firma en el mercado y $(\frac{\epsilon}{\epsilon-1})=\mu$ el \textit{mark-up} sobre los costos marginales puestos por la empresa.\footnote{Tener en cuenta  que se parte inicialmente de una función de producción Cobb-Douglas de tipo $ Y_{t}^{j}=A(N_{t}^{j})^{\alpha}$, sin emplear el capital como factor de producción. La completa derivación del  \textit{mark-up} y los costos marginales se pueden apreciar en el apéndice \ref{apendiceb}.} Los costos marginales ($\frac{N_{t}^{j}W_{t}}{\alpha Y_{t}^{j}}$) son constituidos por el nivel de salarios pagados ($W_{t}$) al contratar $N_{t}^{j}$ empleados necesarios para la producción sobre la cantidad de demanda (y vendida) del producto que la empresa $j$ ofrece ($Y_{t}^{j}$).\footnote{La aproximación de los costos marginales corresponde a \cite{gagnon2005new} como se presentan en \cite{galvis2010estimacion} y \cite{cespedes2005new}. Otras aproximaciones se encuentran en \cite{woodford2011interest}.}\\

Reescribiendo la ecuación \eqref{9}, se tiene la forma siguiente:
  \begin{equation}\label{10}
P_{t}^*=\mu X_{t}^{j}
\end{equation}
 Siendo $X_{t}^{j}=E_{t-1}\frac{N_{t}^{j}W_{t}}{\alpha Y_{t}^{j}}$ los costos marginales nominales.\\
   
Aplicando logaritmos en  \eqref{10}  se tendría:
 \begin{equation}\label{11}
lnP_{t}^*=ln \mu X_{t}^{j}
\end{equation}
 \begin{equation}\label{12}
p_{t}^*=\mu +mc_{t}
\end{equation}
Donde $p_{t}^*=ln P_{t}^*$, $\mu^{*}=ln \mu$ y el costo marginal es igual a $mc=n_{t}^{j}+w_{t}-\alpha y_{t}^{j}$ con variables en logaritmos.\\

Retomando la ecuación \eqref{6}, se introduce la ecuación \eqref{12} para obtener   el precio óptimo con la presencia del  \textit{mark-up} sobre los costos marginales:
%el precio óptimo que la empresa fijaría con base en expectativas sobre el precio futuro y el margen sobre los costos marginales:
\begin{equation}\label{13}
z_{t}=(1-\theta\beta)\sum_{k=0}^{\infty}(\theta\beta)^{k}E(\mu +mc_{t+k})
\end{equation}
Al resolver la sumatoria de manera iterativa la ecuación \eqref{13} se tiene que: 
\begin{equation}\label{14}
z_{t}=\theta\beta Ez_{t+1}+(1-\theta\beta)(\mu +mc_{t}) 
\end{equation}
Por lo anterior, cada empresa fijaría su precio según las expectativas sobre el precio del futuro y el margen sobre los costos marginales.

\subsubsection*{La curva de Phillips neokeynesiana}
En adición, faltaría por agregar los resultados de la fijación de precios de cada empresa -dada en la ecuación \eqref{9}-  en la economía. Para este problema, la estructura de precios del trabajo de \cite{calvo1983staggered} permite que el nivel de precios agregados se determine como una combinación convexa del nivel de precios rezagados y el nuevo precio óptimo. Por lo anterior, se tiene que:\footnote{Las variables presentadas en la ecuación  \eqref{15} se encuentran en logaritmos.} 
\begin{equation}\label{15}
p_{t}=(1-\theta)z_{t}+\theta p_{t-1}
\end{equation}
Cada variable está expresada como porcentaje de desviación según un nivel de inflación cero. En este caso $\theta$ corresponde a la probabilidad de no alterar el precio, independiente del tiempo desde la última revisión, esta probabilidad se presenta en $\theta p_{t-1}$. Por su parte, la probabilidad en que las empresas vuelven a fijar el precio es ($1-\theta$), captado en  $(1-\theta)z_{t}$.\\

Ahora, despejando la ecuación \eqref{15}, el precio óptimo que fijan las empresas puede representarse así:
\begin{equation}\label{16}
z_{t}=\frac{1}{(1-\theta)}(p_{t}-\theta p_{t-1})
\end{equation}
Al igualar las ecuaciones \eqref{16} y \eqref{14}, se tiene que: 
\begin{equation}\label{17}
\frac{1}{(1-\theta)}(p_{t}-\theta p_{t-1})=\theta\beta Ez_{t+1}+(1-\theta\beta)(\mu +mc_{t}) 
\end{equation}
Y al sustituir $z_{t+1}\frac{1}{(1-\theta)}(p_{t+1}-\theta p_{t})$ en \eqref{17} se obtiene:
\begin{align}
\frac{1}{(1-\theta)}(p_{t}-\theta p_{t-1})&=\frac{\theta}{(1-\theta)}\beta (E_{t}p_{t+1}-\theta p_{t})+(1-\theta\beta)(\mu +mc_{t}) \\
(p_{t}-\theta p_{t-1})&=\theta\beta (E_{t}p_{t+1}-\theta p_{t})+(1-\theta\beta)(1-\theta)(\mu +mc_{t}) 
\end{align}
Reordenando y teniendo en cuenta que la tasa de inflación se define como $\pi =p_{t}-p_{t-1}$, se tiene la siguiente aproximación a la NKPC:
\begin{equation}\label{18}
\pi_{t}=\beta E_{t}(\pi_{t+1})+\frac{(1-\theta\beta)(1-\theta)}{\theta}(\mu +mc_{t}-p_{t})
\end{equation}
Así pues, se definirá $mcr_{t}=\mu +mc_{t}-p_{t}$ como el costo marginal real según el nivel de estado estacionario, en otras palabras, $mcr_{t}$ será la log-linealización del costo marginal real. Retomando  la \eqref{18}, si $\lambda =\frac{(1-\theta\beta)(1-\theta)}{\theta} $, se obtiene finalmente la NKPC: 
\begin{equation}\label{19}
\pi_{t}=\beta E_{t}(\pi_{t+1})+\lambda(mcr_{t})
\end{equation}
La ecuación de la NKPC establece que la inflación está compuesta por dos factores: en primer lugar, por la tasa de inflación esperada para el próximo periodo ($\beta E_{t}\pi_{t+1}$), y por otra parte,  la brecha entre el nivel de precio óptimo sin fricción ($\mu +mc_{t}$) y el nivel de precio actual ($p_{t}$). En este sentido, la inflación dependería de manera positiva con respecto a los costos marginales ($mc_{t} - p{t}$).\\

El aporte del modelo de \cite{calvo1983staggered} le permite a las empresas que mantengan su precio como un margen fijo sobre el costo marginal. Las presiones inflacionarias surgen a medida que la relación entre costo marginal y precios se vuelve alto, debido a que las empresas reestablecen en promedio sus precios en mayor cuantía. Además, \cite{gali1999inflation} agrega a la NKPC  el elemento $\lambda$ para explicar el decrecimiento en  $\theta$, es decir, la inflación seria menos sensible a movimientos en los costos marginales según la incidencia de las rigideces de precios.\\

Dentro de los aspectos relevantes que han caracterizado la NKPC, ha sido la participación de la brecha de los costos marginales en la inflación en lugar de la brecha del producto. La razón principal se debe a que al incorporar a las empresas en este modelo, intentaran mantener un margen de beneficio fijo sobre el costo marginal, en consecuencia provocarían inflación si las empresas procuran fijar nuevos precios por el motivo del declive del margen sobre los costos. No obstante, se asume que las rigideces de los precios y su evolución estarán acordes a los productores. La presencia de los participantes de la oferta explicaría que la dinámica inflacionaria parte de los costos de producción \citep{galvis2010estimacion}.

\section{Costos marginales y ciclo económico}\label{seccm}
Al recordar la curva de Phillips con expectativas adaptativas de los años 70 y 80 del siglo XX,\footnote{Teniendo en cuenta el anexo de  \cite{friedman1968role} y \cite{phelps1967phillips}. } es posible asumir que la brecha del producto ($y_{t}$) pueda ser vinculada en la NKPC al igualar $\lambda(mcr_{t})=k(y_{t})$ en la ecuación \eqref{19}. Así pues, se tiene que la NKPC modificada es:\footnote{También  la ecuación que compone a $k(y_{t})$ puede representarse a partir de un modelo de equilibrio general neokeynesiano, en donde $\varphi $ es el coeficiente de aversión al riesgo y $ \sigma$ es la inversa de la elasticidad precio de la oferta de trabajo. En suma, se da el grado de apertura de la economía $k$. En este sentido se supone: $\lambda(mcr_{t})=(\varphi + \sigma )y_{t}=k(y_{t})$.} 
\begin{equation}\label{20}
\pi_{t}=\beta E_{t}(\pi_{t+1})+k(y_{t})
\end{equation}

En los últimos años, ha sido discutido asumir el anterior supuesto principalmente por la estimación empírica que representa. Para \cite{gali1999inflation}, la brecha del producto usado en los modelos teóricos de la NKPC  es diferente a los tomados en las estimaciones empíricas, por la razón que el Producto Interno Bruto (PIB) sin tendencia o filtrada como \textit{proxy} de la brecha del producto, no es un enfoque adecuado debido a que la producción potencial es representada como una función suavizada en el tiempo. Así mismo, dichos autores afirman que la producción potencial en teoría enfrenta fluctuaciones volátiles debido a choques diferentes a los monetarios. Aunque se conoce de dichas falencias en las estimaciones empíricas, aún no hay un criterio teórico  para evaluar la brecha del producto.\\

La dificultad de detectar el efecto significativo de la actividad real (medido en la brecha del producto) sobre la inflación, permitió la búsqueda de otra variable que explicara la dinámica inflacionaria. La aceptación que tendría el costo marginal en lugar de la brecha del producto, origina características deseables en la explicación directa del impacto de las ganancias de productividad en la inflación. Este hecho lo pasaba por alto al adicionar la brecha del producto. De otra manera, la adición de un conjunto de empresas que cuentan con una fijación de precios en una regla empírica retrospectiva, implicó tener en cuenta la persistencia observada de la inflación.\\

El aspecto fundamental en que se basa la estimación de la NKPC es el vínculo entre la actividad agregada y los costos marginales. El factor común se encuentra contenido en los costos laborales unitarios, pero son los costos marginales quienes se van a caracterizar por retrasar la producción durante el ciclo en lugar de moverse al mismo tiempo, en contraste con la predicción del marco macroeconómico estándar de precios fijos.\footnote{Para mayor información frente al marco macroeconómico estándar de precios fijos ver en \cite{christiano1997sticky} }  Por lo tanto, la inercia de la inflación puede estar explicada por el ajuste lento de los costos marginales según los movimientos de la producción.\\

La NKPC al adicionar los costos marginales, revela importantes novedades en el entendimiento de las causas de la inflación. Entre una de ellas está el hecho en que la inflación tiene que ver con el margen superior que establecen las empresas, es decir, al estar sujeto este margen a la elasticidad del mercado, la inflación dependerá de la coyuntura que atraviesa la economía en cada momento. En este sentido, en épocas de auge aumenta el margen sobre los costos y por tanto la inflación, y en épocas de crisis las empresas bajan el margen para deshacerse de inventarios y baja por lo tanto la presión al alza de los precios disminuyendo así la inflación \citep{galvis2010estimacion}.\\

Estas consideraciones han sido evidenciadas en estudios previos realizados al análisis de empresas en la Zona Euro y más reciente en las empresas de Colombia. El patrón de resultados de los trabajos de  \cite{fabiani2005pricing} y \cite{misas2009formacion},\footnote{Para el análisis en la formación de precios para Colombia se tuvieron en cuenta 4626 empresas \citep{misas2009formacion} y para la Zona Euro más de 11000 empresas \citep{fabiani2005pricing}.}  respaldan la reciente ola de estimaciones de versiones híbridas de la NKPC, debido a que las empresas tienen algún tipo de poder de mercado y pueden establecer sus precios por encima de los costes marginales. Esto también sugiere que los modelos con competencia monopolística, como los modelos neokeynesianos, pueden ser una mejor descripción para la mayoría de los mercados de bienes y servicios que aquellos que suponen una competencia perfecta. \\

El comportamiento de las empresas y la manera en que fijan sus precios determinan la forma en que las decisiones de política monetaria afectan a la economía en general, es decir, el grado y tipo de rigidez que presentan los precios afectan el impacto de los cambios de las tasas de interés sobre la inflación y el producto \citep{misas2009formacion}. De esta manera, se evidencia que el comportamiento de las variables monetarias tiene impacto sobre las variables reales, en contradicción con el postulado central de la nueva macroeconomía clásica.\\

Por otra parte, estas encuestas recalcan que el tamaño de la empresa tendrá un comportamiento distinto frente al momento de revisar los precios, en este sentido, entre mayor sea el tamaño de las empresas sus decisiones sobre los precios tendrán un comportamiento  \textit{forward-looking}, es decir, actuarán de manera racional con la información futura, lo cual se le da más énfasis a la meta de inflación que al salario mínimo, mientras que para las no grandes es más relevante el salario mínimo. Esto explica que las empresas con mayor poder de mercado enfatizan sus decisiones actuales en los precios vistas desde el futuro, característica principal de la NKPC.\\

A escala sectorial, \cite{misas2009formacion}  encuentran que, tanto para la agricultura como para la industria, la información presente tiene una mayor relevancia. Para el caso de la pesca, la información futura compite en jerarquía con la información presente. Tanto la inflación presente como la esperada son importantes para la revisión de precios por parte de las empresas colombianas, si se miran tanto por tamaño como por sectores. Sin embargo, en términos relativos, los autores encuentran que la última es más importante que la primera en el caso de industria y pesca. Lo anterior toma importancia por el hecho en que la  heterogeneidad en el comportamiento de fijación de precios principalmente entre los sectores -y entre regiones según su composición económica- no solo complica la conducción de la política monetaria, sino que también afecta el mecanismo de transmisión de la política monetaria \citep{romero2008transmision}.\\

Para el caso del ciclo económico de la NKPC, \cite{gali1999inflation}  destacan la participación del ingreso laboral como \textit{proxy} en la creación del costo marginal real. Los autores encuentran que los costos marginales reales son un determinante significativo y cuantitativamente importante de la inflación. La acogida que tuvo este hallazgo fue masiva en la medida en que el modelo fue exitoso. Sin embargo, varios autores han cuestionado este \textit{proxy} de los costos marginales debido a que los ingresos laborales son un costo promedio y no un costo marginal \citep{rudd2007modeling}, y porque el ingreso laboral actúa de manera anticíclica \citep{mazumder2010new}.\\

La participación del ingreso laboral es anticíclica en el sentido en que aumenta durante los tiempos de recesión, contrario a lo que la intuición y la teoría nos dicen sobre el costo marginal. La teoría microeconómica estándar predice que el costo marginal a corto plazo debería ser procíclico, según autores como \cite{bils1987cyclical} y \cite{rotemberg1999cyclical}. En este sentido, una expansión en la economía generaría que las empresas aumentarán la producción, lo cual lleva a que la curva del costo marginal a corto plazo sea inclinada hacia arriba, teniendo en cuenta que algunos factores de producción permanecen fijos. Existe un consenso de trabajos empíricos que evidencian que los costos marginales tienen una pendiente ascendente a corto plazo, aunque queda en entredicho el grado de su pendiente  \citep{mazumder2010new}. En el caso contrario, si la recesión genera una reducción en la producción, el descenso de los costos conlleva a la disminución marginal en la producción.\\

Inicialmente, este problema esencial ya se había enfatizado por \cite{fuhrer1995inflation}, los cuales sostuvieron que la NKPC de referencia implicaba que la inflación debería asemejarse al comportamiento del ciclo de la brecha del producto,\footnote{Cabe recordar que lo mencionado anteriormente hace énfasis en que los costos marginales explican en gran medida la dinámica de la inflación que la misma brecha del producto.} es decir, por ejemplo, un aumento en la inflación actual esperaría que se diera un aumento en la brecha del producto, lo cual en caso opuesto se podría dar en los datos. Por tal motivo, \cite{gali1999inflation} demuestran por medio de una correlación cruzada que la  brecha de producto actual se mueve positivamente con la inflación futura y negativamente con la inflación rezagada, consistente con la antigua teoría de la curva de Phillips, pero en contradicción directa con la NKPC. Esta verificación del comportamiento de los ciclos se realiza a manera de visualizaciones y por medio del cálculo de correlaciones simples entre  la inflación, el ingreso laboral y/o la brecha del producto. \\

Por lo anterior, \cite{gali1999inflation}  establecen que los costos marginales reales serían la medida más consistente para explicar la inflación, apoyado por las características previamente explicadas y por su condición acertada en la teoría, es decir,  actuaban de manera procíclica los costos marginales, omitiendo así  el caso hipotético en que los datos tomados fueran opuestos a su requerimiento. Sin embargo, una posible solución a dicho inconveniente es la idea de dejar que la inflación dependa de una combinación convexa de la inflación futura esperada y la inflación rezagada, es decir,  adicionar rezagos para capturar la persistencia de la inflación que no se explica en el modelo de referencia. Por lo tanto, ahora la ecuación \eqref{19} tendría la siguiente forma:\footnote{En este caso $\beta$ puede actuar de manera parcial entre $\pi_{t+1}$ y $\pi_{t-1}$} 
\begin{equation}\label{21}
\pi_{t}= \gamma_{f}E_{t}(\pi_{t+1})+\gamma_{b}(\pi_{t-1})+\lambda(mcr_{t})
\end{equation}
 
\section{Evidencia empírica: Revisión de resultados}\label{sec3}
\subsection{Nivel internacional}\label{s141}
Los estudios preliminares sobre el NKPC híbrido, como el de \cite{fuhrer1995inflation}  continuaron utilizando la brecha del producto como la principal variable impulsora de la inflación, pero  \cite{gali1999inflation}, sugirieron usar el costo marginal real con base en la participación del ingreso laboral. La aceptación que tuvo los costos marginales sobre la dinámica inflacionaria en los Estados Unidos dio lugar al uso de modelos dinámicos de equilibrio general en el ámbito monetario, a partir de modelos derivados de fundamentos microeconómicos que explicarían procesos inflacionarios.\\

Los resultados  obtenidos por \cite{gali1999inflation} muestran que ambos parámetros son significativos ($\beta, \lambda$). Además, se encuentran que $\theta=0.829$, infiriendo que el 82,9\% de las empresas dejan fijos los precios en promedio durante cinco trimestres. Así mismo, se explica que alrededor del 17\% de las empresas ajustan su precio según valor actual del costo marginal real en la economía de Estados Unidos para el periodo comprendido de 1960 a 1997 (en datos trimestrales).\footnote{El calculo de $\theta$ se encuentra resolviendo la ecuación \eqref{18} en $\lambda$. En seguida, el promedio del período fijo se calcula como $\frac{1}{1-\theta}$, junto con el tamaño de las empresas que ajustan el precio según los costos marginales, $1-\theta$.} Los autores encuentran que los costos marginales reales son de hecho un determinante estadísticamente significativo y cuantitativamente importante de la inflación, como lo predice la teoría. Esto ha motivado a otros autores a estimar la NKPC en diferentes países (tabla \ref{t1}).

\begin{table}[H]
  \centering
  \caption{Resultados estimaciones de la NKPC nivel internacional}
    \begin{tabular}{ c  c c c c c }
      \hline
        País  & $\beta$ & $\lambda$ & $\theta$ & $\frac{1}{1-\theta}$ & Fecha \\
            \hline
              \hline
    Estados Unidos \dag & 0.926 & 0.047 & 0.829 & 5.8   & 1960:Q1-1997:Q4 \\
    Zona Euro \dag \dag & 0.914 & 0.088 & 0.771 & 4.4   & 1970:Q1-1997:Q4 \\
    Australia * & 0.942 & 0.113 & 0.73  & 3.7   & 1962:Q1-2000:Q4 \\
    Chile ** & 0.946 & 0.385 & 0.553 & 2.2   & 1990:Q1-2004:Q4 \\
      \hline
    \end{tabular}%
  \label{t1}\\
  \raggedright  \scriptsize \textbf{Nota:}  \cite{gali1999inflation}\dag, \cite{gali2001european}\dag\dag, \cite{neiss2005inflation}*, \cite{cespedes2005new}**.   
\end{table}%

Entre estas estimaciones, autores como \cite{neiss2005inflation} se concentran en gran parte por analizar la estabilidad de los parámetros de la NKPC y ampliar la discusión sobre la relación de los costos marginales y la brecha del producto en países como Reino Unido, Estados Unidos y Australia. Además, los autores adicionan a manera de variables \textit{dummy}   reformas que incidieron en el desarrollo normal de sus economías. No obstante, sus resultados no reflejan distanciamiento de los trabajos vistos en la tabla \ref{t1}.\\

Al mismo tiempo, \cite{cespedes2005new} evidencian en una economía emergente como la chilena, el coeficiente de \cite{calvo1983staggered} desciende a un rango de 0.55 a 0.80, sin salirse del rango de 2 a 5 trimestres  en duración promedio en que los precios permanecen sin cambios. Por otra parte, este estudio tiene la particularidad de respaldar la hipótesis de la existencia de una ruptura estructural en el NKPC, por motivo de la convergencia  a un objetivo de inflación de largo plazo. Esto evidenció que el proceso inflacionario tuviera una mirada de tipo \textit{forward-looking} en los últimos años, lo cual explica  mayor credibilidad de la meta de inflación.\\% la ruptura se da en el 2000
 
Otros autores interesados en estudiar la formación de precios, se han concentrado en analizar principalmente la incidencia de las expectativas adaptativas y racionales en la dinámica inflacionaria, partiendo que en el trabajo preliminar de  \cite{gali1999inflation} se encontró que  el comportamiento prospectivo puede proporcionar una descripción razonablemente lógico. \cite{vavsivcek2011inflation} encuentra para cuatro países de la Unión Europea, pruebas sólidas de que la inflación está determinada por las expectativas de inflación futura, aunque con un mayor grado de persistencia de la inflación que el encontrado en economías desarrolladas. Los autores intuyen que este fenómeno se debe a la gran cantidad de empresas que aún fijan precios de manera simple y retrospectiva, consistente con las expectativas adaptativas.\footnote{El estudio comprende un periodo libre de cambios importantes en los regímenes de política monetaria y durante el cual las series de inflación no estuvieron sujetas a una ruptura estructural.} \\ 

Para México, \cite{ramos2008inflation} evidencian de 1992 a 2007, tanto los componentes hacia atrás como los prospectivos, son importantes para explicar la dinámica de la inflación a corto plazo. Aunque las expectativas de inflación son un determinante importante de la inflación, la inflación rezagada (persistencia de la inflación) también juega un papel clave. Además, en la submuestra que realizan los autores para los años 1997-2007, las estimaciones para los coeficientes muestran alta importancia de las expectativas racionales (prospectiva) para la inflación y su vínculo con los costos marginales.\\

Por otra parte,  \cite{leith2007estimated},  \cite{rumler2007estimates} y \cite{mihailov2011small}  se basan en la  versión de economía pequeña y abierta de la NKPC derivada de \cite{gali2005monetary}. Los autores coinciden en que la tasa de inflación en las economías pequeñas y abiertas está impulsada por expectativas sobre factores externos en un grado sustancial. \cite{mihailov2011small} encuentran que para la mayoría de la muestra de los países  que integran la Organización para la Cooperación y el Desarrollo Económico (OCDE),  los términos de intercambio surgen como el  factor fundamental que impulsa la inflación.\\

Otros factores como el tamaño específico, la estructura de producción y/o los patrones comerciales de un país, así como las tendencias mundiales pueden lograr una influencia más fuerte o más débil de factores externos versus nacionales. Por lo anterior,  modelos alternativos del comportamiento de fijación de precios de las empresas o de rigideces reales en un entorno internacional también han incursionado en la explicación de la dinámica inflacionaria.\\

\begin{table}%[H]
  \centering
    \caption{Resumen de los enfoques de estimación en la literatura}
  \resizebox{16.5cm}{!} {
    \begin{tabular}{p{3cm} p{3cm} p{5cm} p{5cm} p{5cm}}
    \hline
  Articulos & Enfoque de estimación & Expectativa vs rezagos & Significancia & ¿Es rechazado el modelo?  \\
 \hline
  \hline
\cite{gali1999inflation}, \cite{gali2001european}, \cite{gali2005monetary}.    & RE GIV. & El comportamiento \textit{forward-looking} (prospectivo) es dominante, pero el término \textit{backward-looking} (retrospectivo) es significativo. & Significativamente positivo para la participación laboral. & No, basado en la prueba de identificación excesiva y el ajuste visual. \\ \\
     \cite{fuhrer1995inflation}, \cite{fuhrer1997importance}.  & RE VAR-ML. & La fijación de precios no es muy prospectiva; Necesita una gran persistencia intrínseca. & Positivo tanto para la participación laboral como para la brecha del producto, pero la importancia varia. & Rechazo puro de la NKPC con base en la prueba LR y los IRF. \\ \\
     \cite{roberts1995new}, \cite{roberts2005well}.       & GIV, VAR-ML, IRF correspondencia IRF; RE y encuestas de pronósticos. & Los pronósticos de encuestas lentas imparten la persistencia necesaria. Para RE, necesita más de un rezago de inflación. & Positivo tanto para la participación laboral como para la brecha del producto, pero la importancia varia. & No. \\ \\
     \cite{sbordone2002prices}, \cite{sbordone2005expected}.    & RE VAR-MD. & El comportamiento prospectivo es claramente dominante, pero el rezago es significativo. & Positivo pero marginalmente insignificante en el modelo híbrido. & No, basado en una prueba de identificación excesiva y ajuste visual. \\ \\
     \cite{rudd2005new}, \cite{rudd2007modeling}.       & RE GIV (iterado) & Inflación rezagada muy significativa. & Ni la participación laboral ni la brecha del producto agregan poder explicativo. & Si, forzar variable no ayuda a explicar la inflación. \\ \\
    \cite{rudebusch2002assessing}.       & OLS; pronósticos de encuestas & Cuarto trimestre de MA de la inflación rezagada recibe un peso mayor al previsto. & Coeficiente de brecha de producto positivo y significativo. & No. \\ \\
    \cite{ravenna2006optimal}.       & RE GIV, tasa de interés agregada a NKPC. & (NKPC puro) & (No estimado directamente). & No, basado en una prueba de identificación excesiva. \\ \\ 
    \cite{cogley2008trend}.       & Estimación bayesiana usando VAR con parámetros de deriva y volatilidad estocástica. & Término retrospectivo insignificante una vez que se tiene en cuenta la tendencia de la inflación. & (No estimado directamente). & No, según el ajuste visual y la magnitud de los errores de pronóstico. \\ \\ 
           \hline
    \end{tabular}%
 }
  \label{e2}\\
  \raggedright  \scriptsize \textbf{Nota:} Las siglas de la Tabla \ref{e2} indican lo siguiente:  expectativas racionales (RE), Variables instrumentales generalizadas (GIV), vectores autoregresivos (VAR), máxima verosimilitud (ML), mínima distancia (MD) función impulso respuesta (IRF), mínimos cuadrados ordinarios (OLS),  prueba de ratio de verosimilitud (LR), media móvil (MA).\\
  \cite{mavroeidis2014empirical} recoge los trabajos más importantes en la  literatura empírica de NKPC. Estos son clasificados por los autores según el número de citas de Google Scholar hasta mediados de septiembre de 2012.
\end{table}%

En resumen, \cite{mavroeidis2014empirical} examina  la literatura empírica sobre la NKPC.\footnote{La figura \ref{A3} recoge la red de histórica de citas de la NKPC, donde se aprecia que \cite{mavroeidis2014empirical} recoge gran parte de la producción académica (Para mayor información ver apéndice \ref{apendicea} ).} En este trabajo agrupa las diversas contribuciones en los principales enfoques econométricos,  resultados de algunos de los estudios más frecuentemente citados. Así mismo, los autores asocian los principales puntos de controversia en la literatura correspondientes a la importancia relativa del comportamiento de fijación de precios a futuro y hacia atrás, como también al grado en que la actividad real influye en la dinámica de la inflación. Estos aspectos pueden observarse en la tabla \ref{e2}.\\

A pesar que la NKPC ha tenido la capacidad de explicar la dinámica inflacionaria, las economías regionales aún siguen sin ser estudiadas. La importancia de la presencia de dicho estudio  radica en que el desempeño económico, diferencias institucionales y diferentes grados de desarrollo del mercado entre departamentos van a permitir diagnosticar el comportamiento de los costos marginales y la inflación.\\

No obstante, en los últimos años China se ha concentrado en analizar el proceso inflacionario de sus provincias aún cuando se ha prestado menor atención, teniendo en cuenta que el país asiático busca desarrollar e implementar una política monetaria independiente por la vía de la adopción de estabilidad de precios. La efectividad de su política monetaria está sujeto a la dinámica de la inflación y su posterior vinculación bajo un entorno regional heterogéneo como el de China.\\

Por lo anterior, recientes trabajos de la NKPC han avanzado para dar entendimiento a los procesos inflacionarios en la economía regional. En este sentido, con datos anuales en el período 1982-2002 en China, \cite{funke2006inflation}  desarrolla una NKPC con el modelo estándar, es decir, incorporación de expectativas de inflación, rezagos en la inflación y  la brecha del producto, en lugar de los costos marginales reales. Dichos componentes van a tener coeficientes consistentes, con excepción de la brecha del producto, la cual presentaría insignificancia estadística. Además, el autor considera variables instrumentales como la tasa de inflación rezagadas y las brechas de producto, el precio real del petróleo y el tipo de cambio efectivo nominal para el control de problemas de endogeneidad. \\

En el mismo periodo, pero en datos trimestrales y por medio de encuestas, \cite{scheibe2005phillips} evidencian que la NKPC se ajusta mejor al futuro que al pasado. Por otra parte, \cite{ha2003causes} encuentran que la NKPC representó mejor la dinámica de inflación que la Curva de Phillips convencional en China para el periodo 1989-2002.\footnote{Investigaciones anteriores con datos regionales que estimaron la Curva de Phillips pueden encontrarse en \cite{coen1999nairu,hassler2003inflation,dinardo1999phillips}, para  44 áreas metropolitanas en los Estados Unidos, estados alemanes y 9 países de la OCDE, respectivamente.}    Adicionalmente, sus hallazgos indican que la deflación o la baja inflación, reflejó un rápido crecimiento de la productividad.\\

Para \cite{mehrotra2010modelling}, 22 de las 29 provincias presentan significancia estadística en la brecha del producto y en el componente de inflación esperada para el periodo 1978-2004 (datos anuales), siendo variables importantes para el proceso de formación de inflación en China. Particularmente, estas provincias se ubican en la costa de China  y comparten características comunes como las de ser más abiertas al comercio internacional y poseer un porcentaje más bajo de empresas controladas por el estado en su producción total.  Por lo tanto, los autores concluyen que bajo un entorno de inflación muy baja, el comportamiento prospectivo de los agentes puede ser beneficioso para estimular la economía.\\%coclusiones de mehr

Otros trabajos encargados de estimar la NKPC ha escala regional han surgido con diferentes métodos al conocido GMM. \cite{yesilyurt2014regional} emplea un enfoque econométrico espacial  para estimar la NKPC en 67 provincias de Turquía (1987-2001). Los autores consideran que el comportamiento prospectivo es más importante que el comportamiento retrospectivo  si la tasa de inflación esperada está instrumentalizada por un rango de variables instrumentales. Además, evidencian significancia a favor de la convergencia, es decir, cuanto mayor sea la inflación atrasada en la propia provincia, o la inflación atrasada más baja en las provincias vecinas, menor será la tasa de inflación actual, lo cual respalda  procesos de integración regional de las tasas de inflación en Turquía. \\

De manera más reciente, \cite{saygili2020sectoral}  utiliza el enfoque de los errores estándar corregidos de los paneles heteroscedasticos de regresión Prais-Winsten (PCSE) para el análisis de la dinámica inflacionaria entre países de la OCDE (1990-2016). La estimación de los coeficientes evidencia variaciones entre sectores, lo cual explica las enormes diferencias de la respuesta sectorial en términos de política monetaria. En adición, los tamaños de los coeficientes estarían asociados al grado de integración a las cadenas de valor globales. Por lo anterior, entre diferentes métodos de estimación, series de tiempos y las diferencias entre regiones y sectores  dan muestra de importantes resultados al analizar el proceso inflacionario.  Por tal razón, estos antecedentes hacen viable la posibilidad de estimar la NKPC  para el análisis de la dinámica inflacionaria en los departamentos en Colombia bajo un entorno de inflación baja.

\subsection{Nivel nacional}
En Colombia,  \cite{bejarano2005estimacion} comprueba empíricamente la NKPC para el periodo 1984 a 2002. El autor encuentra que la inflación y los costos marginales tienen una relación positiva a partir del modelo neokeynesiano de optimización dinámica planteado por \cite{gali1999inflation}. Estos parámetros estructurales son consistentes con los del modelo y a su vez con los encontrados a escala internacional.\\

\cite{bejarano2005estimacion} evidencia que la inflación trimestral responde a cambios futuros de la brecha de los costos marginales, lo cual implicaría que los agentes tengan expectativas  racionales, indicando que no existirían costos de desinflación en Colombia. En datos anuales, la inflación responde ante cambios de la brecha del costo marginal real de manera parcial tanto en expectativas adaptativas como racionales.\\

Para \cite{galvis2010estimacion}, la verificación empírica de la NKPC es acorde para la economía colombiana en el periodo 1990-2006. El autor evidencia que los costos marginales de forma significativa explican la dinámica inflacionaria, dado que las empresas fijan el precio en promedio por cinco periodos para mantener cierto margen de ganancia sobre sus costos marginales (tabla \ref{t2}). En este sentido, se espera que las empresas mantengan sus precios en el tiempo en cuanto exista claridad entre reglas de política y la volatilidad de la inflación sea menor entre periodos. Adicionalmente observa, que cuando aumenta la inflación, las empresas comienzan a cambiar el precio con mayor frecuencia. Por último, \cite{galvis2010estimacion} abre la discusión sobre si la fijación de precios por parte de las empresas por varios periodos está estrechamente vinculado a la credibilidad del modelo o si es explicado por la demanda agregada de la economía en Colombia, la cual ha sido golpeada en los últimos años.\\

En la reciente monografía, \cite{hernandez2020evidencia} evidencian rigideces nominales vía precios, aunque con una pérdida de fuerza del modelo en Colombia, es decir, en comparación con los últimos dos trabajos mencionados para Colombia, existe una menor presencia en el grado de rigidez en los precios,  lo cual indica que un menor porcentaje de empresas mantienen los precios fijos ($\theta$) y por otra parte incrementan la velocidad en que las empresas cambian de precios (tabla \ref{t2}). Los autores enfatizan que este modelo ha evidenciado la disminución del impacto que tiene sobre las variables reales de la economía.

\begin{table}[H]
  \centering
  \caption{ Resultados estimaciones de la NKPC nivel nacional}
    \begin{tabular}{ c  c c c c c }
  \hline
        País  & $\beta$ & $\lambda$ & $\theta$ & $\frac{1}{1-\theta}$ & Fecha \\
         \hline
           \hline
    Colombia \dag & 0.87  & 0.171 & 0.696 & 3.3   & 1984:Q1-2003:Q4 \\
    Colombia \dag \dag & 0.832 & 0.0784 & 0.807 & 5.2   & 1990:Q1-2006:Q4 \\
    Colombia \dag \dag \dag & 0.912 & 0.201 & 0.662 & 3   & 2000:Q1-2019:Q4 \\
      \hline
    \end{tabular}%
  \label{t2}\\
  \raggedright  \scriptsize \textbf{Nota:} \cite{bejarano2005estimacion}\dag, \cite{galvis2010estimacion}\dag\dag, \cite{hernandez2020evidencia}\dag\dag\dag.   
\end{table}%

\chapter{Metodología y análisis de información}\label{cap2}
En este capítulo se describen los elementos que permiten el desarrollo de la estimación del modelo NKPC para los departamentos de Colombia. En la primera parte se establece la metodología de análisis, en seguida se realiza la descripción de las variables observadas que serán ajustadas al modelo de referencia, finalmente se lleva a cabo un análisis previo de la información para continuar con el capitulo final.
\section{Metodología de análisis}
En el proceso de la metodología de análisis se utiliza la información y los datos disponibles consistentes con la teoría prevista. Por lo tanto, este apartado comprende la identificación de los costos marginales  reales que dará lugar a la estimación econométrica de la NKPC (ecuación \eqref{21}). 
\subsection{Identificación de los costos marginales reales}
Inicialmente,  \cite{gali1999inflation} consideran que los costos marginales reales parten de una Cobb-Douglas,\footnote{La función de producción Cobb-Douglas \citep{cobb1928theory,douglas1948there}  es un enfoque que estima la función de producción de un país, involucrando las variaciones de los insumos capital (K), trabajo (N), y en adición la tecnología, para luego ser llamada como la productividad total de los factores (PTF).}  tomando la siguiente forma: 

\begin{equation}\label{23}
MCR_{t}=\frac{W}{P_{t}\delta Y_{t}/\delta N_{t}} 
\end{equation}
Donde  $Y_{t}=K_{t}^{\alpha} N_{t}^{1-\alpha}$, se compone del stock de capital, $K_{t}$, y  el número de ocupados, $N_{t}$. A partir del cociente de la ecuación \eqref{23},  se iguala $\frac{\delta Y_{t}}{\delta N_{t}}=(1-\alpha)Y_{t}/N_{t}$, para obtener  el costo laboral unitario según la división entre los ingresos laborales y  el PIB nominal ($S_{t}=\frac{W_{t}N_{t}}{P_{t}Y_{t}}$). En este sentido, se tiene que el costo  laboral unitario y la elasticidad producto del trabajo, van a ser la aproximación de los costos marginales. Estos se presentan así:
\begin{equation}\label{24}
MCR_{t}=\frac{S_{t}}{1-\alpha}
\end{equation}
Finalmente, se toma en logaritmos la brecha de los costos marginales con respecto de su estado estacionario, $mcr$,\footnote{\cite{galvis2010estimacion} denomina a $mcr_{t}$ como la brecha de los costos marginales.}  para luego ser empleada en la ecuación \eqref{19}. Como resultado, las estimaciones de la NKPC en \cite{gali1999inflation} arrojan la siguiente ecuación:\footnote{La interpretación de los coeficientes se encuentran en la sección \ref{s141}.} 
\begin{equation}\label{25}
\pi_{t}= \underbrace{0.926 E_{t}(\pi_{t+1}}_{(0.024)})+\underbrace{0.047 mcr_{t}}_{(0.008)}
\end{equation}
Los errores estándar se muestran entre paréntesis. Para el caso de la estimación de la NKPC en los departamentos de Colombia, los costos marginales reales serán aproximados a partir de los ingresos laborales obtenidos de la Gran Encuesta Integrada de Hogares (GEIH). Esta \textit{proxy} de los costos marginales reales estará acompañada de la tasa de crecimiento del Índice de Precios al Consumidor (IPC) como estimador de la inflación.

 \subsection{Estrategia econométrica}  \label{secgmm}
Uno de los métodos de estimación más populares en la econometría aplicada es el Método Generalizado de Momentos (GMM, \textit{sigla en inglés}). GMM generaliza el método clásico de estimador de momentos al permitir modelos que tienen más ecuaciones que parámetros desconocidos y, por lo tanto, están sobreidentificados. GMM incluye como casos especiales  mínimos cuadrados ordinarios (OLS, \textit{sigla en inglés}), variables instrumentales, regresión multivariada y mínimos cuadrados de dos etapas (2SLS, \textit{sigla en inglés}). Para el caso de su aplicación en la NKPC, bajo expectativas racionales, la ecuación \eqref{19} va a definir el conjunto de ortogonalidad de la forma reducida de la línea de base:%el error en el pronóstico de $\pi_{t+1}$ no está correlacionado con la información fechada $t$ anterior,:
\begin{equation}\label{22}
E_{t}((\pi_{t}-\beta  \pi_{t+1}-\lambda(mcr_{t}))\zeta_{t})=0
\end{equation}

Para la versión híbrida de la NKPC se parte de la ecuación \eqref{21} para tener que: 
 
\begin{equation}\label{22b}
E_{t}((\pi_{t}-\gamma_{f}\beta  \pi_{t+1}-\gamma_{b}\pi_{t-1}-\lambda(mcr_{t}))\zeta_{t})=0
\end{equation}

En este caso, $\zeta$ es el vector de variables datadas en $t$. Dada esta condición de ortogonalidad se puede estimar el modelo utilizando el  GMM propuesto por \cite{hansen1982generalized}.  En adición, se utilizan instrumentos con fecha $t-1$ (o anteriores) para contener el posible error al obtener los costos marginales. Suponiendo que este error no esté correlacionado con información pasada, es apropiado usar instrumentos rezagados \citep{gali2001european}.  La recomendación en la estimación de los parámetros de la ecuación \eqref{22b} por GMM, se debe precisamente a que los parámetros no son lineales y el número de instrumentos utilizado para esta estimación son mayores que el número de parámetros por estimar. Además, a diferencia de otros estimadores del GMM, no hay necesidad de suponer la distribución de probabilidad de los datos.\\%, $t-1$ y en la inflación en $t+1$

Otra razón para optar por esta estimación subyace en el uso de variables instrumentales. En la práctica resulta altamente plausible que las variables utilizadas que sirven de instrumento estén correlacionadas con $\zeta_{t}$, lo cual no deja de ser un problema en la magnitud del sesgo que pueda generar. Para el caso de \cite{galvis2010estimacion}, utiliza como variables instrumentales $\pi_{t-1}$ y $mcr_{t}$ en tres rezagos, debido a su estacionalidad en orden tres y la composición de los datos, que en este caso son trimestrales.\\

La estimación de la NKPC por GMM favorece la inferencia estadística debido a que no hay necesidad de tener normalidad en los errores de la  estimación, esto se debe fundamentalmente a que las propiedades del estimador de GMM cuentan con normalidad en los errores de la estimación. Sin embargo, a medida que la popularidad y el uso de la curva han crecido, se han suscitado críticas desde su identificación empírica. El problema principal es que los métodos de las variables instrumentales  como el GMM, no son inmunes a la presencia de instrumentos débiles \citep{dufour2006inflation}.\\

Así mismo, también se expresa que el GMM tiene los siguientes problemas: a) procedimientos asintóticos estándar defectuosos y direccionados a rechazos falsos, incluso no solo a muestras pequeñas (se le acuña una de las mayores críticas), sino también a grandes muestras, b) pruebas de tipo t con niveles de significancia que limitan la distribución del estadístico de prueba, c) intervalos de tipo Wald limitados por la construcción, a partir de la forma en que se estima el error estándar asintótico y el punto crítico asintótico \citep{dufour1997some}.\\

No obstante, han surgido múltiples estimadores que han sobresalido para compensar las limitaciones del GMM, entre ellos se destaca la eficiencia que tienen los métodos de estimación por máxima verosimilitud con plena información (FIML). Estos métodos cuentan con propiedades superiores en muestras pequeñas \citep{lendvai2005hungarian} y  funciones adaptables bajo el modelo de especificación errónea y errores de medición no distribuidos normalmente. Esta última condición, está basada en las simulaciones de Monte Carlo, donde \cite{linde2005estimating} concluye que no se pueden obtener estimaciones confiables del NKPC por métodos de ecuación única, lo cual favorece la implementación de FIML.\\

Sin embargo, estos puntos críticos que ha enfrentado GMM fueron revisados por  \cite{gali2005robustness} y argumentaron que las principales conclusiones del trabajo empírico base de la NKPC, 	\cite{gali1999inflation}, permanecen intactas bajo métodos alternativos de estimación. Estos autores concluyen que sus estimaciones son sólidas bajo una variedad de procedimientos econométricos diferentes al escenario establecido por \cite{rudd2005new}, sugiriendo además la inclusión en la estimación GMM de una economía cerrada junto con variables instrumentales no lineales en el espíritu de \cite{linde2005estimating}.\\

En resumen, la estimación de la NKPC por GMM tiene aún gran acogida por la comunidad académica, lo cual es evidenciado recientemente por   \cite{fidrmuc2020meta}, quienes se encargaron de realizar un estudio bibliométrico vinculado a la NKPC. Concluyen que las estimaciones por GMM son usadas con mayor frecuencia en la literatura, lo que significa un mayor apoyo empírico para la crítica generalizada del método GMM. %en la busqueda por encontrar el método ideal para estudiar la dinámica inflacionaria

\section{Datos y análisis descriptivo}
En esta sección se aborda las fuentes de información que permiten estimar la NKPC en los departamentos de Colombia (2010-2019). Para el caso de la construcción de los ingresos laborales se recurre a recabar información en las fuentes de tipo secundaria. De igual manera ocurre con la tasa de crecimiento del IPC y el PIB nominal. Adicionalmente, se realiza la descripción estadística de cada variable.
\subsection{Fuentes de información}
Las variables necesarias para la estimación de la NKPC son: 	$\pi_{t}$ y $mcr_{t}$. En primer lugar,  $\pi_{t}$ se obtiene calculando la tasa de crecimiento del IPC en datos mensuales entre el período 2010 a 2019 de cada una de las 23 ciudades principales que son expresadas en sus respectivos departamentos.\footnote{La tasa de crecimiento del IPC es anualizada, por lo tanto, se tiene en cuenta los datos de 2009. Durante la estimación de la NKPC departamental, como el dato del IPC está por ciudades, se supone que tendrá el mismo comportamiento con el departamento. Para el caso particular de Bogotá (capital del país y del departamento de Cundinamarca), se asume un comportamiento independiente como capital del país y en conjunto con Cundinamarca }  Estos datos se recogen del Departamento Administrativo Nacional de Estadística (DANE).\\

En el caso de $mcr_{t}$, inicialmente se construyen los costos laborales unitarios ($S_{t}$) a partir de los ingresos laborales de los trabajadores ($ W_{t}N_{t}$) y el PIB nominal ($P_{t}Y_{t}$). Los ingresos laborales de los trabajadores en cada departamento se recolectan de la GEIH depositada en el DANE en datos mensuales para 2010-2019.\footnote{Para $ W_{t}N_{t}$, los datos se consideran en meses corridos, en este sentido, de la misma manera que el calculo de $\pi_{t}$, se utilizan los datos del 2009.} Por otra parte, el $P_{t}Y_{t}$ tiene la particularidad que para los datos departamentales  están disponibles únicamente con una frecuencia anual en el DANE. Por esta razón, los datos anuales se desagregan mensualmente usando la metodología de \cite{chow1971best}.\footnote{ Esta metodología fue utilizada de igual manera por \cite{romero2008transmision} en su estudio de Transmisión regional  de la política monetaria en Colombia.} \\

Para tener un marco de referencia de los estudios en Colombia \citep{galvis2010estimacion,bejarano2005estimacion,hernandez2020evidencia}, la elasticidad del producto de la economía al factor trabajo ($1-\alpha$) será del 60\%.\footnote{En \cite{urrutia2002crecimiento} y \cite{tribin2006tasa}  se estima dicha elasticidad y se encuentra que en promedio está entre 56\% y 60\%.} Finalmente, se extrae el componente tendencial a esta variable ($MCR_{t}$) utilizando el filtro Hodrick-Prescott (HP) con un valor lambda estándar de 100 como una desviación de los costos marginales de su valor de tendencia ($mcr_{t}$).

% Table generated by Excel2LaTeX from sheet 'Est.descrip'
\begin{table}[H]
  \centering
  \caption{ Definición de las variables}
   \resizebox{15cm}{!} {
  \begin{tabular}{ p{4cm} p{1.2cm} p{4cm} p{5cm}}
  \hline
    Variable & Simbolo & Unidad & Descripción \\
    \hline
    \hline
    Inflación  &    $	\pi_{t}$   & Porcentaje & $\pi_{t}=\frac{p_{t}-p_{t-12}}{p_{t-12}}*100$  \\
    Costos marginales &  $MCR_{t}$     & Billones de pesos & $MCR_{t}=\frac{W_{t}N_{t}/P_{t}Y_{t}}{1-\alpha}$   \\
    Brecha de los costos marginales &   $mcr_{t}$    & Billones de pesos & Costos marginales  utilizando el método de filtro HP \\ 
    \hline
    \end{tabular}%
    }
  \label{tab:addlabel}\\
  \raggedright  \scriptsize \textbf{Fuente:} elaboración propia.
\end{table}%


\subsection{Estadísticas descriptivas}
En general, la tasa de inflación en Colombia ha disminuido desde que se implementó el régimen de inflación objetivo. La tabla \ref{res3} muestra que el período comprendido entre  2010 a 2019 la inflación en promedio mantiene una senda de crecimiento entre el  3\% y 4\% a pesar que en el 2013 se alcanzó una inflación en promedio menor al 2\% y superior al 8 \% para 2016. Lo último se debió principalmente a la depreciación del peso, generando incremento en el precio de los productos importados. Adicionalmente, como consecuencia de aspectos climáticos relacionados con el fenómeno de El Niño, aumentó el precio de los alimentos.\\

Por otra parte, se destaca  Bogotá, Caquetá, Norte de Santander y Risaralda por sus costos marginales promedio mayores que al resto de los lugares. En caso opuesto según el promedio, los departamentos con menores costos marginales son Meta, Cundinamarca, Cesar y Boyacá. Por períodos, los costos marginales más grandes se encuentran en Norte de Santander, principalmente entre 2010 y 2014, seguido por Risaralda y Caquetá. Por otra parte, Meta conservaría las cifras más inferiores en el período 2010-2014, donde más adelante es reemplazado por departamentos como Cundinamarca y Cesar.



\begin{table}[H]
  \centering
  \caption{Resumen de algunas estadísticas descriptivas de la inflación y los costos marginales (2010-2019)}
    \begin{tabular}{c c c c c c c c c }
     \hline
    \multirow{2}{*}{Departamentos}& \multicolumn{2}{c}{Media} & \multicolumn{2}{c}{st.dev.}& \multicolumn{2}{c}{Min} & \multicolumn{2}{c}{Max}  \\
     &      $\pi_{t}$ & $MCR_{t}$ & $\pi_{t}$ & $MCR_{t}$ & $\pi_{t}$ & $MCR_{t}$ & $\pi_{t}$ & $MCR_{t}$ \\
      \hline
       \hline
    Antioquia  & 0.039 & 1.162 & 0.016 & 0.059 & 0.018 & 1.037 & 0.087 & 1.265 \\
    Atlántico  & 0.039 & 1.071 & 0.017 & 0.141 & 0.014 & 0.769 & 0.085 & 1.340 \\ Bogotá & 0.037 & 1.440 & 0.016 & 0.095 & 0.016 & 1.249 & 0.090 & 1.571 \\
    Bolívar  & 0.037 & 0.721 & 0.017 & 0.092 & 0.011 & 0.573 & 0.082 & 0.926 \\
    Boyacá  & 0.034 & 0.549 & 0.016 & 0.032 & 0.009 & 0.482 & 0.088 & 0.605 \\
    Caldas  & 0.038 & 1.235 & 0.021 & 0.035 & 0.009 & 1.176 & 0.093 & 1.302 \\
    Caquetá  & 0.032 & 1.464 & 0.020 & 0.110 & 0.009 & 1.305 & 0.097 & 1.707 \\
    Cauca  & 0.034 & 0.635 & 0.018 & 0.052 & 0.007 & 0.563 & 0.087 & 0.765 \\
    Cesar  & 0.036 & 0.516 & 0.017 & 0.049 & 0.010 & 0.417 & 0.087 & 0.598 \\
    Córdoba  & 0.035 & 1.061 & 0.017 & 0.041 & 0.011 & 0.970 & 0.087 & 1.177 \\
    Chocó  & 0.029 & 0.901 & 0.018 & 0.161 & 0.004 & 0.573 & 0.081 & 1.335 \\
    Cundinamarca  & - & 0.514 & - & 0.034 & - & 0.459 & - & 0.583 \\
    Huila  & 0.035 & 0.665 & 0.018 & 0.036 & 0.012 & 0.603 & 0.092 & 0.751 \\
    La Guajira & 0.034 & 0.736 & 0.018 & 0.075 & 0.010 & 0.619 & 0.090 & 0.901 \\
    Magdalena  & 0.033 & 0.730 & 0.015 & 0.095 & 0.014 & 0.534 & 0.080 & 0.840 \\
    Meta  & 0.033 & 0.481 & 0.017 & 0.118 & 0.008 & 0.323 & 0.093 & 0.678 \\
    Nariño  & 0.033 & 1.373 & 0.021 & 0.079 & 0.007 & 1.230 & 0.098 & 1.562 \\
    N. Santander  & 0.033 & 1.501 & 0.021 & 0.231 & 0.000 & 1.136 & 0.106 & 1.934 \\
    Quindío  & 0.033 & 1.149 & 0.019 & 0.109 & 0.007 & 0.934 & 0.086 & 1.335 \\
    Risaralda  & 0.035 & 1.472 & 0.017 & 0.045 & 0.010 & 1.384 & 0.078 & 1.618 \\
    Santander  & 0.040 & 0.891 & 0.014 & 0.107 & 0.018 & 0.730 & 0.084 & 1.068 \\
    Sucre  & 0.034 & 0.964 & 0.021 & 0.037 & 0.007 & 0.906 & 0.094 & 1.047 \\
    Tolima  & 0.035 & 1.056 & 0.017 & 0.045 & 0.011 & 0.975 & 0.092 & 1.130 \\
    V. Cauca  & 0.035 & 0.988 & 0.018 & 0.030 & 0.012 & 0.949 & 0.100 & 1.096 \\
    Todas las regiones & 0.037 & 1.055 & 0.016 & 0.043 & 0.018 & 0.974 & 0.090 & 1.150 \\
     \hline
    \end{tabular}%
  \label{res3}\\
  \raggedright  \scriptsize \textbf{Fuente:} DANE. Estimaciones propias.
\end{table}%
 En comparación con la inflación, los costos marginales presentan una mayor volatilidad, principalmente para departamentos como Norte de Santander, Chocó y Atlántico. Estas diferencias regionales pueden observarse en la figura \ref{cm22}. Por lo general, la tasa de inflación regional con el tiempo presenta comportamientos similares, pero al observar el comportamiento según su ubicación, se puede encontrar ciertas diferencias  con la tasa de inflación nacional.\footnote{Para mayor información de la dinámica en  la inflación vista de manera espacial estimando la NKPC, véase \cite{yesilyurt2014regional,vaona2012regional}.} La figura \ref{infla21} presenta el comportamiento de la inflación regional en Colombia en cuatro momentos, en función de cuatro cuantiles. Para el año 2013 y 2016 se evidencia una ruptura estructural en cuanto al cambio radical del comportamiento de la inflación, lo cual  podría  significar cierta divergencia por regiones. Esta ruptura estructural se evidencia al comparar la dispersión de la inflación en diferentes períodos, como se observa en la tabla \ref{dispt} y en la figura \ref{dispf}. 


\begin{figure}[H]
\caption{Costos marginales en Colombia}
\begin{subfigure}{0.22\textwidth}
  \centering
  % include first image
	\includegraphics[width=2\textwidth]{Figuras/a2010a} 
  \caption{2010}
  \label{A41}
\end{subfigure}
\begin{subfigure}{0.22\textwidth}
  \centering
  % include second image
	\includegraphics[width=2\textwidth]{Figuras/b2013b} 
  \caption{2013}
  \label{A42}
\end{subfigure}
\begin{subfigure}{0.22\textwidth}
  \centering
  % include first image
	\includegraphics[width=2\textwidth]{Figuras/c2016c} 
  \caption{2016}
  \label{A43}
\end{subfigure}
\begin{subfigure}{0.22\textwidth}
  \centering
  % include second image
	\includegraphics[width=2\textwidth]{Figuras/d2019d} 
  \caption{2019}
  \label{A44}
\end{subfigure}\\
  \raggedright  \scriptsize \textbf{Fuente:} DANE. Estimaciones propias.\\
\raggedright  \scriptsize \textbf{Nota:} La representación gráfica de los costos marginales ($MCR_{t}$) se expresa en promedio anual por cuantiles. El color oscuro presenta menores costos marginales y el más claro mayor. Color gris no presenta información.
\label{cm22}	
\end{figure}


\begin{figure}[H]
\caption{Inflación en Colombia}
\begin{subfigure}{0.22\textwidth}
  \centering
  % include first image
	\includegraphics[width=2\textwidth]{Figuras/in2010a} 
  \caption{2010}
  \label{A41}
\end{subfigure}
\begin{subfigure}{0.22\textwidth}
  \centering
  % include second image
	\includegraphics[width=2\textwidth]{Figuras/in2013b} 
  \caption{2013}
  \label{A42}
\end{subfigure}
\begin{subfigure}{0.22\textwidth}
  \centering
  % include first image
	\includegraphics[width=2\textwidth]{Figuras/in2016c} 
  \caption{2016}
  \label{A43}
\end{subfigure}
\begin{subfigure}{0.22\textwidth}
  \centering
  % include second image
	\includegraphics[width=2\textwidth]{Figuras/in2019d} 
  \caption{2019}
  \label{A44}
\end{subfigure}\\
  \raggedright  \scriptsize \textbf{Fuente:} DANE. Estimaciones propias.\\
\raggedright  \scriptsize \textbf{Nota:} De la misma manera, la tasa de inflación se expresa en cuantiles. El color oscuro presenta menor inflación y el más claro mayor. Color gris no presenta información.\\
Quibdó (Capital de Chocó) no presenta datos de inflación para el 2019. Para el caso de Cundinamarca, se asume que la inflación de Bogotá es similar a la del departamento ya que no se cuenta con datos propios. Esto se mantendrá durante cada análisis descriptivo (figuras \ref{icm23} y \ref{cor24}) y en las estimaciones econométricas.
\label{infla21}	
\end{figure}

\begin{table}[H]
\centering
\caption{Dispersión regional (puntos porcentuales)}
\begin{tabular}{ccc}
\hline
& Inflación   & Brecha de los costos marginales \\
\hline
\hline
2010-14 & 1.01  & 5.19  \\
2015-19 &  2.05  & 5.24 \\
\hline
\end{tabular}%
\label{dispt} \\
  \raggedright  \scriptsize \textbf{Fuente:} DANE. Estimaciones propias.
\end{table}%

\begin{figure}[H]
\caption{Dispersión de la inflación y la brecha de los costos marginales}
\begin{subfigure}{0.48\textwidth}
  \centering
  % include first image
	\includegraphics[width=1\textwidth]{Figuras/infsd} 
  \caption{Inflación}
  \label{dispfa}
\end{subfigure}
\begin{subfigure}{0.48\textwidth}
  \centering
  % include second image
	\includegraphics[width=1\textwidth]{Figuras/cmsd} 
  \caption{Brecha de los costos marginales}
  \label{}
\end{subfigure}
	\label{dispf}\\
  \raggedright  \scriptsize \textbf{Fuente:} DANE. Estimaciones propias.
\end{figure}	



\begin{figure}%[H]
  	\centering 		
  	\caption{Tasa de inflación y costos marginales (HP) por departamento (2010-2019)}
	\includegraphics[width=1\textwidth]{Figuras/infcm3}
	\raggedright  \scriptsize % \textbf{Nota:} 
		\label{icm23}\\
  \raggedright  \scriptsize \textbf{Fuente:} DANE. Estimaciones propias.
	\end{figure}

En concordancia con la descripción de las dos variables principales del marco de estimación, la tasa de inflación y los costos marginales (HP), la figura \ref{icm23} muestra en detalle su comportamiento desde el 2010 hasta el 2019. Para la mayoría de las regiones, dos casos de mayor presión inflacionaria son prominentes. Estos ocurren primero en el 2011 por problemas originados por el invierno en Colombia, en donde los  alimentos (5,27\%), educación (4,57\%)  y vivienda (3,78\%) fueron los sectores que presionaron la inflación según el DANE. Por ciudades, las capitales de los departamentos de Santander y Huila presentaron mayores variaciones en el año, caso contrario sucedió con Nariño (2,41\%).\\
 	
El segundo repunte de la inflación sucedió en el 2015, coincidiendo con la fuerte presión de la  devaluación del dólar  debido al aumento en el precio del petróleo y en parte por fenómenos climáticos. La inflación en Colombia alcanzó el 6,77\%,  donde  los alimentos tuvieron una variación en precios del 10,85\%, todo esto ocurre  lejos de la meta del Banco de la República que actualmente se encuentra en 3\%. En seguida, el segundo semestre del 2016 presentó una leve recuperación en este indicador, manteniéndose en 5,75\%. La tasa de inflación fue más alta en la capital del departamento de Norte de Santander para mediados del año 2016 con 10,6\%,  junto con una volatilidad pronunciada medida por la desviación estándar (tabla \ref{res3}).\\

En cuanto a la brecha de los costos marginales, estos difieren más entre departamentos que por las tasas de inflación, como se ilustra en la figura \ref{icm23}. Los departamentos con mayor participación en el sector primario presentan mayores fluctuaciones al momento de observar el comportamiento de la brecha de los costos marginales, es decir, la participación laboral es susceptible a los fenómenos climáticos  en departamentos como Chocó, Nariño, Quindio, Caquetá y Magdalena.\footnote{La estructura económica de cada departamento se puede apreciar en la tabla \ref{variab}.} En el caso de los departamentos con mayor participación del sector secundario, principalmente por la actividad que abarca la minería y suministro de energía, coincide la época del aumento en el precio del petróleo con la creciente participación laboral como \textit{proxy} de los costos marginales.\\

No obstante, la consolidación del esquema de inflación objetivo a inicios de la década del 2000 en Colombia trajo consigo la conducción de la política monetaria y la incidencia que pueda generar en la actividad económica real por medio de la tasa de interés de intervención del Banco de la República, principal instrumento de política monetaria. El canal de tasa de interés genera efectos regionales de la política monetaria, toda vez  que los sectores de la economía no reaccionan en la misma proporción en reacción a los cambios en la tasa de interés. \\

Las características de los mercados en los cuales operan las empresas son determinantes importantes de las políticas de fijación de precios \citep{misas2009formacion}. La industria manufacturera o la construcción pueden ser más sensibles que otros sectores como la minería o el sector agropecuario debido a que el movimiento de la tasa de interés repercute en los costos del uso de capital, teniendo efectos en las inversiones y posteriormente en la demanda agregada. En la mayoría de los departamentos, estos sectores se ubican en regiones específicas. \\

Por otra parte, la vinculación económica entre departamentos puede responder de manera desigual frente a choques agregados. \cite{carlino1998differential} evidencian que el efecto en el cambio de los precios del petróleo afecta de manera diferente a las regiones productoras y consumidoras. Igualmente, atribuyen que los choques de la política monetaria -caso de la tasa de interés- difiere según las elasticidades de los sectores productivos y la distribución de las industrias en las regiones.\\
   
El comportamiento de la tasa de interés ha generado un impacto diverso en cada estructura económica en los departamentos vistos desde la incidencia de los costos marginales. En el período 2010 a 2012 cuando se presentó un aumento en la tasa de interés de 2.25 puntos porcentuales,  departamentos como Chocó y Norte de Santander tuvieron un crecimiento pronunciado de los costos marginales por debajo de la tendencia, caso opuesto se observó en Caquetá, siendo el más destacado. Sin embargo, \cite{quintero2019impactos}  propuso una agrupación departamental que vinculara el tipo de actividad principal de cada departamento frente al impacto de política monetaria, coincidencialmente los departamentos mencionados pertenecen a diversos, es decir, su actividad económica principal no se encuentra en la minería, servicios e industriales.\\

El retorno de la tasa de interés en el año 2014, permitió observar a los departamentos en dicha agrupación de la actividad productiva, reaccionar en oposición al período de descenso de la tasa de interés, donde los costos marginales tuvieron un destacado comportamiento en Chocó y Norte de Santander, contrario a Caquetá. Otros departamentos pertenecientes a actividades económicas  diversas como Quindio y Magdalena, presentaron comportamientos acordes a las decisiones de política monetaria, es decir, un aumento -disminución- de la tasa de interés contribuyó en un efecto de menor -mayor- demanda agregada. Para el caso de las actividades económicas industriales, mineras y de servicios no se observa patrones claros que evidencien alguna relación entre la tasa de interés y el comportamiento de la brecha de los costos marginales.


\section{Análisis de la información}
Otro tema importante que debe considerarse en el análisis de la dinámica de la inflación es el efecto temporal de las variables. Mientras que para la curva de Phillips tradicional la inflación debería depender negativamente de la brecha del producto rezagada, caso contrario ocurriría con la NKPC (ecuación \eqref{20}). \cite{fuhrer1995inflation} señala que la NKPC implica que la inflación debe tener un comportamiento procíclico con la brecha del producto o la brecha de los costos marginales para este caso, es decir,  un aumento (disminución) de la inflación actual debe indicar un aumento (disminución) posterior en la brecha de los costos marginales. Esto termina siendo un problema cuando se confronta con los datos debido a que pueden evidenciar un patrón opuesto.\\

Por lo anterior, \cite{gali1999inflation}  observan que  la brecha del producto actual se mueve positivamente con la inflación futura y negativamente con la inflación rezagada. Esto señala que la brecha del producto sobre la inflación explica por qué la brecha del producto rezagada entra con un coeficiente positivo, consistente con la teoría de la curva de Phillips tradicional, pero contraria a la NKPC.\footnote{La forma que tomaría la curva de Phillips tradicional seria: $\pi_{t}=\pi_{t-1}+k(y_{t-1})$ (ecuación 9 en \cite{gali1999inflation}).} Por esta razón, la lectura que se hará a las correlaciones cruzadas será opuesta a las mencionadas para satisfacer la NKPC (ecuación \eqref{21}).\\

Adicionalmente, \cite{gali1999inflation} evidencian que la participación de los ingresos laborales (la medida considerada para determinar la brecha de los costos marginales) superan la brecha de producción en la estimación de la NKPC, toda vez que conduce a la inflación, y no al revés, en contradicción directa con la teoría. Por el contrario, la brecha de los costos marginales muestra una fuerte correlación  contemporánea con la inflación e incluso la inflación rezagada se correlaciona positivamente con la participación del ingreso laboral, de acuerdo con la teoría. En este sentido, no sorprendería que entrara a la ecuación de inflación estructural significativa y con el signo correcto. \\

Por otra parte, el comportamiento lento del costo marginal podría ayudar a explicar la lenta respuesta de la inflación al producto y, por lo tanto, por qué las desinflaciones pueden implicar costosas reducciones del producto. Por ejemplo, \cite{blanchard1993competitiveness} encontraron que las desinflaciones en Francia se han asociado con disminuciones en los costos laborales unitarios reales.  Por esta razón, modificar las teorías existentes para dar cuenta de las rigideces en los costos marginales sugeridos y evidenciadas por  \cite{gali1999inflation}  podrían ofrecer información importante para la dinámica de la inflación. Dado el vínculo entre la participación laboral y los costos marginales, una fuente candidata para la fricción necesaria es la rigidez salarial.\\

En síntesis, las correlaciones cruzadas dinámicas de la brecha del costo marginal y las tasas de inflación en cada región presentadas en la figura \ref{cor24} ayudan a enmarcar el problema. La importancia de esta manera de observar la información, subyace en el sentido en que el comportamiento prospectivo, retrospectivo de la inflación y la brecha de los costos marginales, no solo dará lugar a una buena explicación de la modelación de la dinámica inflacionaria, sino que también   sugerirá el uso de variables instrumentales para el control de problemas de endogeneidad en las estimaciones obtenidas más adelante. \\

Los movimientos negativos de la brecha de los costos marginales sobre la inflación futura van a ser posibles en  departamentos  como Atlántico, Boyacá,  Caldas, Caquetá y Chocó. Estas correlaciones son superiores al 50\%, lo cual puede suponer la aceptación del modelo NKPC. En menor incidencia aparecen los departamentos de Bolivar, Huila, Meta, Norte de Santander y Risaralda. \\

Los principales epicentros de desarrollo del país como Bogotá, Antioquia y Valle del Cauca parecen no evidenciar con claridad las posibles dinámicas inflacionarias que pueda ofrecer el modelo en referencia. De la misma manera, departamentos que principalmente se encuentran ubicados en la zona costera del país van a tener movimientos positivos de la inflación prospectiva y la brecha de los costos marginales, sin dar espacio al posible cumplimiento de la NKPC.\\

\begin{figure}[H]
  	\centering 		
  	\caption{Correlación cruzada entre la tasa de inflación y costos marginales (HP) por departamento (2010-2019)}
	\includegraphics[width=0.95\textwidth]{Figuras/corre}
	\raggedright % \scriptsize \textbf{Nota:} 
		\label{cor24}\\
  \raggedright  \scriptsize \textbf{Fuente:} DANE. Estimaciones propias.	
	\end{figure}

Las diversas direcciones que toma la brecha de los costos marginales frente a la inflación, presume posibles efectos diferenciales de la política monetaria, no solo entre departamentos, sino entre regiones. Una alta variabilidad de los resultados entre departamentos, inclusive entre los pertenecientes a una misma región geográfica indican que la dinámica inflacionaria explicada por los costos marginales y la inflación prospectiva pueden tener incidencia en estructuras económicas no muy claras, opuesto a lo considerado por \cite{quintero2019impactos}, donde encuentra que los sectores más sensibles a la política monetaria son la industria manufacturera, construcción y transporte y comunicaciones.


\chapter{Resultados y discusión} \label{cap3}
Este capítulo aborda la estimación econométrica de la NKPC para los departamentos de Colombia y de manera nacional por medio de un panel dinámico. Los resultados encontrados serán fuente de comparación de variables que puedan dar explicación a la dinámica inflacionaria de cada región. Posteriormente, se discuten los hallazgos.

\section{NKPC por departamentos}
Las estimaciones de la forma reducida del NKPC de referencia para cada departamento, se deriva de la especificación en la ecuación \eqref{22b}, siguiendo las variables instrumentales para datos mensuales  que sugiere \cite{ramos2008inflation}.\footnote{La especificación del modelo considera $\pi_{t-3}$ y $\pi_{t+3}$, dado su nivel estacional.}  Consistente con la teoría NKPC, los coeficientes del término prospectivo ($\gamma_{f}$)  y retrospectivo  ($\gamma_{b}$)  son positivos y estadísticamente significativos al 1\% con GMM (tabla \ref{tab31}). En todos los casos, los coeficientes del término prospectivo, $\gamma_{f}$, es mayor que el coeficiente de inflación rezagada, $\gamma_{b}$. Este último tiene un tamaño del coeficiente entre  0.213-0.399. Para el caso de $\gamma_{f}$, el rango esta entre  0.682-1.002.\\

La estimación para todas las regiones frente a la inflación prospectiva fue de 0.7, inferior a los reportados en trabajos previos (tabla \ref{t2}). Por el lado del coeficiente de la brecha de los costos marginales, se obtiene que no es estadísticamente significativa con signo negativo, evidenciando  diferencias considerables entre procesos inflacionarios por departamento. Sin embargo, se observa que $\lambda$ es positivo y negativo con significancia estadística al 5\% y al 1\% para 42\% de los 24 departamentos (incluida la ciudad capital, Bogotá). La estimación significativa al 5\% aparece para el coeficiente negativo en el caso de Bolivar (-0.074), y al 1\% para Cundinamarca (-0.093) y Tolima (-0.111).  Al 1\%  con coeficiente positivo (consistente con la teoría), la brecha de los costos marginales varia entre 0.018 (Atlántico) y 0.158 (Caldas).

\begin{table}[H]
\centering
\caption{Estimación de forma reducida de la versión híbrida de la NKPC por departamento (2010-2019)}
\begin{tabular}{lllllll}
\hline
Departamento & Constante  & $\gamma_{b}$  & $\gamma_{f}$ & $\lambda$   & $\theta$ & J -Test \\
 &       &   $\pi_{t-3}$    &    $\pi_{t+3}$    &  $mcr_{t}$    &   $\frac{1}{1-\theta}$    & (p-value) \\
\hline
\hline
\vspace{-0.3cm} Antioquia & -0.002** & 0.370*** & 0.682*** & -0.023 &       & 8.156\\   
 & \scriptsize{(0.001)} & \scriptsize{(0.040)} & \scriptsize{(0.054)} & \scriptsize{(0.015)} &       & \scriptsize{(0.518)} \\
\vspace{-0.3cm} Atlántico & -0.003*** & 0.337*** & 0.737*** & 0.018*** & 0.944 & 5.071\\   
& \scriptsize{(0.001)} & \scriptsize{(0.043)} & \scriptsize{(0.051)} & \scriptsize{(0.005)} & \scriptsize{18}    & \scriptsize{(0.828)} \\
\vspace{-0.3cm} Bogotá & -0.001 & 0.294*** & 0.721*** & -0.006 &       & 7.32 \\  
& \scriptsize{(0.002)} & \scriptsize{(0.052)} & \scriptsize{(0.065)} & \scriptsize{(0.007)} &       & \scriptsize{(0.603)} \\
\vspace{-0.3cm} Bolivar & -0.008** &	0.213***  &	1.002*** &	-0.074** &  & 5.464\\   
& \scriptsize{(0.003)} & \scriptsize{(0.077)} & \scriptsize{(0.134)} & \scriptsize{(0.034)} &     & \scriptsize{(0.792)} \\ 
\vspace{-0.3cm} Boyacá & -0.007*** & 0.226*** & 0.975*** & 0.141*** & 0.695 & 4.349\\   
& \scriptsize{(0.002)} & \scriptsize{(0.059)} & \scriptsize{(0.091)} & \scriptsize{(0.026)} & \scriptsize{3}     & \scriptsize{(0.886)} \\
\vspace{-0.3cm} Caldas & -0.008*** & 0.321*** & 0.869*** & 0.158*** & 0.709 & 79.204\\   
& \scriptsize{(0.002)} & \scriptsize{(0.060)} & \scriptsize{(0.077)} & \scriptsize{(0.031)} & \scriptsize{3}     & \scriptsize{(0.542)} \\
\vspace{-0.3cm} Caquetá & -0.002*** & 0.257*** & 0.805*** & 0.029*** & 0.904 & 42.663\\   
& \scriptsize{(0.001)} & \scriptsize{(0.043)} & \scriptsize{(0.044)} & \scriptsize{(0.008)} & \scriptsize{10}    & \scriptsize{(0.893)} \\
\vspace{-0.3cm} Cauca & -0.003** & 0.354*** & 0.739*** & 0.019 & 0.941 & 5.998\\   
& \scriptsize{(0.002)} & \scriptsize{(0.040)} & \scriptsize{(0.052)} & \scriptsize{(0.034)} & \scriptsize{17}    & \scriptsize{(0.74)} \\
\vspace{-0.3cm} Cesar & -0.004* & 0.322*** & 0.789*** & -0.011 &       & 15.304\\   
& \scriptsize{(0.002)} & \scriptsize{(0.070)} & \scriptsize{(0.076)} & \scriptsize{(0.031)} &       & \scriptsize{(0.082)} \\
\vspace{-0.3cm} Córdoba & -0.003 & 0.339*** & 0.715*** & -0.005 &       & 9.606\\   
& \scriptsize{(0.002)} & \scriptsize{(0.059)} & \scriptsize{(0.061)} & \scriptsize{(0.020)} &       & \scriptsize{(0.383)} \\
\vspace{-0.3cm} Cundinamarca & -0.002 & 0.228*** & 0.813*** & -0.093*** &       & 4.311\\   
& \scriptsize{(0.001)} & \scriptsize{(0.037)} & \scriptsize{(0.060)} & \scriptsize{(0.018)} &       & \scriptsize{(0.889)} \\
\vspace{-0.3cm} Chocó & -0.001 & 0.399*** & 0.627*** & 0.001 &       & 83.287\\   
& \scriptsize{(0.002)} & \scriptsize{(0.056)} & \scriptsize{(0.054)} & \scriptsize{(0.008)} &       & \scriptsize{(0.501)} \\
\vspace{-0.3cm} Huila & -0.005*** & 0.360*** & 0.784*** & 0.052*** & 0.862 & 4.202\\   
& \scriptsize{(0.001)} & \scriptsize{(0.030)} & \scriptsize{(0.030)} & \scriptsize{(0.010)} &  \scriptsize{7}    & \scriptsize{(0.897)} \\
\vspace{-0.3cm} La Guajira & -0.007*** & 0.357*** & 0.840*** & -0.009 &       & 4.524\\   
& \scriptsize{(0.002)} & \scriptsize{(0.037)} & \scriptsize{(0.077)} & \scriptsize{(0.014)} &       & \scriptsize{(0.873)} \\
\vspace{-0.3cm} Magdalena & -0.005*** & 	0.310*** &	0.828*** &	-0.023 &  & 5.578\\   
& \scriptsize{(0.001)} & \scriptsize{(0.053)} & \scriptsize{(0.060)} & \scriptsize{(0.014)} &     & \scriptsize{(0.781)} \\
\vspace{-0.3cm} Meta & -0.005*** & 	0.384*** &	0.771*** &	-0.009 &       &  3.325 \\  
& \scriptsize{(0.002)} & \scriptsize{(0.252)} & \scriptsize{(0.048)} & \scriptsize{(0.011)} &       & \scriptsize{(0.950)} \\
\vspace{-0.3cm} Nariño & -0.004** &	0.261*** &	0.844*** &	-0.017 &       & 5.189 \\   
& \scriptsize{(0.002)} & \scriptsize{(0.053)} & \scriptsize{(0.067)} & \scriptsize{(0.014)} &  & \scriptsize{(0.817)} \\
\vspace{-0.3cm} Norte de Santander & -0.003** & 0.332*** & 0.759*** & 0.028*** & 0.916 & 3.959\\   
& \scriptsize{(0.001)} & \scriptsize{(0.045)} & \scriptsize{(0.053)} & \scriptsize{(0.007)} & \scriptsize{12}    & \scriptsize{(0.914)} \\
\vspace{-0.3cm} Quindío & -0.002 & 0.364*** & 0.691*** & -0.004 &       & 10.33 \\  
& \scriptsize{(0.002)} & \scriptsize{(0.056)} & \scriptsize{(0.083)} & \scriptsize{(0.007)} &       & \scriptsize{(0.324)} \\
\vspace{-0.3cm} Risaralda & -0.002*** & 0.297*** & 0.769*** & 0.034*** & 0.900 & 4.911\\   
& \scriptsize{(0.001)} & \scriptsize{(0.021)} & \scriptsize{(0.025)} & \scriptsize{(0.005)} & \scriptsize{10}    & \scriptsize{(0.842)} \\
\vspace{-0.3cm} Santander & -0.005* & 0.229*** & 0.873*** & 0.027 & 0.891 & 8.392\\   
& \scriptsize{(0.003)} & \scriptsize{(0.087)} & \scriptsize{(0.109)} & \scriptsize{(0.036)} & \scriptsize{9}     & \scriptsize{(0.495)} \\
\vspace{-0.3cm} Sucre & -0.004*** &	0.363*** &	0.751*** &	0.012 &  0.958  & 5.873\\   
& \scriptsize{(0.001)} & \scriptsize{(0.040)} & \scriptsize{(0.031)} & \scriptsize{(0.042)} &  \scriptsize{24}  & \scriptsize{(0.752)} \\
\vspace{-0.3cm} Tolima & -0.008*** &	0.225*** &	0.983*** &	-0.111*** &  & 5.812\\   
& \scriptsize{(0.002)} & \scriptsize{(0.032)} & \scriptsize{(0.058)} & \scriptsize{(0.016)} &    & \scriptsize{(0.868)} \\
\end{tabular}
\label{tab31}\\
\raggedright  \scriptsize \
\end{table}%


\begin{table}[H]
\centering
\begin{tabular}{lllllll}
\vspace{-0.3cm} Valle del Cauca& -0.004* & 0.226*** & 0.871*** & 0.033 & 0.877 & 6.493\\   
& \scriptsize{(0.002)} & \scriptsize{(0.064)} & \scriptsize{(0.091)} & \scriptsize{(0.039)} & \scriptsize{8}     & \scriptsize{(0.689)} \\
\vspace{-0.3cm} Todas las regiones & -0.002 & 0.346*** & 0.700*** & -0.017 &       & 7.539\\   
& \scriptsize{(0.001)} & \scriptsize{(0.035)} & \scriptsize{(0.050)} & \scriptsize{(0.014)} &       & \scriptsize{(0.581)} \\
\hline
\end{tabular}
\label{}\\
  \raggedright  \scriptsize \textbf{Fuente:} estimaciones propias.\\
\raggedright  \scriptsize \textbf{Nota:} ***,  **, * representan significancia estadistica al 1\%, 5\% y 10\%, respectivamente. Bartlett KerneL, Newey-West fijo, errores estándar robustos de HAC entre paréntesis.\\
Instrumentos: brecha de los costos marginales: t-2 a t-7, inflación: t-1 a t-6.\\
$\theta$ se encuentra al despejar en $\lambda$ a partir de la ecuación \eqref{18}. El coeficiente $\gamma_{f}$ es utilizado en la ecuación, en lugar de $\beta$ para calcular $\theta$. Las figuras \ref{dispf} y \ref{estim} presentan un panorama amplio del comportamiento de los diferentes coeficientes.
\end{table}%

Para Norte de Santander, donde la volatilidad de la inflación y los costos marginales para el período 2010-2019 fueron más altos (tabla \ref{res3}), la NKPC híbrida se ajusta bien a los datos. De igual manera,  se observa que Atlántico, Boyacá, Caldas, Caquetá, Huila y Risaralda son acordes al modelo de referencia con significancia estadística al 1\%, similar a la inflación en rezagos y vista en prospectiva. Por otra parte, el desajuste  del NKPC híbrido en Santander y Sucre  se debe a  que los coeficientes de la brecha de los costos marginales ($\lambda$) no es estadísticamente significativa, resultando grandes errores estándar alrededor de las estimaciones. Esto evidencia que el proceso inflacionario no es similar en todos los departamentos, sino por el contrario, existe heterogeneidad en la dinámica inflacionaria regional.\\

En concordancia, la fracción de empresas que cambian su precio usando en menor medida una regla empírica retrospectiva, el coeficiente $\gamma_{b}$ de forma reducida asociado con los rezagos de la inflación es menor y el coeficiente asociado con la inflación esperada ($\gamma_{f}$) es mayor. Esto significa que a medida que la fracción de empresas atrasadas es menor, la persistencia de la inflación disminuye, en este sentido, las empresas establecen los precios en su mayoría dependiendo del futuro. Además, esto implica que la empresa fije sus precios cada vez menos según una regla basada en el comportamiento pasado de los precios y que la relación entre la brecha de los costos marginales y la inflación se fortalezca.\\

Esta relación al presentar mayores coeficientes en la inflación esperada y la brecha de los costos marginales explican la baja probabilidad que las empresas mantengan sin cambios los precios en el tiempo, como en el caso de Boyacá y Caldas. Por el contrario, si existe mayor explicación de la dinámica inflacionaria en los departamentos por parte de la persistencia de la inflación ($\gamma_{b}$), de la misma manera las empresas mantendrán los precios sin cambios por un tiempo considerable. \\


\begin{table}%[H]
\centering
\caption{Estimación de forma reducida de la versión híbrida de la NKPC por departamento (Alternativa) (2010-2019)}
\begin{tabular}{lllllll}
\hline
\hline
Departamento & Constante  & $\gamma_{b}$  & $\gamma_{f}$ & $\lambda$   & $\theta$ & J -Test \\
 &       &   $\pi_{t-k}$    &    $\pi_{t+k}$    &  $mcr_{t}$    &   $\frac{1}{1-\theta}$    & (p-value) \\
\hline
\vspace{-0.3cm} Antioquia (c)  & -0.006** & 0.407*** & 0.756*** & -0.056* &       & 6.026\\   
& \scriptsize{(0.003)} & \scriptsize{(0.029)} & \scriptsize{(0.086)} & \scriptsize{(0.033)} &       & \scriptsize{(0.737)} \\
\vspace{-0.3cm} Atlántico (c) & -0.010*** & 0.395*** & 0.878*** & 0.049*** & 0.842 & 4.818\\   
& \scriptsize{(0.002)} & \scriptsize{(0.049)} & \scriptsize{(0.060)} & \scriptsize{(0.014)} & \scriptsize{6}    & \scriptsize{(0.849)} \\
\vspace{-0.3cm} Bogotá (c) & -0.005 & 0.368*** & 0.764*** & 0.001 &       & 6.863\\   
& \scriptsize{(0.003)} & \scriptsize{(0.064)} & \scriptsize{(0.106)} & \scriptsize{(0.019)} &       & \scriptsize{(0.651)} \\
\vspace{-0.3cm} Bolivar (d) & -0.006 & 0.810*** & 0.453*** & 0.370*** & 0.655 & 4.891\\   
& \scriptsize{(0.007)} & \scriptsize{(0.092)} & \scriptsize{(0.156)} & \scriptsize{(0.072)} & \scriptsize{3}     & \scriptsize{(0.843)} \\
\vspace{-0.3cm} Boyacá (c) & -0.020*** & 0.276*** & 1.257*** & 0.345*** &       & 4.032\\   
& \scriptsize{(0.005)} & \scriptsize{(0.092)} & \scriptsize{(0.149)} & \scriptsize{(0.070)} &       & \scriptsize{(0.909)} \\
\vspace{-0.3cm} Caldas (c) & -0.019*** & 0.484*** & 0.987*** & 0.199*** & 0.646 & 3.604\\   
& \scriptsize{(0.002)} & \scriptsize{(0.056)} & \scriptsize{(0.087)} & \scriptsize{(0.043)} & \scriptsize{3}     & \scriptsize{(0.935)} \\
\vspace{-0.3cm}Caquetá  (c) & -0.009*** & 0.356*** & 0.898*** & 0.040** & 0.854 & 3.993\\   
& \scriptsize{(0.003)} & \scriptsize{(0.044)} & \scriptsize{(0.089)} & \scriptsize{(0.016)} & \scriptsize{7}    & \scriptsize{(0.911)} \\
\vspace{-0.3cm}Cauca (c) & -0.009*** & 0.378*** & 0.877*** & -0.041 &       & 3.392\\   
& \scriptsize{(0.003)} & \scriptsize{(0.042)} & \scriptsize{(0.104)} & \scriptsize{(0.053)} &       & \scriptsize{(0.946)} \\
\vspace{-0.3cm}Cesar  (c) & -0.019*** & 0.535*** & 0.985*** & -0.022 &       & 7.239\\   
& \scriptsize{(0.007)} & \scriptsize{(0.113)} & \scriptsize{(0.134)} & \scriptsize{(0.062)} &       & \scriptsize{(0.612)} \\
\vspace{-0.3cm}Córdoba (c) & -0.008 & 0.447*** & 0.763*** & -0.035 &       & 9.036\\   
& \scriptsize{(0.005)} & \scriptsize{(0.079)} & \scriptsize{(0.105)} & \scriptsize{(0.023)} &       & \scriptsize{(0.433)} \\
\vspace{-0.3cm}Cundinamarca (c)   & -0.006*** & 0.288*** & 0.880*** & -0.200*** &       & 3.752\\   
& \scriptsize{(0.002)} & \scriptsize{(0.030)} & \scriptsize{(0.066)} & \scriptsize{(0.035)} &       & \scriptsize{(0.927)} \\
\vspace{-0.3cm}Chocó  (c) & -0.007 & 0.436*** & 0.788*** & -0.012 &       & 3.475\\   
& \scriptsize{(0.006)} & \scriptsize{(0.061)} & \scriptsize{(0.158)} & \scriptsize{(0.019)} &       & \scriptsize{(0.942)} \\
\vspace{-0.3cm}Huila  (c) & -0.023*** & 0.504*** & 1.112*** & 0.157*** &       & 4.635\\   
& \scriptsize{(0.005)} & \scriptsize{(0.058)} & \scriptsize{(0.084)} & \scriptsize{(0.033)} &       & \scriptsize{(0.864)} \\
\vspace{-0.3cm}La Guajira  (a) & 0.000 & 0.518*** & 0.479*** & 0.004** &       & 4.619\\   
& \scriptsize{(0.000)} & \scriptsize{(0.028)} & \scriptsize{(0.031)} & \scriptsize{(0.002)} &       & \scriptsize{(0.866)} \\
\vspace{-0.3cm}Magdalena (b)\dag & 0.000 & 0.570*** & 0.417*** & 0.016*** & 0.974 & 4.954\\   
& \scriptsize{(0.001)} & \scriptsize{(0.028)} & \scriptsize{(0.032)} & \scriptsize{(0.005)} & \scriptsize{38}    & \scriptsize{(0.894)} \\
\vspace{-0.3cm}Meta (d) & 0.116*** & -1.160*** & -1.022** & 0.335*** &       & 2.97 \\  
& \scriptsize{(0.025)} & \scriptsize{(0.267)} & \scriptsize{(0.484)} & \scriptsize{(0.044)} &       & \scriptsize{(0.965)} \\
\vspace{-0.3cm}Nariño (a)\dag & 0.000 & 0.612*** & 0.376*** & 0.007** &       & 6.625\\   
& \scriptsize{(0.000)} & \scriptsize{(0.048)} & \scriptsize{(0.050)} & \scriptsize{(0.003)} &       & \scriptsize{(0.76)} \\
\vspace{-0.3cm}Norte de Santander (c) & -0.013** & 0.451*** & 0.894*** & 0.074*** & 0.796 & 6.818\\   
& \scriptsize{(0.007)} & \scriptsize{(0.107)} & \scriptsize{(0.119)} & \scriptsize{(0.018)} & \scriptsize{5}     & \scriptsize{(0.656)} \\
\vspace{-0.3cm} Quindío  (c) & -0.006*** & 0.447*** & 0.736*** & -0.018* &       & 4.518\\   
& \scriptsize{(0.002)} & \scriptsize{(0.032)} & \scriptsize{(0.056)} & \scriptsize{(0.010)} &       & \scriptsize{(0.874)} \\
\vspace{-0.3cm}Risaralda  (c) & -0.009*** & 0.432*** & 0.835*** & 0.073*** & 0.814 & 5.305\\   
& \scriptsize{(0.003)} & \scriptsize{(0.061)} & \scriptsize{(0.099)} & \scriptsize{(0.017)} & \scriptsize{5}    & \scriptsize{(0.806)} \\
\vspace{-0.3cm}Santander (c) & -0.021*** & 0.477*** & 1.024*** & 0.027 &       & 4.769\\   
& \scriptsize{(0.007)} & \scriptsize{(0.087)} & \scriptsize{(0.121)} & \scriptsize{(0.059)} &       & \scriptsize{(0.854)} \\
\vspace{-0.3cm}Sucre (a)\dag & -0.002** & 0.670*** & 0.388*** & 0.070** & 0.903 & 11.671\\   
& \scriptsize{(0.001)} & \scriptsize{(0.062)} & \scriptsize{(0.042)} & \scriptsize{(0.032)} & \scriptsize{10}    & \scriptsize{(0.307)} \\
\vspace{-0.3cm} Tolima (b)\dag & 0.000 & 0.553*** & 0.449*** & 0.029* & 0.952 & 5.812\\   
& \scriptsize{(0.001)} & \scriptsize{(0.048)} & \scriptsize{(0.061)} & \scriptsize{(0.017)} & \scriptsize{21}    & \scriptsize{(0.83)} \\
\end{tabular}%
\label{tabane1}%
\end{table}%



\begin{table}%[H]
\centering
\begin{tabular}{lllllll}
\vspace{-0.3cm}Valle del Cauca  (c) & -0.009 & 0.320*** & 0.919*** & 0.137 & 0.715 & 6.881\\   
& \scriptsize{(0.007)} & \scriptsize{(0.078)} & \scriptsize{(0.206)} & \scriptsize{(0.095)} & \scriptsize{4}     & \scriptsize{(0.649)} \\
\vspace{-0.3cm}Todas las regiones (c) & -0.009** & 0.421*** & 0.817*** & -0.038 &       & 6.015\\   
& \scriptsize{(0.004)} & \scriptsize{(0.063)} & \scriptsize{(0.124)} & \scriptsize{(0.035)} &       & \scriptsize{(0.738)} \\
\hline
\end{tabular}
\label{tab:addlabel}\\
  \raggedright  \scriptsize \textbf{Fuente:} estimaciones propias.\\
\raggedright  \scriptsize \textbf{Nota:} ***,  **, * representan significancia estadistica al 1\%, 5\% y 10\%, respectivamente. Bartlett KerneL, Newey-West fijo, errores estándar robustos de HAC entre paréntesis.\\
Instrumentos: brecha de los costos marginales: t-2 a t-7, inflación: t-1 a t-6.\\
\dag Instrumentos alternativos: brecha de los costos marginales: t-8 a t-13, inflación: t-6 a t-12. \\
Estos instrumentos alternativos se enfatizan en la captura del nivel de significancia del coeficiente $\lambda$.
Los modelos se estiman de la siguiente manera: (a) $\pi_{t}=\pi_{t-1}+\pi_{t+1}+mcr_{t}$, (b) $\pi_{t}=\pi_{t-3}+\pi_{t+3}+mcr_{t}$, (c) $\pi_{t}=\pi_{t-6}+\pi_{t+6}+mcr_{t}$, (d) $\pi_{t}=\pi_{t-12}+\pi_{t+12}+mcr_{t}$.
\end{table}%


Departamentos con estructuras económicas que no pertenecen -según \cite{quintero2019impactos}- a  la industria, servicios y minería como Chocó, Córdoba, Huila, Magdalena, Norte de Santander, Quindío y Sucre, presentan mayor valor  del coeficiente de persistencia en la inflación, siendo quienes establecen los precios de manera retrospectiva, así como en la adopción de medidas en materia de política monetaria pueden tener un efecto lento en la actividad económica. No obstante, estimaciones alternativas aparecen en la tabla \ref{tabane1}, donde ocurren casos particulares para los departamentos de Tolima, Sucre, Nariño, Magdalena, La Guajira y Bolivar en diferentes rezagos e instrumentos.\\

Por lo anterior, la figura \ref{estim} presenta la ubicación geográfica de los coeficientes $\gamma_{b}$, $\gamma_{b}$, $\lambda$ y $\theta$ a partir de los resultados de la tabla \ref{tab31}, permitiendo observar en primer lugar la baja persistencia de la inflación en departamentos que comunican el pacífico y el caribe con el centro del país, en este sentido, el impacto que pueda tener un mecanismo de transmisión de la política monetaria como es el caso de la tasa de interés pueden tener mayor incidencia por la alta ponderación observada en la inflación futura. En contraste, el coeficiente de la brecha del costo marginal ($\lambda$) difiere su tamaño en diferentes regiones, sin embargo, al apreciar la baja probabilidad de que las empresas mantengan los precios sin cambios en el tiempo, los departamentos ubicados en la zona central del país -a excepción de Bolivar- van a ser quienes cuenten con esta característica.


\begin{figure}[H]
\caption{Resultados de estimaciones espaciales}
\begin{subfigure}{0.22\textwidth}
  \centering
  % include first image
	\includegraphics[width=2\textwidth]{Figuras/bac} 
  \caption{$\gamma_{b}$}
  \label{}
\end{subfigure}
\begin{subfigure}{0.22\textwidth}
  \centering
  % include second image
	\includegraphics[width=2\textwidth]{Figuras/forw} 
  \caption{$\gamma_{f}$}
  \label{}
\end{subfigure}
\begin{subfigure}{0.22\textwidth}
  \centering
  % include first image
	\includegraphics[width=2\textwidth]{Figuras/cm} 
  \caption{$\lambda$}
  \label{}
\end{subfigure}
\begin{subfigure}{0.22\textwidth}
  \centering
  % include second image
	\includegraphics[width=2\textwidth]{Figuras/theta} 
  \caption{$\theta$}
  \label{estimd}
\end{subfigure}\\
  \raggedright  \scriptsize \textbf{Fuente:} estimaciones propias.\\
\raggedright  \scriptsize \textbf{Nota:} la representación gráfica de los coeficientes se expresa  cuantiles. El color oscuro presenta menor tamaño del coeficiente y el más claro mayor. Color gris no presenta información.\\
\label{estim}	
\end{figure}
 
El grado de competencia en que se mueven las empresas es de la mayor importancia para entender la forma como éstas ajustan sus precios. En mercados altamente competitivos las empresas son más susceptibles de cambiar sus precios como respuesta a un choque, dado que el costo de oportunidad de no hacerlo a un nivel óptimo es muy alto \citep{misas2009formacion}. En este sentido, cuando se observa la probabilidad de mantener sin cambio los precios (figura \ref{estimd}), las empresas en los departamentos parecen operar en mercados poco competidos, principalmente en lugares aledaños al epicentro del país.\\
	
Por otra parte, la forma reducida del modelo NKPC para cada departamento muestra que las restricciones de sobreidentificación son válidas  en sus parámetros.\footnote{La hipótesis nula muestra  que las restricciones de sobreidentificación son válidas.} Además,  por naturaleza el modelo cuenta con autocorrelación, debido a que una de las características de la estructura teórica es la adición de la prospección en la dinámica inflacionaria. Para \cite{baardsen2004econometric}, la autocorrelación está relacionada con el problema de especificación incorrecta y puede generar afectación en los instrumentos planteados por GMM como lo evidencia \cite{zhang2009observed}. No obstante, esto no fue tenido en cuenta en los trabajos preliminares de la NKPC \citep{gali1999inflation,gali2001european,gali2005monetary}.\\

Para \cite{mehrotra2010modelling}, la posibilidad de que las tasas de inflación hayan experimentado choques comunes, evidencian movimientos comunes por parte de los residuos de las regresiones. La correlación de los residuos para el caso de los departamentos de Colombia (figura  \ref{anefig2}), es aparentemente alta entre 2010 y 2019. En promedio la correlación del residual del departamento hacia los demás va desde 0.42 (Chocó) a 0.70 (Meta), con un promedio total de 0.59. Sin embargo, al comparar el comportamiento de la tasa de inflación a partir de  los residuos de las regresiones según su estructura económica, los departamentos basados en la minería y  la industria presentan una correlación del 61\% y 62\%, respectivamente, mientras que las economías que no se agrupan en las anteriores estructuras están en 52\%. \\

Por otro lado, la división entre regiones muestra divergencias más claras en el comportamiento de la inflación. Las correlaciones de los residuos en el modelo planteado indican que las regiones del Centro Oriental y Occidental presentan en promedio movimientos similares del 66\%, mientras que para economías de la región del Caribe y el Pacífico los coeficientes de correlación son más bajos, correspondientes al 62\% y 57\%. Estos hallazgos se asemejan a los encontrados por \cite{quintero2019impactos}, donde las zonas costeras del país presentan mayores efectos diferenciales de la política monetaria.\footnote{La agrupación entre estructuras económicas y regiones se toma de \cite{quintero2019impactos}.} \\

Para complementar las estimaciones de la tabla \ref{tab31}, se construye un panel dinámico para estimar el modelo de la ecuación \eqref{21}. No obstante, \cite{mileva2007using} aclara que este tipo de modelos puede enfrentar diferentes problemas como: a) problemas de endogeneidad en donde el regresor se correlaciona con el término de error, b) características invariantes en el tiempo del departamento (efecto fijo), como la geografía y la demografía pueden estar correlacionados con las variables explicativas, c)  la existencia de autocorrelación con variables rezagadas, y d) la naturaleza de los datos panel al tener una dimensión corta de tiempo y amplia dimensión espacial. Por lo anterior, \cite{mileva2007using}  sugiere que estos problemas se resuelven usando GMM a partir del método de \cite{arellano1991some}.\\

\begin{figure}%[H]
  	\centering 		
  	\caption{Residuos de las regresiones}
	\includegraphics[width=1\textwidth]{Figuras/resi}
	\raggedright % \scriptsize \textbf{Nota:} 
		\label{anefig2}\\
  \raggedright  \scriptsize \textbf{Fuente:} estimaciones propias.
	\end{figure}
Inicialmente, se realiza estimaciones constatando que la NKPC se cumpla para el país. La tabla \ref{tab:D2} considera la ecuación preliminar en $\pi_{t-3}$, $\pi_{t+3}$ y $mcr_{t}$. A pesar que los coeficientes $\gamma_{b}$ y $\gamma_{f}$ son estadísticamente significativos en 1\% y para $\lambda$ en 5\%, este último es negativo. De igual manera, en periodos rezagados y en prospección para la inflación, la brecha de los costos marginales en periodos $t$, $t-3$, $t-6$ y $t-12$ presenta coeficientes negativos. Sin embargo, sería hasta  $t-18$ (un año y medio) que el coeficiente de la brecha de los costos marginales ($\lambda$)  explica la dinámica inflacionaria para el caso de Colombia, consistente con la teoría, como se presenta en la Tabla \ref{panel}. Estos modelos evidencian la capacidad explicativa de las variables en su conjunto a partir de la prueba de Wald ($\gamma_{f}+\gamma_{b}=1$), la cual presenta $prob>chi^{2} =0.000$, indicando que el total de los regresores explican significativamente la variable dependiente, $\pi_{t}$.

\begin{table}[H]
\centering
\caption{Panel dinámico de la NKPC en Colombia (2010-2019)}
\resizebox{15cm}{!} {
  \begin{tabular}{ p{3cm} p{1.2cm} p{1.2cm} p{1.5cm} p{1.2cm} p{1.2cm} p{1.2cm} p{1.2cm} p{1.2cm} p{1.2cm}}
  \hline
  Modelo  & $\gamma_{b}$  & $\gamma_{f}$ & $\lambda$   & $\theta$ &  Wald  &  AR(1)  &  AR(2)  & Hansen &  Sargan   \\
  &        $\pi_{t-k}$    &    $\pi_{t+k}$    &   $mcr_{t}$    &   $\frac{1}{1-\theta}$    &  \scriptsize{(p-value)}  &  \scriptsize{(p-value)}  &  \scriptsize{(p-value)}  & \scriptsize{(p-value)} &  \scriptsize{(p-value)}   \\
   \hline
    \hline
   \scriptsize{$\pi_{t-3}+\pi_{t+3}+mcr_{t-18}$} & 0.479*** & 0.509*** & 0.143** &  0.804  &   268.66  &  2.043  &  1.042  &  23.93 &  2927.3    \\
  & \scriptsize{(0.052)} & \scriptsize{(0.064)} & \scriptsize{(0.064)} & \scriptsize{5}     & \scriptsize{(0.000)} & \scriptsize{(0.041)} & \scriptsize{(0.297)} & \scriptsize{(1.000)} & \scriptsize{(0.000)} \\
   \scriptsize{$\pi_{t-6}+\pi_{t+6}+mcr_{t-18}$} & 0.513*** & 0.376* & 0.487*** &  0.612  &   27.74  &  2.467  &  1.948  & 20.93 &  2276.2   \\
  & \scriptsize{(0.139)} & \scriptsize{(0.198)} & \scriptsize{(0.173)} &  \scriptsize{3}    & \scriptsize{(0.000)} & \scriptsize{(0.013)} & \scriptsize{(0.041)} & \scriptsize{(1.000)} & \scriptsize{(0.000)} \\
   \hline
  \end{tabular}%
}
\label{panel}\\
  \raggedright  \scriptsize \textbf{Fuente:} estimaciones propias. \\
\raggedright  \scriptsize \textbf{Nota:} ***,  **, * representan significancia estadistica al 1\%, 5\% y 10\%, respectivamente. Bartlett KerneL, Newey-West fijo, errores estándar robustos de HAC entre paréntesis.\\
Instrumentos: inflación: t-12. \cite{wardhonoestimated}  utiliza  $\pi_{t-4}$ como variable instrumental para estudiar la NKPC para países del Sur de Asia con datos panel en trimestres.\\
\end{table}%

En primer lugar se contrasta con tres periodos de inflación tanto atrás como hacia adelante, luego en seis períodos.  Esta última presenta serios inconvenientes, como lo son:  a) en sus coeficientes la robustez de los errores estándar son grandes, b)  a pesar de que se acepta la hipótesis nula en el test de \cite{hansen1982generalized}  frente a la validez de las restricciones de sobreidentificación, al tomar el valor 1 se anula lo mencionado (no se considera el test \cite{sargan1958estimation} la cual considera la misma hipótesis de Hansen),\footnote{Es más conveniente utilizar el test de Hansen cuando los errores están distribuidos de forma heterocedástica (Two-step para xtabond2 en Stata). Lo anterior se comprueba en la prueba de White al rechazar la hipótesis nula que señala la presencia de homocedasticidad.}  por lo tanto, se rechaza la hipótesis nula, no estaría cumpliendo con el test de \cite{roodman2009note} (este problema ocurre en todas las estimaciones).\footnote{Este test señala que el número de instrumentos debe ser menor que al número de grupos, lo cual para este caso es de 116 instrumentos y 24 grupos.  La reducción de los instrumentos esta sujeta al aumento de los rezagos, por lo tanto, se opta por elegir el modelo acorde con otras condiciones.} c)  Presencia de autocorrelación serial de segundo orden según el test de \cite{arellano1991some} (AR(2)), debido a que se rechaza la hipótesis nula de no autocorrelación.\footnote{Es previsible que exista correlación serial de primer orden (AR (1) $pr > z < 0.05$). En este caso estimar el modelo utilizando directamente el regresor para este caso $\pi_{t-1}$ estaría sesgado.} \\

Por lo anterior, el primer modelo configura gran parte de su validez al considerar que el coeficiente de la inflación vista hacia el futuro, $\gamma_{f}$, aún sigue siendo importante en la explicación de la dinámica inflacionaria del país, como se evidenció en cada uno de los departamentos. Así mismo, el bajo tamaño en la robustez de los errores y la no autocorrelación serial (AR(2)) lo hace apto en su validez, a pesar de los problemas de sobreidentificación. No obstante, los resultados arrojados se asemejan al grado de rigidez de precios en las empresas evidenciado por \cite{galvis2010estimacion}, señalando que alrededor del 80\% empresas dejan fijo los precios en promedio durante 5 periodos, que para el caso del presente modelo sería en meses, a su vez que alrededor del 20\% de las empresas ajustan su precio según el valor previo de los costos marginales reales.

\begin{figure}[H]
\caption{Resultados de las estimaciones en $\gamma{b}$, $\gamma{f}$ y $\lambda$}
\begin{subfigure}{0.48\textwidth}
  \centering
  % include first image
	\includegraphics[width=1\textwidth]{Figuras/pas} 
  \caption{$\gamma{b}$ vs $\lambda$}
  \label{}
\end{subfigure}
\begin{subfigure}{0.48\textwidth}
  \centering
  % include second image
	\includegraphics[width=1\textwidth]{Figuras/fut} 
  \caption{$\gamma{f}$ vs $\lambda$}
  \label{}
\end{subfigure}
	\label{grap}\\
  \raggedright  \scriptsize \textbf{Fuente:} estimaciones propias.
\end{figure}	

En resumen, la figura \ref{grap} recoge los coeficientes expuestos en las tablas \ref{tab31} y \ref{panel}, donde se contrasta la relación entre  la brecha de los costos marginales ($\lambda$), inflación rezagada ($\gamma{b}$) y la inflación esperada ($\gamma{f}$). En general, los coeficientes presentan mayor ponderación en cuanto a brecha de los costos marginales e inflación esperado se refiere, esencial en el modelo NKPC. Sin embargo, al comparar el coeficiente obtenido por información del panel y de manera agregada por departamento, permanecen distantes a los coeficientes individuales. El primero presenta mayor incidencia de la inflación rezagada que al resto de los departamentos según la explicación que se le da a la dinámica inflacionaria, mientras que en el otro caso, el  coeficiente negativo de la brecha de los costos marginales sobresale que al resto de los coeficientes por su valor negativo. 

\section{Rigidez de precios y variables por departamentos}
A partir de las diferencias presentadas en el modelo NKPC híbrido para explicar  la dinámica inflacionaria de cada departamento, este mismo no infiere de las posibles razones de la formación en la rigidez de precios. Por lo tanto, en este apartado se adaptan los modelos probit  y OLS para analizar el proceso inflacionario frente a otras variables económicas. La tabla \ref{probit} muestra los resultados de la estimación probit con base en una variable ficticia que toma valores de uno cuando la inflación es prospectiva y la brecha del costo marginal es significativa al 1\% y 5\%, de lo contrario dicha variable tomaría cero,  similar a las estimaciones realizadas por \cite{mehrotra2010modelling}.

\begin{table}[H]
\centering
\caption{Resultados de la estimación probit}
\begin{tabular}{cccc}
\hline
& (1)   & (2)   & (3) \\ 
\hline
\hline
\vspace{-0.3cm}   Apertura & -0.159* & -0.148* & -0.165* \\
& \scriptsize{(0.089)} & \scriptsize{(0.079)} & \scriptsize{(0.088)} \\
\vspace{-0.3cm}    Tasa de ocupación  & -0.358** & -0.400** & -0.391* \\
& \scriptsize{(0.181)} & \scriptsize{(0.200)} & \scriptsize{(0.202)} \\
\vspace{-0.3cm}    Sector primario  & -0.424* & -0.517** & -0.471* \\
& \scriptsize{(0.230)} & \scriptsize{(0.245)} & \scriptsize{(0.242)} \\
\vspace{-0.3cm}    Tasa de crecimiento del PIB & -6.073 &       &  \\
& \scriptsize{(9.672)} &       &  \\
\vspace{-0.3cm}    Participación del PIB  &       & -0.358 &  \\
&       & \scriptsize{(0.366)} &  \\
\vspace{-0.3cm}     PIB per cápita  &       &       & -0.001 \\
&       &       & \scriptsize{(0.001)} \\
\vspace{-0.3cm}     constante & 26.493** & 30.243** & 29.648** \\
& \scriptsize{(13.427)} & \scriptsize{(14.741)} & \scriptsize{(15.028)} \\
\hline
\end{tabular}%
\label{probit}\\
  \raggedright  \scriptsize \textbf{Fuente:} estimaciones propias. \\
\raggedright  \scriptsize \textbf{Nota:} ***,  **, * representan significancia estadistica al 1\%, 5\% y 10\%, respectivamente. Errores estándar entre paréntesis.
\end{table}%

La apertura definida como la relación entre el comercio exterior y el PIB regional es estadísticamente significativa al 10\% con signo negativo en los tres modelos, esto sugiere que la NKPC se ajusta a los datos de aquellos departamentos que no presentan una transición de la economía basada más en el mercado. La tasa de ocupación  y el sector primario al igual que la variable de apertura, explican la variable dummy de manera negativa. Una menor tasa de ocupación puede estar expresada en la dinámica inflacionaria a partir de la inflexibilidad parcial o total de los precios a la baja, desajustes sectoriales que afecten a bienes determinados o por características estructurales de los mercados de trabajo y de bienes.\\

El marco neokeynesiano con visión de futuro no implica ningún compromiso entre inflación y costos marginales o estabilización de la brecha del producto, señalando \cite{blanchard2007real} que la coincidencia divina desaparece una vez que se introducen imperfecciones reales (por ejemplo, rigideces de los salarios reales) al modelo. Por otra parte, el sector primario caracterizado por revisar sus precios con mayor frecuencia que en otros sectores \citep{misas2009formacion}, presume un posible direccionamiento  contrario al que supone el modelo en un entorno de rigideces de precios. No obstante, variables que no se tienen en cuenta en el momento pueden explicar las estimaciones obtenidos en la NKPC para los departamentos de Colombia.\\

Los resultados encontrados en la tabla 	\ref{tab31}  referente a la probabilidad  de mantener  los precios sin cambios en el tiempo en empresas para cada departamento ($\theta$), se intentan dar posibles razones de tal fenómeno, por medio de la estimación OLS explicadas por diferentes variables (tabla \ref{mco}). En este caso el índice departamental de competitividad (IDC)\footnote{Los componentes que integran el IDC pueden ser vistos en el pie de tabla \ref{variab}.}  explica de manera negativa la rigidez de precios en los departamentos, es decir, los supuestos implícitos en el modelo sobre competencia imperfecta hacen que el IDC  se comporte con el coeficiente esperado, por lo tanto, a mayor valor del IDC que tome el departamento menor será la probabilidad de mantener los precios rígidos por parte de las empresas.\\

En contraste, la rigidez en los precios es explicado positivamente para variables como el PIB per cápita y el sector terciario. En el caso especial del sector terciario, la mayor participación en el PIB departamental por parte de  subsectores como información y comunicaciones, actividades financieras, inmobiliarias, profesionales y en materia de administración pública y defensa, han originado posibles rigideces de precios en los departamentos. La menor productividad del sector terciario por su baja exposición a la competencia tanto a nivel local como internacional, así como la participación estatal en la baja competitividad, generando mayor rigidez en la oferta, permite inferir un comportamiento de precios rígidos en el tiempo establecidos por las empresas.\\

\begin{table}%[H]
\centering
\caption{Resultados de  estimación por OLS}
\begin{tabular}{cccc}
\hline
& (1)   & (2)  &  (3) \\
\hline
\hline
\vspace{-0.3cm}    IDC  & -0.067** & -0.076*** & -0.071*** \\
& \scriptsize{(0.025)} & \scriptsize{(0.026)} & \scriptsize{(0.025)} \\
\vspace{-0.3cm}    PIB per cápita  & 0.001* & 0.001** & 0.001** \\
& \scriptsize{(0.001)} & \scriptsize{(0.001)} & \scriptsize{(0.001)} \\
\vspace{-0.3cm}    Sector terciario  & 0.009*** & 0.009*** & 0.008*** \\
& \scriptsize{(0.003)} & \scriptsize{(0.003)} & \scriptsize{(0.003)} \\
\vspace{-0.3cm}    Tasa de crecimiento del PIB & 0.481 &       &  \\
& \scriptsize{(0.682)} &       &  \\
\vspace{-0.3cm}    Tasa de ocupación &       & 0.003 &  \\
&       & \scriptsize{(0.004)} &  \\
\vspace{-0.3cm}    Apertura &       &       & 0.001 \\
&       &       & \scriptsize{(0.001)} \\
\vspace{-0.3cm}     constante & 0.683*** & 0.535** & 0.709*** \\
& \scriptsize{(0.126)} & \scriptsize{(0.242)} & \scriptsize{(0.112)} \\
\hline
\end{tabular}
\label{mco}\\
  \raggedright  \scriptsize \textbf{Fuente:} estimaciones propias. \\
\raggedright  \scriptsize \textbf{Nota:} ***,  **, * representan significancia estadistica al 1\%, 5\% y 10\%, respectivamente. Errores estándar entre paréntesis.\\
El grado de rigidez de precios ($\theta$) obtenido en la Tabla \ref{tab31} y que hace de variable explicada en estos modelos, es despejada a partir de los valores absolutos de $\lambda$.\\
Los modelos estimados no presentan multicolinealidad, se acepta la hipótesis de normalidad en los residuales de los  modelos, y por último, se encuentra que  las varianzas de los
residuales son constantes, por lo que se dice que la varianza de los residuales es homocedastica.
\end{table}%

Por otro lado, otras variables como la tasa de crecimiento del PIB, tasa de ocupación y el grado de apertura de manera aparecen con coeficientes no significativos para explicar este fenómeno. Esta última esperaría que las economías caracterizadas por problemas estructurales e instituciones débiles (incluidos, por ejemplo, los mercados emergentes)  presenten mayores pesos de persistencia de la inflación en el NKPC que en mercados liberales. Sin embargo, al igual que los modelos probit, aún queda por explicar el proceso inflacionario en variables que posiblemente fueron descartadas por ausencia de información. 

\section{Discusión de los resultados}
Los resultados anteriores específicos para cada departamento permiten sacar algunas conclusiones generales. En particular, las tasas de inflación en todos los departamentos analizados tienen importantes componentes prospectivos y, por lo tanto, la inflación actual está determinada (al menos parcialmente) por su valor futuro esperado. Además, el término retrospectivo también es significativo, pero tiene menos incidencia que el coeficiente de inflación esperada. Esto es consistente con el resultado de estudios previos utilizando mayor relevancia en la inflación con expectativas \citep{gali2001european,ramos2008inflation,mehrotra2010modelling,vavsivcek2011inflation}.\\ 

Los resultados son robustos a la inclusión de rezagos adicionales de la inflación (tabla \ref{tabane1}).\footnote {Los resultados por la metodología de eliminación de tendencia para el caso de la brecha de los costos marginales. Sin embargo, los datos con su tendencia no cambian significativamente los resultados de las estimaciones.}  Sin embargo,  aún queda por abordar problemas referentes a la instrumentación débil e impertinente vistos en la inexactitud de la estimación de los parámetros estructurales de la NKPC \citep{ma2002gmm}  y, por otra parte, revisar con mayor detenimiento los componentes de los costos marginales. Por lo tanto,  la  solución a estos inconvenientes, posiblemente van a facilitar la   comprensión  del proceso inflacionario en un país como Colombia que conduce su política monetaria dentro de un marco de metas de inflación. \\

El modelo estructural detrás de la NKPC sugiere que el efecto de la política monetaria sobre la inflación se produce a través del costo marginal, pero si esta no es una variable que impulse la inflación, la política monetaria puede influir en la inflación solo a través de su credibilidad y su efecto sobre las expectativas de inflación. Los resultados utilizando el GMM tienden a respaldar la NKPC  y  resaltan la participación de la inflación esperada en cada uno de los departamentos, a pesar que el coeficiente $\lambda$  es estadísticamente significativo y consistente con la teoría en  solo 7 de los 24 departamentos (incluido la capital del país, Bogotá). \\

Por otra parte, cada vez es más amplia la discusión de si la participación del ingreso laboral es una buena \textit{proxy} del costo marginal real, por ejemplo, para este caso se evidenció que tan solo la mitad de los departamentos cumplió con el requisito en que $\lambda>0$, para explicar la prociclicidad de los costos marginales. Sin embargo, para mejorar este supuesto de una manera razonable, \cite{mazumder2010new} sugiere reconocer el comportamiento del empleo cuasi fijo y que el salario real este en  función de las horas, lo cual supondría más realismo durante el ciclo económico al tenerse en cuenta en la medida de los costos marginales reales.  \\

Los costos marginales en algunos departamentos son de hecho relevantes en la explicación de la dinámica inflacionaria, enfatizada por la NKPC. Las diferencias en el proceso de formación de precios entre los departamentos son importantes, porque de ello dependerá directamente la efectividad de la política monetaria. El coeficiente de \cite{calvo1983staggered} ($\theta$)  que explica la probabilidad de mantener los precios fijos en el tiempo es heterogénea entre departamentos, donde mantiene un rango de 0.695 a 0.958, lo cual implica duraciones de precios en promedio entre 3 hasta 24 meses.  Al comparar por regiones, la zona central perteneciente a Boyacá, Cundinamarca, Tolima y Caldas presentan menor rigidez de precios (figura \ref{estim}), asociado a la baja persistencia de la inflación ($\gamma_{b}$) y por consiguiente mayor énfasis en la inflación esperada ($\gamma_{b}$) por parte de las empresas. \\
%En el caso del coeficiente de la brecha de los costos marginales ($\lambda$) no es observable algún patrón espacial.\\

Acerca del total de las regiones, cuando se considera que la fijación de precios se realiza con base en el valor actual de los costos marginales reales, estos presentan comportamiento anticíclico, contrario a la relación procíclica que debe presentar la inflación y los costos marginales. No obstante, la estimación de la NKPC por medio de un panel dinámico, explica que hasta el valor en el período 18 de los costos marginales se observan de manera significativa movimientos similares a la inflación, dando como resultados iguales a los obtenidos por \cite{galvis2010estimacion} en términos de la probabilidad de mantener inalterados los precios, aunque en diferentes periodos y tamaño en los coeficientes $\gamma_{b}$, $\gamma_{f}$ y $\lambda$.\footnote{Al tomar los datos en trimestres \cite{galvis2010estimacion}, se dice que en promedio las empresas mantienen los precios rígidos en 6 trimestres, contrario en este caso que es a 6 meses.} Este hecho puede deberse a un lento movimiento del costo marginal, factor  importante que explica el alto grado evidenciado en la persistencia de la inflación.
 


\chapter*{Conclusiones} \label{conclusiones}
\addcontentsline{toc}{chapter}{Conclusiones}
Este estudio emplea una Curva de Phillips Neokeynesiana (NKPC) híbrida para analizar la dinámica de los precios en los departamentos en Colombia. La evidencia muestra que el proceso inflacionario ha sido más prospectivo en los últimos años, lo que también es consistente con una mayor credibilidad en la meta de inflación, puesto que la eficacia de la política monetaria depende del papel de expectativas en la determinación de la inflación, que es de importancia para la conducción de la política en una economía con diferencias regionales como la colombiana. No obstante, la inflación rezagada (persistencia de la inflación)  también es significativa para todos los departamentos y aunque con menor incidencia   también juega un papel clave en la dinámica inflacionaria. \\

Los resultados obtenidos muestran que la formación de precios presenta comportamientos diferenciales en las regiones geográficas del país. Las regiones ubicadas en el centro del país  presentan en su mayoría menor probabilidad de que sus empresas mantengan los precios sin cambios en el tiempo, destacando el papel de la inflación esperada y los costos marginales. Por el contrario, las regiones situadas en la zonas costeras parecen evidenciar mayor grado de rigidez en los precios, puesto que las empresas tardan en modificar los precios. Además, al evaluar los choques comunes que resulta de las correlaciones de los residuos en las estimaciones, el comportamiento de la inflación diverge en departamentos que no se basan en economías mineras e industriales y por regiones como en el caso del Pacifico y el Caribe. \\

No obstante, al evaluar la dinámica inflacionaria independiente sobre cada uno de los departamentos, las diferencias encontradas no están relacionadas con la ubicación geográfica. En el caso de la región Caribe, el fuerte impacto de la política monetaria puede ser explicado por la importancia que tiene las expectativas en la inflación principalmente en el departamento de Bolivar. Para la región Pacifica es clave Nariño y Valle del Cauca, para el Centro Oriente son Boyacá, Santander y Tolima, y por último, en el caso de Centro Occidental es determinante el departamento de Caldas.\\

Dadas las diferencias departamentales encontradas, en este trabajo se buscó también determinar si se podía encontrar una explicación a estas diferencias y su incidencia en el comportamiento de los precios a partir de la composición económica de cada departamento, el grado de apertura y otras variables económicas. Por lo tanto, se detecta un bajo grado de desarrollo del sistema de mercado (apertura al comercio) y la relativa exposición a presiones excesivas de demanda (tasa de ocupación, sector primario, tasa de crecimiento del PIB) explica la relevancia del modelo NKPC en los departamentos en Colombia. Así mismo, el grado de  rigidez de precios presentados en las empresas de cada departamento es explicado por su baja competitividad, incidencia por mayores ingresos en la población  y alta participación del sector terciario en las economías. Las diferencias en los procesos y mecanismos de inflación entre departamentos tienen implicaciones importantes para la conducción de política monetaria en Colombia.\\

Los resultados encontrados son alentadores en relación con la literatura internacional, evidenciando que la curva de Phillips neokeynesiana es también verificada empíricamente para la economía colombiana y puede dar luces sobre la explicación de la dinámica inflacionaria. La importancia de tener una estimación a escala regional a partir de micro fundamentos alrededor de la NKPC, sugiere  la participación de la política económica como herramienta estabilizadora ante los ciclos adversos que enfrenta la economía colombiana.

%\include{Kap4/Kap4}
%\include{Kap5/Kap5}
%\include{Kap6/Kap6}
\addcontentsline{toc}{chapter}{\numberline{}Bibliografia}
\bibliography{biblio2}
\bibliographystyle{apalike2}
\include{Anexos/Anexos}
\end{document}