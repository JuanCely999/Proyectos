\documentclass[10pt]{beamer}
%\usepackage[xllnames,table]{xcolor}
%https://deic-web.uab.cat/~iblanes/beamer_gallery/index.html
%La opción pages puede ser all (para todo el documento) o some, para algunas partes del documento
\usepackage[pages=all]{background}
\usepackage{eso-pic}


\usepackage{color} % si
%\usetheme{AnnArbor} % AnnArbor-defaulttema de presentación si
 
%\usefonttheme{serif} % tema de fuente
%\useinnertheme{circles} % tema interior
\useoutertheme{split} % tema exterior si
%\definecolor{naranja}{cmyk}{0,0.5,1,0}
\usetheme{Antibes}
\usecolortheme{dove}% tema de color siwhale
\usepackage[utf8]{inputenc}
\usepackage[spanish]{babel}
\usepackage{ragged2e}
\usepackage{amsmath}
\usepackage{amsfonts}
\usepackage{amssymb}
\usepackage{graphicx}
\usepackage{amssymb, amsmath, amsbsy} % simbolitos
\usepackage{upgreek} % para poner letras griegas sin cursiva
\usepackage{cancel} % para tachar
\usepackage{mathdots} % para el comando \iddots
\usepackage{mathrsfs} % para formato de letra
\usepackage{stackrel} % para el comando \stackbin
\setbeamercovered{transparent}
\author[Alonso Ramírez Hernández]{Alonso Ramírez Hernández}
\institute[]{Asesor en proyectos}
%\date[2018]{\scriptsize{Octubre 3, 2018}}
\title[Proyecto VISOTA]{Proyecto de Aprovechamiento de residuos sólidos para su utilización en equipamiento urbano}
\subtitle{Bemore Group SAS}
%\setbeamercovered{transparent} 
%\setbeamertemplate{navigation symbols}{} 
%\logo{} 
%\institute{} 
%\date{} 
%\subject{} 
%Para todo el documento
%\logo{\includegraphics[width=1.5cm]{Fig/visota}} 
%importante sisolo el grafico en la portada o en todo el documento
\titlegraphic{\includegraphics[width=1.0cm]{Fig/visota}}
\justifying
\begin{document}

\begin{frame}
\titlepage
\end{frame}

%\begin{frame}
%\tableofcontents
%\end{frame}


\begin{frame}
\frametitle{Contenido}
\begin{enumerate}
\item Análisis de competencia 
\item  Escenario nacional
\begin{enumerate} [a]
\item Oferta
\item Demanda
\end{enumerate}
\item Escenario departamental
\begin{enumerate} [a]
\item Oferta
\item Demanda
\end{enumerate}
\item Proyecciones
\end{enumerate}
\end{frame}


\begin{frame}
\frametitle{Contenido}
\begin{enumerate}[<i->]
\item<1-> Análisis de competencia 
\item  Escenario nacional
\begin{enumerate} [a]
\item Oferta
\item Demanda
\end{enumerate}
\item Escenario departamental
\begin{enumerate} [a]
\item Oferta
\item Demanda
\end{enumerate}
\item Proyecciones
\end{enumerate}
\end{frame}


\section{Análisis de competencia}
\begin{frame}
\frametitle{1. Análisis de competencia}
%\framesubtitle{Matriz Insumo-Producto}
\begin{figure}
   
\includegraphics[width=1\textwidth]{Fig/flujo2}
\centering
\caption{Esquema de oferta y demanda de residuos sólidos}
  \label{fig:ejemplo}  
\end{figure}
%{\small Las matrices de insumo producto son cuadros de doble entrada, en los cuales se registran las principales transacciones que sostienen los agentes de un sistema económico. Pueden desagregarse en múltiples industrias o productos, dependiendo de la disponibilidad de información, así como del grado de diversificación productiva de la economía que representan}

\end{frame}



\begin{frame}
\frametitle{Contenido}
\begin{enumerate}[<i->]
\item Análisis de competencia 
\item<1->  Escenario nacional
\begin{enumerate} [a]
\item<1-> Oferta
\item<1-> Demanda
\end{enumerate}
\item Escenario departamental
\begin{enumerate} [a]
\item Oferta
\item Demanda
\end{enumerate}
\item Proyecciones
\end{enumerate}
\end{frame}



\section{Escenario nacional}
\begin{frame}
\frametitle{2. Escenario nacional: Oferta}
%\framesubtitle{Escenario nacional}
 \begin{flushleft}
 {\tiny 562 empresas}
   \end{flushleft} 
   
\begin{figure}
\includegraphics[width=1\textwidth]{Fig/colombia5}
\centering
  \label{fig:ejemplo}  
\end{figure}


\end{frame}


\begin{frame}
\frametitle{2. Escenario nacional: Oferta}
%\framesubtitle{Escenario nacional}
\begin{figure}
\includegraphics[width=0.5\textwidth]{Fig/serie}
\centering
  \label{fig:ejemplo}  
\end{figure}
\end{frame}



\begin{frame}
\frametitle{2. Escenario nacional: Oferta}
%\framesubtitle{Escenario nacional}
\begin{flushleft}
{\tiny Balance de oferta, utilización de residuos sólidos }
 \end{flushleft} 

\begin{figure}
{\includegraphics[width=0.47\textwidth]{Fig/oferdem}} 	{\includegraphics[width=0.47\textwidth]{Fig/tasas}}	
	\end{figure}

\end{frame}

\begin{frame}
\frametitle{2. Escenario nacional: Demanda}
%\framesubtitle{Escenario nacional}

\begin{flushleft}
{\tiny Para el año 2019 se aprobaron 42.570.883 $m^2$ de los cuales 33.436.660 $m^2$ fueron destinados para vivienda en Colombia.}
 \end{flushleft} 

\begin{figure}
\includegraphics[width=1\textwidth]{Fig/destino}
\centering
  \label{fig:ejemplo}  
\end{figure}
\end{frame}

\begin{frame}
\frametitle{2. Escenario nacional: Demanda}
%\framesubtitle{Escenario nacional}

\begin{flushleft}
{\tiny Viviendas de interés social (VIS) vs Vivienda diferentes a las de interés social}
 \end{flushleft} 

\begin{figure}
\includegraphics[width=1\textwidth]{Fig/colvivi}
\centering
  \label{fig:ejemplo}  
\end{figure}
\end{frame}


\section{Escenario departamental}
\begin{frame}
\frametitle{Contenido}
\begin{enumerate}[<i->]
\item Análisis de competencia 
\item  Escenario nacional
\begin{enumerate} [a]
\item Oferta
\item Demanda
\end{enumerate}
\item<1-> Escenario departamental
\begin{enumerate} [a]
\item<1-> Oferta
\item<1-> Demanda
\end{enumerate}
\item Proyecciones
\end{enumerate}
\end{frame}


\begin{frame}
\frametitle{3. Escenario departamental: Oferta}
%\framesubtitle{Escenario nacional}

\begin{flushleft}
{\tiny 32 empresas}
 \end{flushleft} 

\begin{figure}
\includegraphics[width=1\textwidth]{Fig/boyaca5}
\centering
  \label{fig:ejemplo}  
\end{figure}
\end{frame}


\begin{frame}
\frametitle{3. Escenario departamental: Demanda}
%\framesubtitle{Escenario nacional}

\begin{flushleft}
{\tiny }
 \end{flushleft} 

\begin{figure}
\includegraphics[width=1\textwidth]{Fig/boyvivi}
\centering
  \label{fig:ejemplo}  
\end{figure}
\end{frame}

\section{Proyecciones}
\begin{frame}
\frametitle{Contenido}
\begin{enumerate}[<i->]
\item Análisis de competencia 
\item  Escenario nacional
\begin{enumerate} [a]
\item Oferta
\item Demanda
\end{enumerate}
\item Escenario departamental
\begin{enumerate} [a]
\item Oferta
\item Demanda
\end{enumerate}
\item<1-> Proyecciones
\end{enumerate}
\end{frame}


\begin{frame}
\frametitle{4. Proyecciones: Fabricación de ladrillos en Colombia}
%\framesubtitle{Escenario nacional}

\begin{flushleft}
{\tiny }
 \end{flushleft} 

\begin{figure}
\includegraphics[width=1\textwidth]{Fig/ladricol}
\centering
  \label{fig:ejemplo}  
\end{figure}
\end{frame}

\begin{frame}
\frametitle{4. Proyecciones: Fabricación de ladrillos en Boyacá}
%\framesubtitle{Escenario nacional}

\begin{flushleft}
{\tiny }
 \end{flushleft} 

\begin{figure}
\includegraphics[width=1\textwidth]{Fig/ladriboy}
\centering
  \label{fig:ejemplo}  
\end{figure}
\end{frame}
%
%\begin{frame}
%\frametitle{1. Introducción}
%\framesubtitle{Revisión de la literatura}
%%\tiny\scriptsize\small\normalsize\large\Large\LARGE\huge\Huge
%\begin{itemize}
%
%\item {\small Isard (1951), quien propuso un modelo interregional para desagregar el comercio entre diferentes regiones, y Leontief (1953), quien concibió un modelo intranacional que muestra cómo medir el impacto de políticas nacionales sobre economías departamentales. De forma paralela, Isard y Kuenne (1953) y luego Miller (1957) progresan en técnicas para medir el impacto de la expansión de un sector industrial sobre una región particular.}
%
%\item {\small Bonet (2000), propone un modelo econométrico integrando todos los departamentos de la región Caribe y estima los multiplicadores de producto, ingreso y empleo.
%Banguero, Duque, Garizado y Parra (2006), analiza la interdependencia sectorial del Valle del Cauca.
%Villa y Giraldo (2014), examina los encadenamientos intersectoriales y multiplicadores de producto, empleo e ingreso en la economía de Medellín.}
%
%\end{itemize}
%\end{frame}
%
%
%
%\begin{frame}
%\frametitle{1. Introducción}
%\framesubtitle{Construcción matriz Insumo-Producto
%}
%\begin{block}
%
%
%\begin{equation}
%X=(I-A)^{-1} Y
%\end{equation}
%\end{block}
%\end{frame}
%
%\begin{frame}
%\frametitle{1. Introducción}
%
%\framesubtitle{Regionalización de la matriz de coeficientes técnicos}
%Métodos non-survey (Coeficientes de departamentalización, versión ajustada por medio  del Coeficiente de Flegg)
%
%{\scriptsize
%\begin{equation}
% a_{ij}^R=LQ*a_{ij}^N
%\end{equation}
%
%\begin{equation}
%SLQ_{i}=\frac{{x_{i}^R}/{x^R}}{{x_{i}^N}/{x^N}}
%\end{equation}
%
%\begin{equation}
%CILQ_{ij}=\frac{SLQ_{i}}{SLQ{j}}
%\end{equation}
%
%\begin{equation}
%FLQ_{ij}=CILQ{ij}*\lambda  {si}{i}\neq{j}
%\end{equation}
%
%\begin{equation}
%FLQ_{ij}=SLQ{ij}*\lambda  {si}{i}={j}
%\end{equation}
%
%\begin{equation}
%\lambda=(log_2(1+\frac{x^R}{x^N}))^\delta
%\end{equation}
%}
%\end{frame}
%
%
%
%\section{Parte l: Multiplicadores}
%
%\begin{frame}
%\frametitle{Contenido}
%\begin{enumerate}[<i->]
%\item Introducción 
%\begin{enumerate} [a]
%\item Matriz Insumo-Producto
%\item Revisión de la literatura
%\item Construcción matriz Insumo-Producto
%\item Regionalización de la matriz de coeficientes técnicos
%\end{enumerate}
%\item<1-> Parte l: Multiplicadores
%\begin{enumerate} [a]
%
%\item<1-> Construcción  multiplicadores producto, empleo e ingreso
%\item<1-> Resultados
%\end{enumerate}
%\item Parte ll: Eslabonamientos
%\begin{enumerate} [a]
%\item Construcción de eslabonamientos hacia atrás y hacia adelante
%\item Resultados
%\end{enumerate}
%\item Conclusiones
%
%\end{enumerate}
%\end{frame}
%
%
%
%\begin{frame}
%\frametitle{2. Parte l: Multiplicadores}
%\framesubtitle{Construcción  multiplicadores producto, empleo e ingreso}
%
%
%\begin{itemize}
%\item Multiplicador de producción
%
%{\tiny
%\begin{equation}
%MP_{j}=\sum_{i=1}^n(I-A)^{-1}
%\end{equation}
%}
%
%\item Multiplicador de empleo
%\begin{block}
%{\tiny
%\begin{equation}
%L_{1}=E(I-A)^{-1}
%\end{equation}}
%\end{block}
%
%{\tiny
%\begin{equation}
%X=E(I-N)^{-1}Y
%\end{equation}}
%
%
%\begin{alertblock}
%{\tiny
%\begin{equation}
%L_{2}=E(I-N)^{-1}
%\end{equation}}
%\end{alertblock}
%{\tiny(GEIH) del DANE para 2010}
%
%\item Multiplicador de ingreso
%\begin{block}
%{\tiny
%\begin{equation}
%P_{1}=\gamma(I-A)^{-1}
%\end{equation}}
%\end{block}
%
%\begin{alertblock}
%{\tiny
%\begin{equation}
%P_{2}=\gamma(I-N)^{-1}
%\end{equation}}
%\end{alertblock}
%\end{itemize}
%
%\end{frame}
%
%
%\begin{frame}
%\frametitle{2. Parte l: Multiplicadores}
%\framesubtitle{Resultados}
%{\tiny
%\begin{table}
%\centering
%\caption{Correlación entre multiplicadores}
%\begin{tabular}{|c|c|c|c|}
%
%\hline 
% & Multiplicador de producto & Multiplicador de Ingreso & Multiplicador de empleo \\ 
%\hline 
%Multiplicador de producto & 1.0000  & • & • \\ 
%\hline 
%Multiplicador de Ingreso & 0.4236  & 1.0000  & • \\ 
%&  0.0140 & & \\
%\hline 
%Multiplicador de empleo &  0.3472 & \textbf{0.9063}  & 1.0000 \\ 
%&  0.0477  & \textbf{0.0000} & \\
%\hline 
%
% \label{Tabla:ejemplo}
%\end{tabular} 
%\end{table}
%}
%%\begin{figure}
%  % 
%%\includegraphics[width=11.0cm]{Graficas/correlaciones}
%%
%
%\begin{figure}
%   %\caption{Correlación entre multiplicadores}
%%\put(-100,50){\footnotesize{efkrewfire}}
%\includegraphics[scale=0.15]{Graficas/productoingreso}
%\includegraphics[scale=0.15]{Graficas/productoempleo} 
%  
%%https://www.youtube.com/watch?v=8sA2z1Qkqtw&t=316s
%\includegraphics[scale=0.2]{Graficas/empleoingreso}
%\end{figure}
%  
%%\end{figure} 
%\end{frame}
%
%\begin{frame}
%\frametitle{2. Parte l: Multiplicadores}
%\framesubtitle{Resultados}
%\begin{figure}
%   \caption{Multiplicadores de producto, ingreso y empleo}
%\includegraphics[width=10.0cm]{Graficas/multiplicadores3}
%\centering
%\label{fig:ejemplo}
%\end{figure} 
%\end{frame}
%
%
%\section{Parte ll: Eslabonamientos}
%
%\begin{frame}
%\frametitle{Contenido}
%\begin{enumerate}[<i->]
%\item Introducción 
%\begin{enumerate} [a]
%\item Matriz Insumo-Producto
%\item Revisión de la literatura
%\item Construcción matriz Insumo-Producto
%\item Regionalización de la matriz de coeficientes técnicos
%\end{enumerate}
%\item Parte l: Multiplicadores
%\begin{enumerate} [a]
%
%\item Construcción  multiplicadores producto, empleo e ingreso
%\item Resultados
%\end{enumerate}
%\item<1-> Parte ll: Eslabonamientos
%\begin{enumerate} [a]
%\item<1-> Construcción de eslabonamientos hacia atrás y hacia adelante
%\item<1-> Resultados
%\end{enumerate}
%\item Conclusiones
%
%\end{enumerate}
%\end{frame}
%
%\begin{frame}
%\frametitle{3. Parte ll: Eslabonamientos}
%\framesubtitle{Construcción de eslabonamientos hacia atrás y hacia adelante}
%{\tiny
%\begin{equation}
%A= \left(
%\begin{array}{cc}
%A_{jj}  & A_{jr}\\
%A_{rj} & A_{rr} 
%\end{array}
%\right)= \left(
%\begin{array}{cc}
%A_{jj}  & A_{jr}\\
%A_{rj} & 0 
%\end{array}
%\right)+ \left(
%\begin{array}{cc}
%0  & 0\\
%0 & A_{rr} 
%\end{array}
%\right)=A_{j}+A_{r}
%\end{equation}
%
%\begin{equation}
%P_{1}=(I-A_{r})^{-1}
%\end{equation}
%
%
%\begin{equation}
%P_{1}A_{j}= \left(
%\begin{array}{cc}
%A_{jj}  & A_{jr}\\
%\Delta A_{rj} & 0 
%\end{array}
%\right)
%\end{equation}}
%\begin{itemize}
%
%\item Eslabonamientos hacia atrás
%\begin{block}
%{\tiny
%\begin{equation}
%PBL=i_{rr}\Delta_{r}A_{rj}q_{jj}
%\end{equation}}
%\end{block}
%\item Eslabonamientos hacia adelante
%{\tiny
%\begin{equation}
%A_{j}P_{1}= \left(
%\begin{array}{cc}
%A_{jj}  & \Delta_{r}\\
%A_{rj} & 0 
%\end{array}
%\right)
%\end{equation}}
%\begin{block}
%{\tiny
%\begin{equation}
%PFL=A_{jr}\Delta_{r}q_{rr}
%\end{equation}}
%\end{block}
%\end{itemize}
%\end{frame}
%
%\begin{frame}
%\frametitle{3. Parte ll: Eslabonamientos}
%\framesubtitle{Resultados}
%\begin{figure}
%   \caption{Clasificación de sectores según eslabonamientos. Tomada de Chenery y Watanabe (1958)}
%\includegraphics[width=10.0cm]{Graficas/esla}
%\centering
%\label{fig:ejemplo}
%\end{figure} 
%\end{frame}
%
%\begin{frame}
%\frametitle{3. Parte ll: Eslabonamientos}
%\framesubtitle{Resultados}
%\begin{figure}
%   \caption{Eslabonamientos puros totales(En miles de millones de pesos de 2010)}
%\includegraphics[width=10.0cm]{Graficas/eslabonamientos}
%\centering
%\label{fig:ejemplo}
%\end{figure} 
%\end{frame}
%
%
%\begin{frame}
%\frametitle{3. Parte ll: Eslabonamientos}
%\framesubtitle{Resultados}
%\begin{figure}
%   %\caption{Correlación entre multiplicadores}
%%\put(-100,50){\footnotesize{efkrewfire}}
%\includegraphics[scale=0.4]{Graficas/e1}
%\includegraphics[scale=0.4]{Graficas/e2} 
%  
%%https://www.youtube.com/watch?v=8sA2z1Qkqtw&t=316s
%
%\end{figure}
%  
%%\end{figure} 
%\end{frame}
%
%
%
%\section{Conclusiones}
%
%\begin{frame}
%\frametitle{Contenido}
%\begin{enumerate}[<i->]
%\item Introducción 
%\begin{enumerate} [a]
%\item Matriz Insumo-Producto
%\item Revisión de la literatura
%\item Construcción matriz Insumo-Producto
%\item Regionalización de la matriz de coeficientes técnicos
%\end{enumerate}
%\item Parte l: Multiplicadores
%\begin{enumerate} [a]
%
%\item Construcción  multiplicadores producto, empleo e ingreso
%\item Resultados
%\end{enumerate}
%\item Parte ll: Eslabonamientos
%\begin{enumerate} [a]
%\item Construcción de eslabonamientos hacia atrás y hacia adelante
%\item Resultados
%\end{enumerate}
%\item<1-> Conclusiones
%
%\end{enumerate}
%\end{frame}
%
%
%
%
%
%
%
%
%
%
%
%
%
%
%
%
%
%
%%\begin{frame}
%%\frametitle{3. Resultados}
%%\begin{figure}
% %  \caption{Sector independiente (miles de millones)}
%%\includegraphics[width=10.0cm]{Graficas/independiente}%12
%%\centering
%%\label{fig:ejemplo}
%%\end{figure} 
%%\end{frame}
%
%%\begin{frame}
%%\frametitle{3. Resultados}
%%\begin{figure}
% %  \caption{Sector impulsor (miles de millones)}
%%\includegraphics[width=10.0cm]{Graficas/impulsor}%12
%%\centering
%%\label{fig:ejemplo}
%%\end{figure} 
%%\end{frame}
%
%%\begin{frame}
%%\frametitle{3. Resultados}
%%\begin{figure}
% %  \caption{Sector clave (miles de millones)}
%%\includegraphics[width=10.0cm]{Graficas/clave}
%%\centering
%%\label{fig:ejemplo}
%%\end{figure} 
%%\end{frame}
%
%
%
%
%
%
%
%
%
%\begin{frame}
%\frametitle{4. Conclusiones}
%\begin{itemize}
%\justifying
%\item Importancia de sectores agrícola y ganadero tan solo en producción, sin resultados deseados en empleo e ingresos. Un sector industrial que emerge como el de mayores eslabonamientos, a pesar del bajo nivel de industrialización del departamento.
%\item Bajos multiplicadores en la mayoría de los ramas lo cual demuestra la baja diversificación de la economía y atraso respecto a otros departamentos. 
%\item Poca integración de la economía, esto se denota por la cantidad exagerada de sectores independientes y los insipientes sectores clave 
%\end{itemize}
%\end{frame}
%
%\begin{frame}
%\frametitle{4. Conclusiones}
%\begin{itemize}
%\justifying
%\item Para palear un poco esta situación, se deberían tomar medidas tales como aumentar la inversión en aquellos sectores que resulten importantes para el progreso de la economía, específicamente los sectores clave e impulsores. Con el objetivo de diversificarlos y que generar impactos positivos en el resto de la estructura económica.
%\item Mejoras en infraestructura vial y de comunicaciones contribuirán a hacer la economía más competitiva, ayudando a integrar los distintos sectores, ya sea por reducción de costos o porque especialmente el sector de transporte por carretera emerge como un sector clave que “conecta” actividades de diferentes ramas productivas.
% 
%\end{itemize}
%\end{frame}
%
%
%\begin{frame}
%\frametitle{Referencias}
%\begin{itemize}
%
%\justifying
%\item Bonet, J. A. (2000). La matriz insumo-producto del Caribe colombiano (Documento de Economía Regional, 15). Banco de la República de Colombia.
%\item Banguero, H., Duque, H., Garizado, P.,  Parra, D. (2006). Estimación de la matriz insumo-producto simétrica para el Valle del Cauca: año 1994. Cali: Universidad Autónoma de Occidente.
%\item Banco de la República de Colombia, Departamento Administrativo Nacional de Estadística. (2012). Informe de Coyuntura Económica Regional Departamento de Boyacá. Tunja: DANE - Banco de la República.
%\item Chenery, H. B.,  Watanabe, T. (1958). International comparisons of the structure of production. Econometrica, 26(4), 487-521.
%
%\end{itemize}
%\end{frame}
%
%
%\begin{frame}
%\frametitle{Referencias}
%\begin{itemize}
%\justifying
%\item Hernández, G. (2012). Matrices insumo-producto y análisis de multiplicadores: una aplicación para Colombia. Revista de Economía Institucional,14(26), 203-221
%\item Isard, W.,  Kuenne, R. E. (1953). The impact of steel upon the Greater New York-Philadelphia urban industrial region. The Review of Economics
%and Statistics, 35(4), 289-301.
%
%\item Leontief, W. (1953). Interregional theory. En W. W. Leontief et al. (eds.), Studies in the structure of the American Economy (pp. 93-115). Nueva York: Oxford University Press.
%
%\item Flegg, A. T.,  Tohmo, T. (2013). Regional input-output tables and the FLQ formula: A case study of Finland. Regional Studies, 47(5), 703-721.
%\item Villa, G.,  Giraldo, S. (2014). La economía de Medellín vista desde sus indicadores económicos intersectoriales 
%
%
%\end{itemize}
%\end{frame}
%%http://minisconlatex.blogspot.com/2010/11/ecuaciones.html (Latex matrices

\end{document}